\batchmode %% Suppresses most terminal output.
\documentclass{article}
\usepackage{color}
\definecolor{boxshade}{gray}{0.85}
\setlength{\textwidth}{360pt}
\setlength{\textheight}{541pt}
\usepackage{latexsym}
\usepackage{ifthen}
% \usepackage{color}
%%%%%%%%%%%%%%%%%%%%%%%%%%%%%%%%%%%%%%%%%%%%%%%%%%%%%%%%%%%%%%%%%%%%%%%%%%%%%
% SWITCHES                                                                  %
%%%%%%%%%%%%%%%%%%%%%%%%%%%%%%%%%%%%%%%%%%%%%%%%%%%%%%%%%%%%%%%%%%%%%%%%%%%%%
\newboolean{shading} 
\setboolean{shading}{false}
\makeatletter
 %% this is needed only when inserted into the file, not when
 %% used as a package file.
%%%%%%%%%%%%%%%%%%%%%%%%%%%%%%%%%%%%%%%%%%%%%%%%%%%%%%%%%%%%%%%%%%%%%%%%%%%%%
%                                                                           %
% DEFINITIONS OF SYMBOL-PRODUCING COMMANDS                                  %
%                                                                           %
%    TLA+      LaTeX                                                        %
%    symbol    command                                                      %
%    ------    -------                                                      %
%    =>        \implies                                                     %
%    <:        \ltcolon                                                     %
%    :>        \colongt                                                     %
%    ==        \defeq                                                       %
%    ..        \dotdot                                                      %
%    ::        \coloncolon                                                  %
%    =|        \eqdash                                                      %
%    ++        \pp                                                          %
%    --        \mm                                                          %
%    **        \stst                                                        %
%    //        \slsl                                                        %
%    ^         \ct                                                          %
%    \A        \A                                                           %
%    \E        \E                                                           %
%    \AA       \AA                                                          %
%    \EE       \EE                                                          %
%%%%%%%%%%%%%%%%%%%%%%%%%%%%%%%%%%%%%%%%%%%%%%%%%%%%%%%%%%%%%%%%%%%%%%%%%%%%%
\newlength{\symlength}
\newcommand{\implies}{\Rightarrow}
\newcommand{\ltcolon}{\mathrel{<\!\!\mbox{:}}}
\newcommand{\colongt}{\mathrel{\!\mbox{:}\!\!>}}
\newcommand{\defeq}{\;\mathrel{\smash   %% keep this symbol from being too tall
    {{\stackrel{\scriptscriptstyle\Delta}{=}}}}\;}
\newcommand{\dotdot}{\mathrel{\ldotp\ldotp}}
\newcommand{\coloncolon}{\mathrel{::\;}}
\newcommand{\eqdash}{\mathrel = \joinrel \hspace{-.28em}|}
\newcommand{\pp}{\mathbin{++}}
\newcommand{\mm}{\mathbin{--}}
\newcommand{\stst}{*\!*}
\newcommand{\slsl}{/\!/}
\newcommand{\ct}{\hat{\hspace{.4em}}}
\newcommand{\A}{\forall}
\newcommand{\E}{\exists}
\renewcommand{\AA}{\makebox{$\raisebox{.05em}{\makebox[0pt][l]{%
   $\forall\hspace{-.517em}\forall\hspace{-.517em}\forall$}}%
   \forall\hspace{-.517em}\forall \hspace{-.517em}\forall\,$}}
\newcommand{\EE}{\makebox{$\raisebox{.05em}{\makebox[0pt][l]{%
   $\exists\hspace{-.517em}\exists\hspace{-.517em}\exists$}}%
   \exists\hspace{-.517em}\exists\hspace{-.517em}\exists\,$}}
\newcommand{\whileop}{\.{\stackrel
  {\mbox{\raisebox{-.3em}[0pt][0pt]{$\scriptscriptstyle+\;\,$}}}%
  {-\hspace{-.16em}\triangleright}}}

% Commands are defined to produce the upper-case keywords.
% Note that some have space after them.
\newcommand{\ASSUME}{\textsc{assume }}
\newcommand{\ASSUMPTION}{\textsc{assumption }}
\newcommand{\AXIOM}{\textsc{axiom }}
\newcommand{\BOOLEAN}{\textsc{boolean }}
\newcommand{\CASE}{\textsc{case }}
\newcommand{\CONSTANT}{\textsc{constant }}
\newcommand{\CONSTANTS}{\textsc{constants }}
\newcommand{\ELSE}{\settowidth{\symlength}{\THEN}%
   \makebox[\symlength][l]{\textsc{ else}}}
\newcommand{\EXCEPT}{\textsc{ except }}
\newcommand{\EXTENDS}{\textsc{extends }}
\newcommand{\FALSE}{\textsc{false}}
\newcommand{\IF}{\textsc{if }}
\newcommand{\IN}{\settowidth{\symlength}{\LET}%
   \makebox[\symlength][l]{\textsc{in}}}
\newcommand{\INSTANCE}{\textsc{instance }}
\newcommand{\LET}{\textsc{let }}
\newcommand{\LOCAL}{\textsc{local }}
\newcommand{\MODULE}{\textsc{module }}
\newcommand{\OTHER}{\textsc{other }}
\newcommand{\STRING}{\textsc{string}}
\newcommand{\THEN}{\textsc{ then }}
\newcommand{\THEOREM}{\textsc{theorem }}
\newcommand{\LEMMA}{\textsc{lemma }}
\newcommand{\PROPOSITION}{\textsc{proposition }}
\newcommand{\COROLLARY}{\textsc{corollary }}
\newcommand{\TRUE}{\textsc{true}}
\newcommand{\VARIABLE}{\textsc{variable }}
\newcommand{\VARIABLES}{\textsc{variables }}
\newcommand{\WITH}{\textsc{ with }}
\newcommand{\WF}{\textrm{WF}}
\newcommand{\SF}{\textrm{SF}}
\newcommand{\CHOOSE}{\textsc{choose }}
\newcommand{\ENABLED}{\textsc{enabled }}
\newcommand{\UNCHANGED}{\textsc{unchanged }}
\newcommand{\SUBSET}{\textsc{subset }}
\newcommand{\UNION}{\textsc{union }}
\newcommand{\DOMAIN}{\textsc{domain }}
% Added for tla2tex
\newcommand{\BY}{\textsc{by }}
\newcommand{\OBVIOUS}{\textsc{obvious }}
\newcommand{\HAVE}{\textsc{have }}
\newcommand{\QED}{\textsc{qed }}
\newcommand{\TAKE}{\textsc{take }}
\newcommand{\DEF}{\textsc{ def }}
\newcommand{\HIDE}{\textsc{hide }}
\newcommand{\RECURSIVE}{\textsc{recursive }}
\newcommand{\USE}{\textsc{use }}
\newcommand{\DEFINE}{\textsc{define }}
\newcommand{\PROOF}{\textsc{proof }}
\newcommand{\WITNESS}{\textsc{witness }}
\newcommand{\PICK}{\textsc{pick }}
\newcommand{\DEFS}{\textsc{defs }}
\newcommand{\PROVE}{\settowidth{\symlength}{\ASSUME}%
   \makebox[\symlength][l]{\textsc{prove}}\@s{-4.1}}%
  %% The \@s{-4.1) is a kludge added on 24 Oct 2009 [happy birthday, Ellen]
  %% so the correct alignment occurs if the user types
  %%   ASSUME X
  %%   PROVE  Y
  %% because it cancels the extra 4.1 pts added because of the 
  %% extra space after the PROVE.  This seems to works OK.
  %% However, the 4.1 equals Parameters.LaTeXLeftSpace(1) and
  %% should be changed if that method ever changes.
\newcommand{\SUFFICES}{\textsc{suffices }}
\newcommand{\NEW}{\textsc{new }}
\newcommand{\LAMBDA}{\textsc{lambda }}
\newcommand{\STATE}{\textsc{state }}
\newcommand{\ACTION}{\textsc{action }}
\newcommand{\TEMPORAL}{\textsc{temporal }}
\newcommand{\ONLY}{\textsc{only }}              %% added by LL on 2 Oct 2009
\newcommand{\OMITTED}{\textsc{omitted }}        %% added by LL on 31 Oct 2009
\newcommand{\@pfstepnum}[2]{\ensuremath{\langle#1\rangle}\textrm{#2}}
\newcommand{\bang}{\@s{1}\mbox{\small !}\@s{1}}
%% We should format || differently in PlusCal code than in TLA+ formulas.
\newcommand{\p@barbar}{\ifpcalsymbols
   \,\,\rule[-.25em]{.075em}{1em}\hspace*{.2em}\rule[-.25em]{.075em}{1em}\,\,%
   \else \,||\,\fi}
%% PlusCal keywords
\newcommand{\p@fair}{\textbf{fair }}
\newcommand{\p@semicolon}{\textbf{\,; }}
\newcommand{\p@algorithm}{\textbf{algorithm }}
\newcommand{\p@mmfair}{\textbf{-{}-fair }}
\newcommand{\p@mmalgorithm}{\textbf{-{}-algorithm }}
\newcommand{\p@assert}{\textbf{assert }}
\newcommand{\p@await}{\textbf{await }}
\newcommand{\p@begin}{\textbf{begin }}
\newcommand{\p@end}{\textbf{end }}
\newcommand{\p@call}{\textbf{call }}
\newcommand{\p@define}{\textbf{define }}
\newcommand{\p@do}{\textbf{ do }}
\newcommand{\p@either}{\textbf{either }}
\newcommand{\p@or}{\textbf{or }}
\newcommand{\p@goto}{\textbf{goto }}
\newcommand{\p@if}{\textbf{if }}
\newcommand{\p@then}{\,\,\textbf{then }}
\newcommand{\p@else}{\ifcsyntax \textbf{else } \else \,\,\textbf{else }\fi}
\newcommand{\p@elsif}{\,\,\textbf{elsif }}
\newcommand{\p@macro}{\textbf{macro }}
\newcommand{\p@print}{\textbf{print }}
\newcommand{\p@procedure}{\textbf{procedure }}
\newcommand{\p@process}{\textbf{process }}
\newcommand{\p@return}{\textbf{return}}
\newcommand{\p@skip}{\textbf{skip}}
\newcommand{\p@variable}{\textbf{variable }}
\newcommand{\p@variables}{\textbf{variables }}
\newcommand{\p@while}{\textbf{while }}
\newcommand{\p@when}{\textbf{when }}
\newcommand{\p@with}{\textbf{with }}
\newcommand{\p@lparen}{\textbf{(\,\,}}
\newcommand{\p@rparen}{\textbf{\,\,) }}   
\newcommand{\p@lbrace}{\textbf{\{\,\,}}   
\newcommand{\p@rbrace}{\textbf{\,\,\} }}

%%%%%%%%%%%%%%%%%%%%%%%%%%%%%%%%%%%%%%%%%%%%%%%%%%%%%%%%%
% REDEFINE STANDARD COMMANDS TO MAKE THEM FORMAT BETTER %
%                                                       %
% We redefine \in and \notin                            %
%%%%%%%%%%%%%%%%%%%%%%%%%%%%%%%%%%%%%%%%%%%%%%%%%%%%%%%%%
\renewcommand{\_}{\rule{.4em}{.06em}\hspace{.05em}}
\newlength{\equalswidth}
\let\oldin=\in
\let\oldnotin=\notin
\renewcommand{\in}{%
   {\settowidth{\equalswidth}{$\.{=}$}\makebox[\equalswidth][c]{$\oldin$}}}
\renewcommand{\notin}{%
   {\settowidth{\equalswidth}{$\.{=}$}\makebox[\equalswidth]{$\oldnotin$}}}


%%%%%%%%%%%%%%%%%%%%%%%%%%%%%%%%%%%%%%%%%%%%%%%%%%%%
%                                                  %
% HORIZONTAL BARS:                                 %
%                                                  %
%   \moduleLeftDash    |~~~~~~~~~~                 %
%   \moduleRightDash    ~~~~~~~~~~|                %
%   \midbar            |----------|                %
%   \bottombar         |__________|                %
%%%%%%%%%%%%%%%%%%%%%%%%%%%%%%%%%%%%%%%%%%%%%%%%%%%%
\newlength{\charwidth}\settowidth{\charwidth}{{\small\tt M}}
\newlength{\boxrulewd}\setlength{\boxrulewd}{.4pt}
\newlength{\boxlineht}\setlength{\boxlineht}{.5\baselineskip}
\newcommand{\boxsep}{\charwidth}
\newlength{\boxruleht}\setlength{\boxruleht}{.5ex}
\newlength{\boxruledp}\setlength{\boxruledp}{-\boxruleht}
\addtolength{\boxruledp}{\boxrulewd}
\newcommand{\boxrule}{\leaders\hrule height \boxruleht depth \boxruledp
                      \hfill\mbox{}}
\newcommand{\@computerule}{%
  \setlength{\boxruleht}{.5ex}%
  \setlength{\boxruledp}{-\boxruleht}%
  \addtolength{\boxruledp}{\boxrulewd}}

\newcommand{\bottombar}{\hspace{-\boxsep}%
  \raisebox{-\boxrulewd}[0pt][0pt]{\rule[.5ex]{\boxrulewd}{\boxlineht}}%
  \boxrule
  \raisebox{-\boxrulewd}[0pt][0pt]{%
      \rule[.5ex]{\boxrulewd}{\boxlineht}}\hspace{-\boxsep}\vspace{0pt}}

\newcommand{\moduleLeftDash}%
   {\hspace*{-\boxsep}%
     \raisebox{-\boxlineht}[0pt][0pt]{\rule[.5ex]{\boxrulewd
               }{\boxlineht}}%
    \boxrule\hspace*{.4em }}

\newcommand{\moduleRightDash}%
    {\hspace*{.4em}\boxrule
    \raisebox{-\boxlineht}[0pt][0pt]{\rule[.5ex]{\boxrulewd
               }{\boxlineht}}\hspace{-\boxsep}}%\vspace{.2em}

\newcommand{\midbar}{\hspace{-\boxsep}\raisebox{-.5\boxlineht}[0pt][0pt]{%
   \rule[.5ex]{\boxrulewd}{\boxlineht}}\boxrule\raisebox{-.5\boxlineht%
   }[0pt][0pt]{\rule[.5ex]{\boxrulewd}{\boxlineht}}\hspace{-\boxsep}}

%%%%%%%%%%%%%%%%%%%%%%%%%%%%%%%%%%%%%%%%%%%%%%%%%%%%%%%%%%%%%%%%%%%%%%%%%%%%%
% FORMATING COMMANDS                                                        %
%%%%%%%%%%%%%%%%%%%%%%%%%%%%%%%%%%%%%%%%%%%%%%%%%%%%%%%%%%%%%%%%%%%%%%%%%%%%%

%%%%%%%%%%%%%%%%%%%%%%%%%%%%%%%%%%%%%%%%%%%%%%%%%%%%%%%%%%%%%%%%%%%%%%%%%%%%%
% PLUSCAL SHADING                                                           %
%%%%%%%%%%%%%%%%%%%%%%%%%%%%%%%%%%%%%%%%%%%%%%%%%%%%%%%%%%%%%%%%%%%%%%%%%%%%%

% The TeX pcalshading switch is set on to cause PlusCal shading to be
% performed.  This changes the behavior of the following commands and
% environments to cause full-width shading to be performed on all lines.
% 
%   \tstrut \@x cpar mcom \@pvspace
% 
% The TeX pcalsymbols switch is turned on when typesetting a PlusCal algorithm,
% whether or not shading is being performed.  It causes symbols (other than
% parentheses and braces and PlusCal-only keywords) that should be typeset
% differently depending on whether they are in an algorithm to be typeset
% appropriately.  Currently, the only such symbol is "||".
%
% The TeX csyntax switch is turned on when typesetting a PlusCal algorithm in
% c-syntax.  This allows symbols to be format differently in the two syntaxes.
% The "else" keyword is the only one that is.

\newif\ifpcalshading \pcalshadingfalse
\newif\ifpcalsymbols \pcalsymbolsfalse
\newif\ifcsyntax     \csyntaxtrue

% The \@pvspace command makes a vertical space.  It uses \vspace
% except with \ifpcalshading, in which case it sets \pvcalvspace
% and the space is added by a following \@x command.
%
\newlength{\pcalvspace}\setlength{\pcalvspace}{0pt}%
\newcommand{\@pvspace}[1]{%
  \ifpcalshading
     \par\global\setlength{\pcalvspace}{#1}%
  \else
     \par\vspace{#1}%
  \fi
}

% The lcom environment was changed to set \lcomindent equal to
% the indentation it produces.  This length is used by the
% cpar environment to make shading extend for the full width
% of the line.  This assumes that lcom environments are not
% nested.  I hope TLATeX does not nest them.
%
\newlength{\lcomindent}%
\setlength{\lcomindent}{0pt}%

%\tstrut: A strut to produce inter-paragraph space in a comment.
%\rstrut: A strut to extend the bottom of a one-line comment so
%         there's no break in the shading between comments on 
%         successive lines.
\newcommand\tstrut%
  {\raisebox{\vshadelen}{\raisebox{-.25em}{\rule{0pt}{1.15em}}}%
   \global\setlength{\vshadelen}{0pt}}
\newcommand\rstrut{\raisebox{-.25em}{\rule{0pt}{1.15em}}%
 \global\setlength{\vshadelen}{0pt}}


% \.{op} formats operator op in math mode with empty boxes on either side.
% Used because TeX otherwise vary the amount of space it leaves around op.
\renewcommand{\.}[1]{\ensuremath{\mbox{}#1\mbox{}}}

% \@s{n} produces an n-point space
\newcommand{\@s}[1]{\hspace{#1pt}}           

% \@x{txt} starts a specification line in the beginning with txt
% in the final LaTeX source.
\newlength{\@xlen}
\newcommand\xtstrut%
  {\setlength{\@xlen}{1.05em}%
   \addtolength{\@xlen}{\pcalvspace}%
    \raisebox{\vshadelen}{\raisebox{-.25em}{\rule{0pt}{\@xlen}}}%
   \global\setlength{\vshadelen}{0pt}%
   \global\setlength{\pcalvspace}{0pt}}

\newcommand{\@x}[1]{\par
  \ifpcalshading
  \makebox[0pt][l]{\shadebox{\xtstrut\hspace*{\textwidth}}}%
  \fi
  \mbox{$\mbox{}#1\mbox{}$}}  

% \@xx{txt} continues a specification line with the text txt.
\newcommand{\@xx}[1]{\mbox{$\mbox{}#1\mbox{}$}}  

% \@y{cmt} produces a one-line comment.
\newcommand{\@y}[1]{\mbox{\footnotesize\hspace{.65em}%
  \ifthenelse{\boolean{shading}}{%
      \shadebox{#1\hspace{-\the\lastskip}\rstrut}}%
               {#1\hspace{-\the\lastskip}\rstrut}}}

% \@z{cmt} produces a zero-width one-line comment.
\newcommand{\@z}[1]{\makebox[0pt][l]{\footnotesize
  \ifthenelse{\boolean{shading}}{%
      \shadebox{#1\hspace{-\the\lastskip}\rstrut}}%
               {#1\hspace{-\the\lastskip}\rstrut}}}


% \@w{str} produces the TLA+ string "str".
\newcommand{\@w}[1]{\textsf{``{#1}''}}             


%%%%%%%%%%%%%%%%%%%%%%%%%%%%%%%%%%%%%%%%%%%%%%%%%%%%%%%%%%%%%%%%%%%%%%%%%%%%%
% SHADING                                                                   %
%%%%%%%%%%%%%%%%%%%%%%%%%%%%%%%%%%%%%%%%%%%%%%%%%%%%%%%%%%%%%%%%%%%%%%%%%%%%%
\def\graymargin{1}
  % The number of points of margin in the shaded box.

% \definecolor{boxshade}{gray}{.85}
% Defines the darkness of the shading: 1 = white, 0 = black
% Added by TLATeX only if needed.

% \shadebox{txt} puts txt in a shaded box.
\newlength{\templena}
\newlength{\templenb}
\newsavebox{\tempboxa}
\newcommand{\shadebox}[1]{{\setlength{\fboxsep}{\graymargin pt}%
     \savebox{\tempboxa}{#1}%
     \settoheight{\templena}{\usebox{\tempboxa}}%
     \settodepth{\templenb}{\usebox{\tempboxa}}%
     \hspace*{-\fboxsep}\raisebox{0pt}[\templena][\templenb]%
        {\colorbox{boxshade}{\usebox{\tempboxa}}}\hspace*{-\fboxsep}}}

% \vshade{n} makes an n-point inter-paragraph space, with
%  shading if the `shading' flag is true.
\newlength{\vshadelen}
\setlength{\vshadelen}{0pt}
\newcommand{\vshade}[1]{\ifthenelse{\boolean{shading}}%
   {\global\setlength{\vshadelen}{#1pt}}%
   {\vspace{#1pt}}}

\newlength{\boxwidth}
\newlength{\multicommentdepth}

%%%%%%%%%%%%%%%%%%%%%%%%%%%%%%%%%%%%%%%%%%%%%%%%%%%%%%%%%%%%%%%%%%%%%%%%%%%%%
% THE cpar ENVIRONMENT                                                      %
% ^^^^^^^^^^^^^^^^^^^^                                                      %
% The LaTeX input                                                           %
%                                                                           %
%   \begin{cpar}{pop}{nest}{isLabel}{d}{e}{arg6}                            %
%     XXXXXXXXXXXXXXX                                                       %
%     XXXXXXXXXXXXXXX                                                       %
%     XXXXXXXXXXXXXXX                                                       %
%   \end{cpar}                                                              %
%                                                                           %
% produces one of two possible results.  If isLabel is the letter "T",      %
% it produces the following, where [label] is the result of typesetting     %
% arg6 in an LR box, and d is is a number representing a distance in        %
% points.                                                                   %
%                                                                           %
%   prevailing |<-- d -->[label]<- e ->XXXXXXXXXXXXXXX                      %
%         left |                       XXXXXXXXXXXXXXX                      %
%       margin |                       XXXXXXXXXXXXXXX                      %
%                                                                           %
% If isLabel is the letter "F", then it produces                            %
%                                                                           %
%   prevailing |<-- d -->XXXXXXXXXXXXXXXXXXXXXXX                            %
%         left |         <- e ->XXXXXXXXXXXXXXXX                            %
%       margin |                XXXXXXXXXXXXXXXX                            %
%                                                                           %
% where d and e are numbers representing distances in points.               %
%                                                                           %
% The prevailing left margin is the one in effect before the most recent    %
% pop (argument 1) cpar environments with "T" as the nest argument, where   %
% pop is a number \geq 0.                                                   %
%                                                                           %
% If the nest argument is the letter "T", then the prevailing left          %
% margin is moved to the left of the second (and following) lines of        %
% X's.  Otherwise, the prevailing left margin is left unchanged.            %
%                                                                           %
% An \unnest{n} command moves the prevailing left margin to where it was    %
% before the most recent n cpar environments with "T" as the nesting        %
% argument.                                                                 %
%                                                                           %
% The environment leaves no vertical space above or below it, or between    %
% its paragraphs.  (TLATeX inserts the proper amount of vertical space.)    %
%%%%%%%%%%%%%%%%%%%%%%%%%%%%%%%%%%%%%%%%%%%%%%%%%%%%%%%%%%%%%%%%%%%%%%%%%%%%%

\newcounter{pardepth}
\setcounter{pardepth}{0}

% \setgmargin{txt} defines \gmarginN to be txt, where N is \roman{pardepth}.
% \thegmargin equals \gmarginN, where N is \roman{pardepth}.
\newcommand{\setgmargin}[1]{%
  \expandafter\xdef\csname gmargin\roman{pardepth}\endcsname{#1}}
\newcommand{\thegmargin}{\csname gmargin\roman{pardepth}\endcsname}
\newcommand{\gmargin}{0pt}

\newsavebox{\tempsbox}

\newlength{\@cparht}
\newlength{\@cpardp}
\newenvironment{cpar}[6]{%
  \addtocounter{pardepth}{-#1}%
  \ifthenelse{\boolean{shading}}{\par\begin{lrbox}{\tempsbox}%
                                 \begin{minipage}[t]{\linewidth}}{}%
  \begin{list}{}{%
     \edef\temp{\thegmargin}
     \ifthenelse{\equal{#3}{T}}%
       {\settowidth{\leftmargin}{\hspace{\temp}\footnotesize #6\hspace{#5pt}}%
        \addtolength{\leftmargin}{#4pt}}%
       {\setlength{\leftmargin}{#4pt}%
        \addtolength{\leftmargin}{#5pt}%
        \addtolength{\leftmargin}{\temp}%
        \setlength{\itemindent}{-#5pt}}%
      \ifthenelse{\equal{#2}{T}}{\addtocounter{pardepth}{1}%
                                 \setgmargin{\the\leftmargin}}{}%
      \setlength{\labelwidth}{0pt}%
      \setlength{\labelsep}{0pt}%
      \setlength{\itemindent}{-\leftmargin}%
      \setlength{\topsep}{0pt}%
      \setlength{\parsep}{0pt}%
      \setlength{\partopsep}{0pt}%
      \setlength{\parskip}{0pt}%
      \setlength{\itemsep}{0pt}
      \setlength{\itemindent}{#4pt}%
      \addtolength{\itemindent}{-\leftmargin}}%
   \ifthenelse{\equal{#3}{T}}%
      {\item[\tstrut\footnotesize \hspace{\temp}{#6}\hspace{#5pt}]
        }%
      {\item[\tstrut\hspace{\temp}]%
         }%
   \footnotesize}
 {\hspace{-\the\lastskip}\tstrut
 \end{list}%
  \ifthenelse{\boolean{shading}}%
          {\end{minipage}%
           \end{lrbox}%
           \ifpcalshading
             \setlength{\@cparht}{\ht\tempsbox}%
             \setlength{\@cpardp}{\dp\tempsbox}%
             \addtolength{\@cparht}{.15em}%
             \addtolength{\@cpardp}{.2em}%
             \addtolength{\@cparht}{\@cpardp}%
            % I don't know what's going on here.  I want to add a
            % \pcalvspace high shaded line, but I don't know how to
            % do it.  A little trial and error shows that the following
            % does a reasonable job approximating that, eliminating
            % the line if \pcalvspace is small.
            \addtolength{\@cparht}{\pcalvspace}%
             \ifdim \pcalvspace > .8em
               \addtolength{\pcalvspace}{-.2em}%
               \hspace*{-\lcomindent}%
               \shadebox{\rule{0pt}{\pcalvspace}\hspace*{\textwidth}}\par
               \global\setlength{\pcalvspace}{0pt}%
               \fi
             \hspace*{-\lcomindent}%
             \makebox[0pt][l]{\raisebox{-\@cpardp}[0pt][0pt]{%
                 \shadebox{\rule{0pt}{\@cparht}\hspace*{\textwidth}}}}%
             \hspace*{\lcomindent}\usebox{\tempsbox}%
             \par
           \else
             \shadebox{\usebox{\tempsbox}}\par
           \fi}%
           {}%
  }

%%%%%%%%%%%%%%%%%%%%%%%%%%%%%%%%%%%%%%%%%%%%%%%%%%%%%%%%%%%%%%%%%%%%%%%%%%%%%%
% THE ppar ENVIRONMENT                                                       %
% ^^^^^^^^^^^^^^^^^^^^                                                       %
% The environment                                                            %
%                                                                            %
%   \begin{ppar} ... \end{ppar}                                              %
%                                                                            %
% is equivalent to                                                           %
%                                                                            %
%   \begin{cpar}{0}{F}{F}{0}{0}{} ... \end{cpar}                             %
%                                                                            %
% The environment is put around each line of the output for a PlusCal        %
% algorithm.                                                                 %
%%%%%%%%%%%%%%%%%%%%%%%%%%%%%%%%%%%%%%%%%%%%%%%%%%%%%%%%%%%%%%%%%%%%%%%%%%%%%%
%\newenvironment{ppar}{%
%  \ifthenelse{\boolean{shading}}{\par\begin{lrbox}{\tempsbox}%
%                                 \begin{minipage}[t]{\linewidth}}{}%
%  \begin{list}{}{%
%     \edef\temp{\thegmargin}
%        \setlength{\leftmargin}{0pt}%
%        \addtolength{\leftmargin}{\temp}%
%        \setlength{\itemindent}{0pt}%
%      \setlength{\labelwidth}{0pt}%
%      \setlength{\labelsep}{0pt}%
%      \setlength{\itemindent}{-\leftmargin}%
%      \setlength{\topsep}{0pt}%
%      \setlength{\parsep}{0pt}%
%      \setlength{\partopsep}{0pt}%
%      \setlength{\parskip}{0pt}%
%      \setlength{\itemsep}{0pt}
%      \setlength{\itemindent}{0pt}%
%      \addtolength{\itemindent}{-\leftmargin}}%
%      \item[\tstrut\hspace{\temp}]}%
% {\hspace{-\the\lastskip}\tstrut
% \end{list}%
%  \ifthenelse{\boolean{shading}}{\end{minipage}  
%                                 \end{lrbox}%
%                                 \shadebox{\usebox{\tempsbox}}\par}{}%
%  }

 %%% TESTING
 \newcommand{\xtest}[1]{\par
 \makebox[0pt][l]{\shadebox{\xtstrut\hspace*{\textwidth}}}%
 \mbox{$\mbox{}#1\mbox{}$}} 

% \newcommand{\xxtest}[1]{\par
% \makebox[0pt][l]{\shadebox{\xtstrut{#1}\hspace*{\textwidth}}}%
% \mbox{$\mbox{}#1\mbox{}$}} 

%\newlength{\pcalvspace}
%\setlength{\pcalvspace}{0pt}
% \newlength{\xxtestlen}
% \setlength{\xxtestlen}{0pt}
% \newcommand\xtstrut%
%   {\setlength{\xxtestlen}{1.15em}%
%    \addtolength{\xxtestlen}{\pcalvspace}%
%     \raisebox{\vshadelen}{\raisebox{-.25em}{\rule{0pt}{\xxtestlen}}}%
%    \global\setlength{\vshadelen}{0pt}%
%    \global\setlength{\pcalvspace}{0pt}}
   
   %%%% TESTING
   
   %% The xcpar environment
   %%  Note: overloaded use of \pcalvspace for testing.
   %%
%   \newlength{\xcparht}%
%   \newlength{\xcpardp}%
   
%   \newenvironment{xcpar}[6]{%
%  \addtocounter{pardepth}{-#1}%
%  \ifthenelse{\boolean{shading}}{\par\begin{lrbox}{\tempsbox}%
%                                 \begin{minipage}[t]{\linewidth}}{}%
%  \begin{list}{}{%
%     \edef\temp{\thegmargin}%
%     \ifthenelse{\equal{#3}{T}}%
%       {\settowidth{\leftmargin}{\hspace{\temp}\footnotesize #6\hspace{#5pt}}%
%        \addtolength{\leftmargin}{#4pt}}%
%       {\setlength{\leftmargin}{#4pt}%
%        \addtolength{\leftmargin}{#5pt}%
%        \addtolength{\leftmargin}{\temp}%
%        \setlength{\itemindent}{-#5pt}}%
%      \ifthenelse{\equal{#2}{T}}{\addtocounter{pardepth}{1}%
%                                 \setgmargin{\the\leftmargin}}{}%
%      \setlength{\labelwidth}{0pt}%
%      \setlength{\labelsep}{0pt}%
%      \setlength{\itemindent}{-\leftmargin}%
%      \setlength{\topsep}{0pt}%
%      \setlength{\parsep}{0pt}%
%      \setlength{\partopsep}{0pt}%
%      \setlength{\parskip}{0pt}%
%      \setlength{\itemsep}{0pt}%
%      \setlength{\itemindent}{#4pt}%
%      \addtolength{\itemindent}{-\leftmargin}}%
%   \ifthenelse{\equal{#3}{T}}%
%      {\item[\xtstrut\footnotesize \hspace{\temp}{#6}\hspace{#5pt}]%
%        }%
%      {\item[\xtstrut\hspace{\temp}]%
%         }%
%   \footnotesize}
% {\hspace{-\the\lastskip}\tstrut
% \end{list}%
%  \ifthenelse{\boolean{shading}}{\end{minipage}  
%                                 \end{lrbox}%
%                                 \setlength{\xcparht}{\ht\tempsbox}%
%                                 \setlength{\xcpardp}{\dp\tempsbox}%
%                                 \addtolength{\xcparht}{.15em}%
%                                 \addtolength{\xcpardp}{.2em}%
%                                 \addtolength{\xcparht}{\xcpardp}%
%                                 \hspace*{-\lcomindent}%
%                                 \makebox[0pt][l]{\raisebox{-\xcpardp}[0pt][0pt]{%
%                                      \shadebox{\rule{0pt}{\xcparht}\hspace*{\textwidth}}}}%
%                                 \hspace*{\lcomindent}\usebox{\tempsbox}%
%                                 \par}{}%
%  }
%  
% \newlength{\xmcomlen}
%\newenvironment{xmcom}[1]{%
%  \setcounter{pardepth}{0}%
%  \hspace{.65em}%
%  \begin{lrbox}{\alignbox}\sloppypar%
%      \setboolean{shading}{false}%
%      \setlength{\boxwidth}{#1pt}%
%      \addtolength{\boxwidth}{-.65em}%
%      \begin{minipage}[t]{\boxwidth}\footnotesize
%      \parskip=0pt\relax}%
%       {\end{minipage}\end{lrbox}%
%       \setlength{\xmcomlen}{\textwidth}%
%       \addtolength{\xmcomlen}{-\wd\alignbox}%
%       \settodepth{\alignwidth}{\usebox{\alignbox}}%
%       \global\setlength{\multicommentdepth}{\alignwidth}%
%       \setlength{\boxwidth}{\alignwidth}%
%       \global\addtolength{\alignwidth}{-\maxdepth}%
%       \addtolength{\boxwidth}{.1em}%
%       \raisebox{0pt}[0pt][0pt]{%
%        \ifthenelse{\boolean{shading}}%
%          {\hspace*{-\xmcomlen}\shadebox{\rule[-\boxwidth]{0pt}{0pt}%
%                                 \hspace*{\xmcomlen}\usebox{\alignbox}}}%
%          {\usebox{\alignbox}}}%
%       \vspace*{\alignwidth}\pagebreak[0]\vspace{-\alignwidth}\par}
% % a multi-line comment, whose first argument is its width in points.
%  
   
%%%%%%%%%%%%%%%%%%%%%%%%%%%%%%%%%%%%%%%%%%%%%%%%%%%%%%%%%%%%%%%%%%%%%%%%%%%%%%
% THE lcom ENVIRONMENT                                                       %
% ^^^^^^^^^^^^^^^^^^^^                                                       %
% A multi-line comment with no text to its left is typeset in an lcom        % 
% environment, whose argument is a number representing the indentation       % 
% of the left margin, in points.  All the text of the comment should be      % 
% inside cpar environments.                                                  % 
%%%%%%%%%%%%%%%%%%%%%%%%%%%%%%%%%%%%%%%%%%%%%%%%%%%%%%%%%%%%%%%%%%%%%%%%%%%%%%
\newenvironment{lcom}[1]{%
  \setlength{\lcomindent}{#1pt} % Added for PlusCal handling.
  \par\vspace{.2em}%
  \sloppypar
  \setcounter{pardepth}{0}%
  \footnotesize
  \begin{list}{}{%
    \setlength{\leftmargin}{#1pt}
    \setlength{\labelwidth}{0pt}%
    \setlength{\labelsep}{0pt}%
    \setlength{\itemindent}{0pt}%
    \setlength{\topsep}{0pt}%
    \setlength{\parsep}{0pt}%
    \setlength{\partopsep}{0pt}%
    \setlength{\parskip}{0pt}}
    \item[]}%
  {\end{list}\vspace{.3em}\setlength{\lcomindent}{0pt}%
 }


%%%%%%%%%%%%%%%%%%%%%%%%%%%%%%%%%%%%%%%%%%%%%%%%%%%%%%%%%%%%%%%%%%%%%%%%%%%%%
% THE mcom ENVIRONMENT AND \mutivspace COMMAND                              %
% ^^^^^^^^^^^^^^^^^^^^^^^^^^^^^^^^^^^^^^^^^^^^                              %
%                                                                           %
% A part of the spec containing a right-comment of the form                 %
%                                                                           %
%      xxxx (*************)                                                 %
%      yyyy (* ccccccccc *)                                                 %
%      ...  (* ccccccccc *)                                                 %
%           (* ccccccccc *)                                                 %
%           (* ccccccccc *)                                                 %
%           (*************)                                                 %
%                                                                           %
% is typeset by                                                             %
%                                                                           %
%     XXXX \begin{mcom}{d}                                                  %
%            CCCC ... CCC                                                   %
%          \end{mcom}                                                       %
%     YYYY ...                                                              %
%     \multivspace{n}                                                       %
%                                                                           %
% where the number d is the width in points of the comment, n is the        %
% number of xxxx, yyyy, ...  lines to the left of the comment.              %
% All the text of the comment should be typeset in cpar environments.       %
%                                                                           %
% This puts the comment into a single box (so no page breaks can occur      %
% within it).  The entire box is shaded iff the shading flag is true.       %
%%%%%%%%%%%%%%%%%%%%%%%%%%%%%%%%%%%%%%%%%%%%%%%%%%%%%%%%%%%%%%%%%%%%%%%%%%%%%
\newlength{\xmcomlen}%
\newenvironment{mcom}[1]{%
  \setcounter{pardepth}{0}%
  \hspace{.65em}%
  \begin{lrbox}{\alignbox}\sloppypar%
      \setboolean{shading}{false}%
      \setlength{\boxwidth}{#1pt}%
      \addtolength{\boxwidth}{-.65em}%
      \begin{minipage}[t]{\boxwidth}\footnotesize
      \parskip=0pt\relax}%
       {\end{minipage}\end{lrbox}%
       \setlength{\xmcomlen}{\textwidth}%       % For PlusCal shading
       \addtolength{\xmcomlen}{-\wd\alignbox}%  % For PlusCal shading
       \settodepth{\alignwidth}{\usebox{\alignbox}}%
       \global\setlength{\multicommentdepth}{\alignwidth}%
       \setlength{\boxwidth}{\alignwidth}%      % For PlusCal shading
       \global\addtolength{\alignwidth}{-\maxdepth}%
       \addtolength{\boxwidth}{.1em}%           % For PlusCal shading
      \raisebox{0pt}[0pt][0pt]{%
        \ifthenelse{\boolean{shading}}%
          {\ifpcalshading
             \hspace*{-\xmcomlen}%
             \shadebox{\rule[-\boxwidth]{0pt}{0pt}\hspace*{\xmcomlen}%
                          \usebox{\alignbox}}%
           \else
             \shadebox{\usebox{\alignbox}}
           \fi
          }%
          {\usebox{\alignbox}}}%
       \vspace*{\alignwidth}\pagebreak[0]\vspace{-\alignwidth}\par}
 % a multi-line comment, whose first argument is its width in points.


% \multispace{n} produces the vertical space indicated by "|"s in 
% this situation
%   
%     xxxx (*************)
%     xxxx (* ccccccccc *)
%      |   (* ccccccccc *)
%      |   (* ccccccccc *)
%      |   (* ccccccccc *)
%      |   (*************)
%
% where n is the number of "xxxx" lines.
\newcommand{\multivspace}[1]{\addtolength{\multicommentdepth}{-#1\baselineskip}%
 \addtolength{\multicommentdepth}{1.2em}%
 \ifthenelse{\lengthtest{\multicommentdepth > 0pt}}%
    {\par\vspace{\multicommentdepth}\par}{}}

%\newenvironment{hpar}[2]{%
%  \begin{list}{}{\setlength{\leftmargin}{#1pt}%
%                 \addtolength{\leftmargin}{#2pt}%
%                 \setlength{\itemindent}{-#2pt}%
%                 \setlength{\topsep}{0pt}%
%                 \setlength{\parsep}{0pt}%
%                 \setlength{\partopsep}{0pt}%
%                 \setlength{\parskip}{0pt}%
%                 \addtolength{\labelsep}{0pt}}%
%  \item[]\footnotesize}{\end{list}}
%    %%%%%%%%%%%%%%%%%%%%%%%%%%%%%%%%%%%%%%%%%%%%%%%%%%%%%%%%%%%%%%%%%%%%%%%%
%    % Typesets a sequence of paragraphs like this:                         %
%    %                                                                      %
%    %      left |<-- d1 --> XXXXXXXXXXXXXXXXXXXXXXXX                       %
%    %    margin |           <- d2 -> XXXXXXXXXXXXXXX                       %
%    %           |                    XXXXXXXXXXXXXXX                       %
%    %           |                                                          %
%    %           |                    XXXXXXXXXXXXXXX                       %
%    %           |                    XXXXXXXXXXXXXXX                       %
%    %                                                                      %
%    % where d1 = #1pt and d2 = #2pt, but with no vspace between            %
%    % paragraphs.                                                          %
%    %%%%%%%%%%%%%%%%%%%%%%%%%%%%%%%%%%%%%%%%%%%%%%%%%%%%%%%%%%%%%%%%%%%%%%%%

%%%%%%%%%%%%%%%%%%%%%%%%%%%%%%%%%%%%%%%%%%%%%%%%%%%%%%%%%%%%%%%%%%%%%%
% Commands for repeated characters that produce dashes.              %
%%%%%%%%%%%%%%%%%%%%%%%%%%%%%%%%%%%%%%%%%%%%%%%%%%%%%%%%%%%%%%%%%%%%%%
% \raisedDash{wd}{ht}{thk} makes a horizontal line wd characters wide, 
% raised a distance ht ex's above the baseline, with a thickness of 
% thk em's.
\newcommand{\raisedDash}[3]{\raisebox{#2ex}{\setlength{\alignwidth}{.5em}%
  \rule{#1\alignwidth}{#3em}}}

% The following commands take a single argument n and produce the
% output for n repeated characters, as follows
%   \cdash:    -
%   \tdash:    ~
%   \ceqdash:  =
%   \usdash:   _
\newcommand{\cdash}[1]{\raisedDash{#1}{.5}{.04}}
\newcommand{\usdash}[1]{\raisedDash{#1}{0}{.04}}
\newcommand{\ceqdash}[1]{\raisedDash{#1}{.5}{.08}}
\newcommand{\tdash}[1]{\raisedDash{#1}{1}{.08}}

\newlength{\spacewidth}
\setlength{\spacewidth}{.2em}
\newcommand{\e}[1]{\hspace{#1\spacewidth}}
%% \e{i} produces space corresponding to i input spaces.


%% Alignment-file Commands

\newlength{\alignboxwidth}
\newlength{\alignwidth}
\newsavebox{\alignbox}

% \al{i}{j}{txt} is used in the alignment file to put "%{i}{j}{wd}"
% in the log file, where wd is the width of the line up to that point,
% and txt is the following text.
\newcommand{\al}[3]{%
  \typeout{\%{#1}{#2}{\the\alignwidth}}%
  \cl{#3}}

%% \cl{txt} continues a specification line in the alignment file
%% with text txt.
\newcommand{\cl}[1]{%
  \savebox{\alignbox}{\mbox{$\mbox{}#1\mbox{}$}}%
  \settowidth{\alignboxwidth}{\usebox{\alignbox}}%
  \addtolength{\alignwidth}{\alignboxwidth}%
  \usebox{\alignbox}}

% \fl{txt} in the alignment file begins a specification line that
% starts with the text txt.
\newcommand{\fl}[1]{%
  \par
  \savebox{\alignbox}{\mbox{$\mbox{}#1\mbox{}$}}%
  \settowidth{\alignwidth}{\usebox{\alignbox}}%
  \usebox{\alignbox}}



  
%%%%%%%%%%%%%%%%%%%%%%%%%%%%%%%%%%%%%%%%%%%%%%%%%%%%%%%%%%%%%%%%%%%%%%%%%%%%%
% Ordinarily, TeX typesets letters in math mode in a special math italic    %
% font.  This makes it typeset "it" to look like the product of the         %
% variables i and t, rather than like the word "it".  The following         %
% commands tell TeX to use an ordinary italic font instead.                 %
%%%%%%%%%%%%%%%%%%%%%%%%%%%%%%%%%%%%%%%%%%%%%%%%%%%%%%%%%%%%%%%%%%%%%%%%%%%%%
\ifx\documentclass\undefined
\else
  \DeclareSymbolFont{tlaitalics}{\encodingdefault}{cmr}{m}{it}
  \let\itfam\symtlaitalics
\fi

\makeatletter
\newcommand{\tlx@c}{\c@tlx@ctr\advance\c@tlx@ctr\@ne}
\newcounter{tlx@ctr}
\c@tlx@ctr=\itfam \multiply\c@tlx@ctr"100\relax \advance\c@tlx@ctr "7061\relax
\mathcode`a=\tlx@c \mathcode`b=\tlx@c \mathcode`c=\tlx@c \mathcode`d=\tlx@c
\mathcode`e=\tlx@c \mathcode`f=\tlx@c \mathcode`g=\tlx@c \mathcode`h=\tlx@c
\mathcode`i=\tlx@c \mathcode`j=\tlx@c \mathcode`k=\tlx@c \mathcode`l=\tlx@c
\mathcode`m=\tlx@c \mathcode`n=\tlx@c \mathcode`o=\tlx@c \mathcode`p=\tlx@c
\mathcode`q=\tlx@c \mathcode`r=\tlx@c \mathcode`s=\tlx@c \mathcode`t=\tlx@c
\mathcode`u=\tlx@c \mathcode`v=\tlx@c \mathcode`w=\tlx@c \mathcode`x=\tlx@c
\mathcode`y=\tlx@c \mathcode`z=\tlx@c
\c@tlx@ctr=\itfam \multiply\c@tlx@ctr"100\relax \advance\c@tlx@ctr "7041\relax
\mathcode`A=\tlx@c \mathcode`B=\tlx@c \mathcode`C=\tlx@c \mathcode`D=\tlx@c
\mathcode`E=\tlx@c \mathcode`F=\tlx@c \mathcode`G=\tlx@c \mathcode`H=\tlx@c
\mathcode`I=\tlx@c \mathcode`J=\tlx@c \mathcode`K=\tlx@c \mathcode`L=\tlx@c
\mathcode`M=\tlx@c \mathcode`N=\tlx@c \mathcode`O=\tlx@c \mathcode`P=\tlx@c
\mathcode`Q=\tlx@c \mathcode`R=\tlx@c \mathcode`S=\tlx@c \mathcode`T=\tlx@c
\mathcode`U=\tlx@c \mathcode`V=\tlx@c \mathcode`W=\tlx@c \mathcode`X=\tlx@c
\mathcode`Y=\tlx@c \mathcode`Z=\tlx@c
\makeatother

%%%%%%%%%%%%%%%%%%%%%%%%%%%%%%%%%%%%%%%%%%%%%%%%%%%%%%%%%%
%                THE describe ENVIRONMENT                %
%%%%%%%%%%%%%%%%%%%%%%%%%%%%%%%%%%%%%%%%%%%%%%%%%%%%%%%%%%
%
%
% It is like the description environment except it takes an argument
% ARG that should be the text of the widest label.  It adjusts the
% indentation so each item with label LABEL produces
%%      LABEL             blah blah blah
%%      <- width of ARG ->blah blah blah
%%                        blah blah blah
\newenvironment{describe}[1]%
   {\begin{list}{}{\settowidth{\labelwidth}{#1}%
            \setlength{\labelsep}{.5em}%
            \setlength{\leftmargin}{\labelwidth}% 
            \addtolength{\leftmargin}{\labelsep}%
            \addtolength{\leftmargin}{\parindent}%
            \def\makelabel##1{\rm ##1\hfill}}%
            \setlength{\topsep}{0pt}}%% 
                % Sets \topsep to 0 to reduce vertical space above
                % and below embedded displayed equations
   {\end{list}}

%   For tlatex.TeX
\usepackage{verbatim}
\makeatletter
\def\tla{\let\%\relax%
         \@bsphack
         \typeout{\%{\the\linewidth}}%
             \let\do\@makeother\dospecials\catcode`\^^M\active
             \let\verbatim@startline\relax
             \let\verbatim@addtoline\@gobble
             \let\verbatim@processline\relax
             \let\verbatim@finish\relax
             \verbatim@}
\let\endtla=\@esphack

\let\pcal=\tla
\let\endpcal=\endtla
\let\ppcal=\tla
\let\endppcal=\endtla

% The tlatex environment is used by TLATeX.TeX to typeset TLA+.
% TLATeX.TLA starts its files by writing a \tlatex command.  This
% command/environment sets \parindent to 0 and defines \% to its
% standard definition because the writing of the log files is messed up
% if \% is defined to be something else.  It also executes
% \@computerule to determine the dimensions for the TLA horizonatl
% bars.
\newenvironment{tlatex}{\@computerule%
                        \setlength{\parindent}{0pt}%
                       \makeatletter\chardef\%=`\%}{}


% The notla environment produces no output.  You can turn a 
% tla environment to a notla environment to prevent tlatex.TeX from
% re-formatting the environment.

\def\notla{\let\%\relax%
         \@bsphack
             \let\do\@makeother\dospecials\catcode`\^^M\active
             \let\verbatim@startline\relax
             \let\verbatim@addtoline\@gobble
             \let\verbatim@processline\relax
             \let\verbatim@finish\relax
             \verbatim@}
\let\endnotla=\@esphack

\let\nopcal=\notla
\let\endnopcal=\endnotla
\let\noppcal=\notla
\let\endnoppcal=\endnotla

%%%%%%%%%%%%%%%%%%%%%%%% end of tlatex.sty file %%%%%%%%%%%%%%%%%%%%%%% 
% last modified on Fri  3 August 2012 at 14:23:49 PST by lamport

\begin{document}
\tlatex
\setboolean{shading}{true}
\@x{}\moduleLeftDash\@xx{ {\MODULE} CBCCasperSpec}\moduleRightDash\@xx{}%
\@x{ {\EXTENDS} FiniteSets ,\, Integers ,\, Sequences ,\, TLC}%
\@pvspace{8.0pt}%
\@x{ {\CONSTANTS}}%
\@x{\@s{8.2} nodes ,\,\@s{24.23}}%
\@y{\@s{0}%
 set of validator ids
}%
\@xx{}%
\@x{\@s{8.2} weights ,\,\@s{16.82}}%
\@y{\@s{0}%
 tuple of validator weights
}%
\@xx{}%
\@x{\@s{8.2} threshold ,\,\@s{9.32}}%
\@y{\@s{0}%
 fault tolerance threshold
}%
\@xx{}%
\@x{\@s{8.2} values ,\,\@s{21.93}}%
\@y{\@s{0}%
 set of consensus values
}%
\@xx{}%
\@x{\@s{8.2} genesis\@s{23.69}}%
\@y{\@s{0}%
 genesis message
}%
\@xx{}%
\@pvspace{8.0pt}%
\@x{ {\VARIABLES}}%
\@x{\@s{8.2} dags ,\,\@s{29.34}}%
\@y{\@s{0}%
 tuple of local \ensuremath{DAGs} for each validator (only contains parent
 pointers)
}%
\@xx{}%
\@x{\@s{8.2} faulty ,\,\@s{23.91}}%
\@y{\@s{0}%
 tuple of sets of observed equivocating validators
}%
\@xx{}%
\@x{\@s{8.2} scored\_q ,\,\@s{11.86}}%
\@y{\@s{0}%
 tuple of records of scored messages with score
}%
\@xx{}%
\@x{\@s{8.2} unscored\_q ,\,\@s{0.87}}%
\@y{\@s{0}%
 tuple of tuples of messages which have not been scored
}%
\@xx{}%
\@x{\@s{8.2} sent\_msgs ,\,\@s{4.21}}%
\@y{\@s{0}%
 tuple of tuples of messages sent by each validator
}%
\@xx{}%
\@x{\@s{8.2} equiv\_msgs ,\,}%
\@y{\@s{0}%
 tuple of sets of tuples of equivocated messages
}%
\@xx{}%
\@x{\@s{8.2} estimates ,\,\@s{7.87}}%
\@y{\@s{0}%
 tuple of sets of best current estimates
}%
\@xx{}%
\@x{\@s{8.2} states\@s{29.83}}%
\@y{\@s{0}%
 tuple of validator states (contains all justification pointers)
}%
\@xx{}%
\@pvspace{8.0pt}%
 \@x{ vars \.{\defeq} {\langle} faulty ,\, scored\_q ,\, unscored\_q ,\,
 sent\_msgs ,\, equiv\_msgs ,\, estimates ,\, states {\rangle}}%
\@pvspace{8.0pt}%
\@x{}\midbar\@xx{}%
\begin{lcom}{0}%
\begin{cpar}{0}{F}{F}{0}{0}{}%
Messages
\end{cpar}%
\end{lcom}%
\@x{}%
\@y{\@s{0}%
 Unscored message \ensuremath{\.{=}} (estimate, sender, justification)
}%
\@xx{}%
 \@x{ Msg ( est ,\, from ,\, just ) \.{\defeq} [ estimate \.{\mapsto} est ,\,
 sender \.{\mapsto} from ,\, justification \.{\mapsto} just ]}%
\@pvspace{8.0pt}%
\@x{}%
\@y{\@s{0}%
 Scored message.
}%
\@xx{}%
 \@x{ ScoredMsg ( \_msg ,\, \_score ) \.{\defeq} [ msg \.{\mapsto} \_msg ,\,
 score \.{\mapsto} \_score ]}%
\@pvspace{8.0pt}%
\@x{}%
\@y{\@s{0}%
 Scored estimate.
}%
\@xx{}%
 \@x{ ScoredEst ( \_est ,\, \_score ) \.{\defeq} [ est \.{\mapsto} \_est ,\,
 score \.{\mapsto} \_score ]}%
\@pvspace{8.0pt}%
\@x{}%
\@y{\@s{0}%
 Message decomposition functions.
}%
\@xx{}%
\@x{ Estimate ( msg ) \.{\defeq} msg . estimate}%
\@x{ Sender ( msg ) \.{\defeq} msg . sender}%
\@x{ Justification ( msg ) \.{\defeq} msg . justification}%
\@x{}%
\@y{%
 \ensuremath{\.{\LET} j \.{\defeq} msg.justification
}}%
\@xx{}%
\@x{}%
\@y{%
 \ensuremath{\.{\IN} j.parents \.{\cup} j.nonparents
}}%
\@xx{}%
\@x{}%
\@y{%
 \ensuremath{Just(p,\,n) \.{\defeq}} [parents \ensuremath{\.{\mapsto} p},
 \ensuremath{nonparents \.{\mapsto} n}]
}%
\@xx{}%
\@x{}%
\@y{%
 \ensuremath{OnlyPar(p) \.{\defeq}} [parents \ensuremath{\.{\mapsto} p},
 \ensuremath{nonparents \.{\mapsto} \{\}}]
}%
\@xx{}%
\@x{}%
\@y{%
 \ensuremath{Parents(msg) \.{\defeq} msg.justification.parents
}}%
\@xx{}%
\@pvspace{8.0pt}%
\@x{}%
\@y{\@s{0}%
 The genesis message is abstract - it does not have estimate, sender, or
 justification fields.
}%
\@xx{}%
\@pvspace{8.0pt}%
\@x{}%
\@y{\@s{0}%
 Set of nodes who have sent at least one message.
}%
\@xx{}%
 \@x{ Senders \.{\defeq} \{ n \.{\in} nodes \.{:} sent\_msgs [ n ] \.{\neq}
 {\langle} {\rangle} \}}%
 \@x{ Observed ( msgs ) \.{\defeq} \{ Sender ( m ) \.{:} m \.{\in} ( msgs
 \.{\,\backslash\,} \{ genesis \} ) \}}%
\@pvspace{8.0pt}%
\@x{}\midbar\@xx{}%
\begin{lcom}{0}%
\begin{cpar}{0}{F}{F}{0}{0}{}%
Estimator
\end{cpar}%
\end{lcom}%
\@x{}%
\@y{%
 \ensuremath{TODO}: make precise
}%
\@xx{}%
\@x{}%
\@y{\@s{0}%
 \ensuremath{GHOST} fork choice rule - latest honest estimate driven
}%
\@xx{}%
\@x{}%
\@y{\@s{0}%
 output should be ranked set of tips
}%
\@xx{}%
\@x{ GHOST ( state ) \.{\defeq}}%
\@x{\@s{8.2} {\IF} state \.{=} \{ genesis \}}%
\@x{\@s{8.2} \.{\THEN} values}%
 \@x{\@s{8.2} \.{\ELSE} \{ Estimate ( m ) \.{:} m \.{\in} ( state
 \.{\,\backslash\,} \{ genesis \} ) \}}%
\@pvspace{8.0pt}%
\@x{}\midbar\@xx{}%
\begin{lcom}{0}%
\begin{cpar}{0}{F}{F}{0}{0}{}%
Auxiliary functions from \ensuremath{SequencesExt} community module.
 See https:\ensuremath{\.{\slsl}github.com}/tlaplus/
\end{cpar}%
\end{lcom}%
\@x{ ToSet ( s ) \.{\defeq} \{ s [ i ] \.{:} i \.{\in} {\DOMAIN} s \}}%
\@pvspace{8.0pt}%
 \@x{ IsInjective ( f ) \.{\defeq} \A\, i ,\, j \.{\in} {\DOMAIN} f \.{:} ( f
 [ i ] \.{=} f [ j ] ) \.{\implies} ( i \.{=} j )}%
\@pvspace{8.0pt}%
 \@x{ SetToSeq ( S ) \.{\defeq} {\CHOOSE} f \.{\in} [ 1 \.{\dotdot}
 Cardinality ( S ) \.{\rightarrow} S ] \.{:} IsInjective ( f )}%
\@pvspace{8.0pt}%
 \@x{ Max ( S ) \.{\defeq} {\CHOOSE} n \.{\in} S \.{:} \A\, m \.{\in} S \.{:}
 m \.{\leq} n}%
\@pvspace{8.0pt}%
\@x{}\midbar\@xx{}%
\begin{lcom}{0}%
\begin{cpar}{0}{F}{F}{0}{0}{}%
Auxiliary Functions \& Definitions
\end{cpar}%
\end{lcom}%
\@x{}%
\@y{\@s{0}%
 Returns the tuple of \ensuremath{unscored} messages from tuple of scored
 messages.
}%
\@xx{}%
\@x{ {\RECURSIVE} Unscore ( \_ )}%
\@x{ Unscore ( seq ) \.{\defeq}}%
\@x{\@s{8.2} {\IF} seq \.{=} {\langle} {\rangle}}%
\@x{\@s{8.2} \.{\THEN} {\langle} {\rangle}}%
 \@x{\@s{8.2} \.{\ELSE} {\langle} Head ( seq ) . msg {\rangle} \.{\circ}
 Unscore ( Tail ( seq ) )}%
\@pvspace{8.0pt}%
\@x{}%
\@y{\@s{0}%
 Set of all messages received by a given validator.
}%
\@xx{}%
 \@x{ ReceivedMsgs ( n ) \.{\defeq} ToSet ( unscored\_q [ n ] \.{\circ}
 Unscore ( scored\_q [ n ] ) ) \.{\,\backslash\,} \{ genesis \}}%
\@pvspace{8.0pt}%
\@x{}%
\@y{\@s{0}%
 Pick an arbitrary element from the given set.
}%
\@xx{}%
\@x{ Pick ( S ) \.{\defeq} {\CHOOSE} s \.{\in} S \.{:} {\TRUE}}%
\@pvspace{8.0pt}%
\@x{}%
\@y{\@s{0}%
 Set of nodes who have received at least one message (excludes genesis).
}%
\@xx{}%
 \@x{ Receivers \.{\defeq} \{ n \.{\in} nodes \.{:} ReceivedMsgs ( n )
 \.{\neq} \{ \} \}}%
\@pvspace{8.0pt}%
\@x{}%
\@y{\@s{0}%
 Broadcast given message to all other validators in given set.
}%
\@xx{}%
\@x{}%
\@y{\@s{0}%
 arguments: message, sender, set of receivers (sender is excluded)
}%
\@xx{}%
\@x{ Broadcast ( msg ,\, n ,\, rec ) \.{\defeq}}%
 \@x{\@s{8.2} [ i \.{\in} nodes \.{\mapsto} {\IF} i \.{\in} ( rec
 \.{\,\backslash\,} \{ n \} )}%
\@x{\@s{68.79} \.{\THEN} {\langle} msg {\rangle}}%
\@x{\@s{68.79} \.{\ELSE} {\langle} {\rangle}}%
\@x{\@s{8.2} ]}%
\@pvspace{8.0pt}%
\@x{}%
\@y{\@s{0}%
 Apply binary operation over entire set.
}%
\@xx{}%
\@x{ {\RECURSIVE} SetReduce ( \_ ,\, \_ ,\, \_ )}%
\@x{ SetReduce ( Op ( \_ ,\, \_ ) ,\, S ,\, value ) \.{\defeq}}%
\@x{\@s{8.2} {\IF} S \.{=} \{ \}}%
\@x{\@s{8.2} \.{\THEN} value}%
\@x{\@s{8.2} \.{\ELSE} \.{\LET} s \.{\defeq} Pick ( S )}%
 \@x{\@s{39.51} \.{\IN} SetReduce ( Op ,\, S \.{\,\backslash\,} \{ s \} ,\, Op
 ( s ,\, value ) )}%
\@pvspace{8.0pt}%
 \@x{ SetSum ( S ) \.{\defeq} \.{\LET} op ( a ,\, b ) \.{\defeq} a \.{+}
 b\@s{4.1} \.{\IN} SetReduce ( op ,\, S ,\, 0 )}%
 \@x{ SetAnd ( S )\@s{0.98} \.{\defeq} \.{\LET} op ( a ,\, b ) \.{\defeq} a
 \.{\land} b\@s{5.21} \.{\IN} SetReduce ( op ,\, S ,\, {\TRUE} )}%
 \@x{ SetOr ( S )\@s{6.97} \.{\defeq} \.{\LET} op ( a ,\, b ) \.{\defeq} a
 \.{\lor} b\@s{5.21} \.{\IN} SetReduce ( op ,\, S ,\, {\FALSE} )}%
\@pvspace{8.0pt}%
\@x{}%
\@y{\@s{0}%
 Apply binary operation over entire tuple.
}%
\@xx{}%
\@x{ {\RECURSIVE} SeqReduce ( \_ ,\, \_ ,\, \_ )}%
\@x{ SeqReduce ( Op ( \_ ,\, \_ ) ,\, s ,\, value ) \.{\defeq}}%
\@x{\@s{8.2} {\IF} s \.{=} {\langle} {\rangle}}%
\@x{\@s{8.2} \.{\THEN} value}%
\@x{\@s{8.2} \.{\ELSE} \.{\LET} h \.{\defeq} Head ( s )}%
 \@x{\@s{39.51} \.{\IN} SeqReduce ( Op ,\, Tail ( s ) ,\, Op ( h ,\, value )
 )}%
\@pvspace{8.0pt}%
 \@x{ SeqSum ( S ) \.{\defeq} \.{\LET} op ( a ,\, b ) \.{\defeq} a \.{+}
 b\@s{4.1} \.{\IN} SeqReduce ( op ,\, S ,\, 0 )}%
 \@x{ SeqAnd ( S )\@s{0.98} \.{\defeq} \.{\LET} op ( a ,\, b ) \.{\defeq} a
 \.{\land} b\@s{5.21} \.{\IN} SeqReduce ( op ,\, S ,\, {\TRUE} )}%
 \@x{ SeqOr ( S )\@s{6.97} \.{\defeq} \.{\LET} op ( a ,\, b ) \.{\defeq} a
 \.{\lor} b\@s{5.21} \.{\IN} SeqReduce ( op ,\, S ,\, {\FALSE} )}%
\@pvspace{8.0pt}%
\@x{}%
\@y{\@s{0}%
 Turns a set of 2-tuples into the set of individual elements.
}%
\@xx{}%
\@x{ {\RECURSIVE} UnSeqSet ( \_ )}%
\@x{ UnSeqSet ( S ) \.{\defeq}}%
\@x{\@s{8.2} {\IF} S \.{=} \{ \}}%
\@x{\@s{8.2} \.{\THEN} \{ \}}%
\@x{\@s{8.2} \.{\ELSE} \.{\LET} s \.{\defeq} Pick ( S )}%
 \@x{\@s{39.51} \.{\IN} \{ Head ( s ) ,\, Head ( Tail ( s ) ) \} \.{\cup}
 UnSeqSet ( S \.{\,\backslash\,} \{ s \} )}%
\@pvspace{8.0pt}%
\@x{}%
\@y{\@s{0}%
 Turns set of elements into the set of all possible 2-tuples.
}%
\@xx{}%
 \@x{ Pairs ( S ) \.{\defeq} \{ s \.{\in} Seq ( S ) \.{:} Len ( s ) \.{=} 2
 \}}%
\@pvspace{8.0pt}%
\@x{}%
\@y{\@s{0}%
 Global set of faulty validators.
}%
\@xx{}%
\@x{ GlobalFaultySet \.{\defeq} {\UNION} ( ToSet ( faulty ) )}%
\@pvspace{8.0pt}%
\@x{}%
\@y{\@s{0}%
 Initialize tuple with given value.
}%
\@xx{}%
 \@x{ Initialize ( val ) \.{\defeq} [ i \.{\in} 1 \.{\dotdot} Cardinality (
 nodes ) \.{\mapsto} val ]}%
\@pvspace{8.0pt}%
\@x{}\midbar\@xx{}%
\@x{}%
\@y{\@s{0}%
 The dependencies of a message \ensuremath{m} are the messages in the
 justification
}%
\@xx{}%
\@x{}%
\@y{\@s{0}%
 of \ensuremath{m} and in the justifications of the justifications of
 \ensuremath{m} and so on,
}%
\@xx{}%
\@x{}%
\@y{\@s{0}%
 \ensuremath{i.e}. justifications all the way down.
}%
\@xx{}%
\@x{ {\RECURSIVE} Dep ( \_ )}%
\@x{ Dep ( msg ) \.{\defeq}}%
\@x{\@s{8.2} {\IF} msg \.{=} genesis}%
\@x{\@s{8.2} \.{\THEN} \{ genesis \}}%
\@x{\@s{8.2} \.{\ELSE} {\IF} Cardinality ( Justification ( msg ) ) \.{=} 1}%
\@x{\@s{39.51} \.{\THEN} Justification ( msg )}%
\@x{\@s{39.51} \.{\ELSE} Justification ( msg )}%
 \@x{\@s{70.82} \.{\cup} {\UNION} \{ Dep ( m ) \.{:} m \.{\in} ( Justification
 ( msg ) \.{\,\backslash\,} \{ genesis \} ) \}}%
\@pvspace{8.0pt}%
\@x{}%
\@y{\@s{0}%
 Gets the set of dependencies of all the messages in a set of messages.
}%
\@xx{}%
 \@x{ DepSet ( msgs ) \.{\defeq} {\UNION} \{ Dep ( m ) \.{:} m \.{\in} msgs
 \}}%
\@pvspace{8.0pt}%
\@x{}%
\@y{\@s{0}%
 Dependency depth of a message.
}%
\@xx{}%
\@x{ {\RECURSIVE} Depth ( \_ )}%
\@x{ Depth ( msg ) \.{\defeq}}%
\@x{\@s{8.2} {\IF} msg \.{=} genesis}%
\@x{\@s{8.2} \.{\THEN} 0}%
 \@x{\@s{8.2} \.{\ELSE} 1 \.{+} Max ( \{ Depth ( m ) \.{:} m \.{\in} Dep ( msg
 ) \} )}%
\@pvspace{8.0pt}%
\@x{}%
\@y{\@s{0}%
 Dependency depth of a set of messages.
}%
\@xx{}%
\@x{ DepthSet ( msgs ) \.{\defeq}}%
\@x{\@s{8.2} {\IF} msgs \.{=} \{ genesis \}}%
\@x{\@s{8.2} \.{\THEN} 0}%
\@x{\@s{8.2} \.{\ELSE} Max ( \{ Depth ( m ) \.{:} m \.{\in} msgs \} )}%
\@pvspace{8.0pt}%
\@x{}%
\@y{\@s{0}%
 Latest \ensuremath{message(s)} from a validator in a given set of messages.
}%
\@xx{}%
\@x{ LatestMsgs ( n ,\, msgs ) \.{\defeq} \{ genesis \} \.{\cup}}%
\@x{\@s{8.2} \{ m \.{\in} ( msgs \.{\,\backslash\,} \{ genesis \} ) \.{:}}%
\@x{\@s{22.14} \.{\land} Sender ( m ) \.{=} n}%
 \@x{\@s{22.14} \.{\land} {\lnot} \E\, m0 \.{\in} ( msgs \.{\,\backslash\,} \{
 genesis \} ) \.{:}}%
\@x{\@s{33.25} \.{\land} Sender ( m0 ) \.{=} n}%
\@x{\@s{33.25} \.{\land} m \.{\neq} m0}%
\@x{\@s{33.25} \.{\land} m \.{\in} Dep ( m0 )}%
\@x{\@s{8.2} \}}%
\@pvspace{8.0pt}%
\@x{}%
\@y{\@s{0}%
 Latest \ensuremath{estimate(s)} from a validator in a given set of messages.
}%
\@xx{}%
 \@x{ LatestEsts ( n ,\, msgs ) \.{\defeq} \{ Estimate ( m ) \.{:} m \.{\in} (
 LatestMsgs ( n ,\, msgs ) \.{\,\backslash\,} \{ genesis \} ) \}}%
\@pvspace{8.0pt}%
\@x{}%
\@y{\@s{0}%
 Set of estimates in a state.
}%
\@xx{}%
 \@x{ Estimates ( state ) \.{\defeq} \{ Estimate ( m ) \.{:} m \.{\in} ( (
 state \.{\cup} DepSet ( state ) ) \.{\,\backslash\,} \{ genesis \} ) \}}%
\@pvspace{8.0pt}%
\@x{}%
\@y{\@s{0}%
 Justifications of a set of messages.
}%
\@xx{}%
 \@x{ Justifications ( msgs ) \.{\defeq} {\UNION} \{ Justification ( m ) \.{:}
 m \.{\in} ( msgs \.{\,\backslash\,} \{ genesis \} ) \}}%
\@pvspace{8.0pt}%
\@x{}%
\@y{\@s{0}%
 Two messages are equivocating if they have the same sender, but do not
 justify each other.
}%
\@xx{}%
\@x{ Equivocation ( m1 ,\, m2 ) \.{\defeq}}%
\@x{\@s{8.2} \.{\land} m1 \.{\neq} m2}%
\@x{\@s{8.2} \.{\land} Sender ( m1 ) \.{=} Sender ( m2 )}%
 \@x{\@s{8.2} \.{\land} m1 \.{\notin} ( Dep ( m2 ) \.{\,\backslash\,} \{
 genesis \} )}%
 \@x{\@s{8.2} \.{\land} m2 \.{\notin} ( Dep ( m1 ) \.{\,\backslash\,} \{
 genesis \} )}%
\@pvspace{8.0pt}%
\@x{ CheckDepsForEquiv ( msgs ) \.{\defeq}}%
\@x{\@s{8.2} \.{\LET} deps \.{\defeq} DepSet ( msgs )}%
\@x{\@s{8.2} \.{\IN} \.{\land} Cardinality ( deps ) \.{>} 1}%
 \@x{\@s{28.59} \.{\land} \E\, m1 ,\, m2 \.{\in} ( deps \.{\,\backslash\,} \{
 genesis \} ) \.{:} Equivocation ( m1 ,\, m2 )}%
\@pvspace{8.0pt}%
\@x{ EquivPairsInDeps ( msgs ) \.{\defeq}}%
\@x{\@s{8.2} {\IF} {\lnot} CheckDepsForEquiv ( msgs )}%
\@x{\@s{8.2} \.{\THEN} \{ \}}%
 \@x{\@s{8.2} \.{\ELSE} \{ {\langle} m1 ,\, m2 {\rangle} \.{\in} ( DepSet (
 msgs ) \.{\,\backslash\,} \{ genesis \} ) \.{\times} ( DepSet ( msgs )
 \.{\,\backslash\,} \{ genesis \} ) \.{:} Equivocation ( m1 ,\, m2 ) \}}%
\@pvspace{8.0pt}%
\@x{}%
\@y{\@s{0}%
 A validator is faulty if it sends equivocating messages.
}%
\@xx{}%
\@x{}%
\@y{\@s{0}%
 Checks if a validator equivicates in a given set of messages.
}%
\@xx{}%
\@x{ FaultyNode ( n ,\, msgs ) \.{\defeq}}%
 \@x{\@s{8.2} \.{\land} \E\, m1 \.{\in} ( DepSet ( msgs ) \.{\,\backslash\,}
 \{ genesis \} ) \.{:}}%
 \@x{\@s{19.31} \.{\land} \E\, m2 \.{\in} ( DepSet ( msgs ) \.{\,\backslash\,}
 \{ genesis \} ) \.{:}}%
\@x{\@s{30.42} \.{\land} Sender ( m1 ) \.{=} n}%
\@x{\@s{30.42} \.{\land} Equivocation ( m1 ,\, m2 )}%
\@pvspace{8.0pt}%
\@x{}%
\@y{\@s{0}%
 Set of faulty validators in an observed set of messages.
}%
\@xx{}%
 \@x{ FaultyNodes ( msgs ) \.{\defeq} \{ n \.{\in} nodes \.{:} FaultyNode ( n
 ,\, msgs ) \}}%
\@pvspace{8.0pt}%
\@x{}%
\@y{\@s{0}%
 Messages from equivocating validators in a given set of messages.
}%
\@xx{}%
 \@x{ EquivocatedMsgs ( n ,\, msgs ) \.{\defeq} DepSet ( msgs ) \.{\cap}
 UnSeqSet ( equiv\_msgs [ n ] )}%
\@pvspace{8.0pt}%
\@x{}%
\@y{\@s{0}%
 Checks existence of equivocated messages received by the given validator.
}%
\@xx{}%
 \@x{ EquivReceived ( n ) \.{\defeq} \E\, {\langle} m1 ,\, m2 {\rangle}
 \.{\in} Pairs ( DepSet ( ReceivedMsgs ( n ) ) ) \.{:} Equivocation ( m1 ,\,
 m2 )}%
\@pvspace{8.0pt}%
\@x{}%
\@y{\@s{0}%
 Arbitrary node who has observed an equivocation.
}%
\@xx{}%
 \@x{ Equiv\_node \.{\defeq} {\CHOOSE} n \.{\in} nodes \.{:} EquivReceived ( n
 )}%
\@pvspace{8.0pt}%
\@x{}%
\@y{\@s{0}%
 Set of messages later than a given message in a given set of messages.
}%
\@xx{}%
 \@x{ Later ( msg ,\, msgs ) \.{\defeq} \{ m \.{\in} ( msgs \.{\,\backslash\,}
 \{ genesis \} ) \.{:} msg \.{\in} Justification ( m ) \}}%
\@pvspace{8.0pt}%
\@x{}%
\@y{\@s{0}%
 Honest messages - messages from non-faulty validators.
}%
\@xx{}%
 \@x{ HonestMsgs ( n ,\, msgs ) \.{\defeq} DepSet ( msgs ) \.{\,\backslash\,}
 ( EquivocatedMsgs ( n ,\, msgs ) \.{\cup} \{ genesis \} )}%
\@pvspace{8.0pt}%
\@x{}%
\@y{\@s{0}%
 Set of latest honest messages received by a validator.
}%
\@xx{}%
 \@x{ LatestHonestMsgs ( n ,\, msgs ) \.{\defeq} \{ m \.{\in} HonestMsgs ( n
 ,\, msgs ) \.{:} m \.{\in} LatestMsgs ( n ,\, msgs ) \}}%
\@pvspace{8.0pt}%
\@x{}%
\@y{\@s{0}%
 Set of latest honest estimates received by a validator.
}%
\@xx{}%
 \@x{ LatestHonestEsts ( n ,\, msgs ) \.{\defeq} \{ Estimate ( m ) \.{:} m
 \.{\in} LatestHonestMsgs ( n ,\, msgs ) \}}%
\@pvspace{8.0pt}%
\@x{}%
\@y{\@s{0}%
 Weights of subsets of validators.
}%
\@xx{}%
\@x{ Weight ( set ) \.{\defeq}}%
 \@x{\@s{8.2} SeqSum ( [ n \.{\in} 1 \.{\dotdot} Cardinality ( nodes )
 \.{\mapsto} {\IF} n \.{\in} set \.{\THEN} weights [ n ] \.{\ELSE} 0 ] )}%
\@pvspace{8.0pt}%
\@x{ TotalWeight \.{\defeq} Weight ( nodes )}%
\@x{ FaultWeight ( state ) \.{\defeq} Weight ( FaultyNodes ( state ) )}%
\@pvspace{8.0pt}%
\@x{}%
\@y{\@s{0}%
 Two validators are agreeing with each other on an estimate in a set of
 messages if:
}%
\@xx{}%
\@x{}%
\@y{\@s{2.5}%
 \ensuremath{\.{-} n1} has exactly one latest message in the set
}%
\@xx{}%
\@x{}%
\@y{\@s{2.5}%
 \ensuremath{\.{-} n2} has exactly one latest message in the justification of
 \ensuremath{n1}\mbox{'}s latest message
}%
\@xx{}%
\@x{}%
\@y{\@s{2.5}%
 - the estimates of these latest messages agree with the given estimate
}%
\@xx{}%
\@x{}%
\@y{\@s{0}%
 \ensuremath{i.e}. \ensuremath{n1} is not equivocating in the set of messages
 and
}%
\@xx{}%
\@x{}%
\@y{\@s{12.5}%
 \ensuremath{n2} is not equivocating in the justification of
 \ensuremath{n1}\mbox{'}s latest message
}%
\@xx{}%
\@x{ Agreeing ( n1 ,\, n2 ,\, estimate ,\, msgs ) \.{\defeq}}%
 \@x{\@s{8.2} \.{\LET} n1\_latest\_msg\@s{4.10} \.{\defeq} Pick ( LatestMsgs (
 n1 ,\, msgs ) )}%
 \@x{\@s{32.69} n2\_latest\_msg \.{\defeq} Pick ( LatestMsgs ( n2 ,\,
 Justification ( n1\_latest\_msg ) ) )}%
 \@x{\@s{8.2} \.{\IN}\@s{4.09} \.{\land} Cardinality ( LatestMsgs ( n1 ,\,
 msgs ) ) \.{=} 1}%
 \@x{\@s{32.69} \.{\land} Cardinality ( LatestMsgs ( n2 ,\, Justification (
 n1\_latest\_msg ) ) ) \.{=} 1}%
\@x{\@s{32.69} \.{\land} Estimate ( n1\_latest\_msg ) \.{=} estimate}%
\@x{\@s{32.69} \.{\land} Estimate ( n2\_latest\_msg ) \.{=} estimate}%
\@pvspace{8.0pt}%
\@x{}%
\@y{\@s{0}%
 Two validators are disagreeing with each other on an estimate in a set of
 messages if:
}%
\@xx{}%
\@x{}%
\@y{\@s{2.5}%
 \ensuremath{\.{-} n1} has exactly one latest message in messages
}%
\@xx{}%
\@x{}%
\@y{\@s{2.5}%
 \ensuremath{\.{-} n2} has exactly one latest message in the justification of
 \ensuremath{n1}\mbox{'}s latest message
}%
\@xx{}%
\@x{}%
\@y{\@s{2.5}%
 \ensuremath{\.{-} n2} has a new latest message that doens\mbox{'}t agree
 with the estimate
}%
\@xx{}%
\@x{ Disagreeing ( n1 ,\, n2 ,\, estimate ,\, msgs ) \.{\defeq}}%
\@x{\@s{8.2} \.{\land} Cardinality ( LatestMsgs ( n1 ,\, msgs ) ) \.{=} 1}%
 \@x{\@s{8.2} \.{\land} \.{\LET} n1\_latest\_msg \.{\defeq} Pick ( LatestMsgs
 ( n1 ,\, msgs ) )}%
 \@x{\@s{19.31} \.{\IN} \.{\land} Cardinality ( LatestMsgs ( n2 ,\,
 Justification ( n1\_latest\_msg ) ) ) \.{=} 1}%
 \@x{\@s{39.71} \.{\land} \.{\LET} n2\_latest\_msg \.{\defeq} Pick (
 LatestMsgs ( n2 ,\, Justification ( n1\_latest\_msg ) ) )}%
 \@x{\@s{50.82} \.{\IN} \E\, m \.{\in} msgs \.{:} \.{\land} n2\_latest\_msg
 \.{\in} Dep ( m )}%
\@x{\@s{130.83} \.{\land} estimate \.{\neq} Estimate ( m )}%
\@pvspace{8.0pt}%
\@x{}%
\@y{\@s{0}%
 An \ensuremath{e}-clique is a group of non-faulty nodes in a set of observed
 messages such that:
}%
\@xx{}%
\@x{}%
\@y{\@s{2.5}%
 - they mutually see each other agreeing with the given estimate in the given
 set of messages, and
}%
\@xx{}%
\@x{}%
\@y{\@s{2.5}%
 - they mutually cannot see each other disagreeing with the given estimate in
 the given set of messages.
}%
\@xx{}%
\@x{}%
\@y{\@s{0}%
 If nodes in an \ensuremath{e}-clique see each other agreeing on
 \ensuremath{e} and can\mbox{'}t see each other disagreeing on \ensuremath{e},
}%
\@xx{}%
\@x{}%
\@y{\@s{0}%
 then there does not exist any new message from inside the clique that will
 cause them to assign
}%
\@xx{}%
\@x{}%
\@y{\@s{0}%
 lower scores to \ensuremath{e}. Further, if the clique has more than half of
 the validators by weight,
}%
\@xx{}%
\@x{}%
\@y{\@s{0}%
 then no messages external to the clique can raise the scores these
 validators assign to
}%
\@xx{}%
\@x{}%
\@y{\@s{0}%
 a competing estimate to cause it to become larger than the score they assign
 to \ensuremath{e}.
}%
\@xx{}%
\@x{ Eclique ( estimate ,\, state ) \.{\defeq}}%
\@x{\@s{8.2} \{ sub \.{\in} {\SUBSET} ( nodes ) \.{:}}%
\@x{\@s{21.40} \.{\land} Cardinality ( sub ) \.{>} 1}%
\@x{\@s{21.40} \.{\land} \A\, n1 \.{\in} sub \.{:}}%
\@x{\@s{40.71} \A\, n2 \.{\in} ( sub \.{\,\backslash\,} \{ n1 \} ) \.{:}}%
\@x{\@s{48.91} \.{\land} Agreeing ( n1 ,\, n2 ,\, estimate ,\, state )}%
 \@x{\@s{48.91} \.{\land} {\lnot} Disagreeing ( n1 ,\, n2 ,\, estimate ,\,
 state )}%
\@x{\@s{48.91} \.{\land} {\lnot} FaultyNode ( n1 ,\, state )}%
\@x{\@s{48.91} \.{\land} {\lnot} FaultyNode ( n2 ,\, state )}%
\@x{\@s{8.2} \}}%
\@pvspace{8.0pt}%
\@x{}%
\@y{\@s{0}%
 Checks for existence of an \ensuremath{e}-clique with cumulative weight
 \ensuremath{\.{>} 50}\.{\%} of total validator weight.
}%
\@xx{}%
\@x{ EcliqueEstimateSafety ( estimate ,\, state ) \.{\defeq}}%
\@x{\@s{8.2}}%
\@y{\@s{0}%
 state is valid
}%
\@xx{}%
\@x{\@s{8.2} \E\, ec \.{\in} Eclique ( estimate ,\, state ) \.{:}}%
 \@x{\@s{16.4} 2 \.{*} SeqSum ( Weight ( ec ) ) \.{>} TotalWeight \.{+}
 threshold \.{-} FaultWeight ( state )}%
\@pvspace{8.0pt}%
\@x{}%
\@y{\@s{0}%
 Set of messges received from honest validators by a particular valiadtor.
}%
\@xx{}%
 \@x{ HonestReceivedMsgs ( n ) \.{\defeq} \{ m \.{\in} ReceivedMsgs ( n )
 \.{:} m \.{\notin} UnSeqSet ( equiv\_msgs [ n ] ) \}}%
\@pvspace{8.0pt}%
\@x{}%
\@y{\@s{0}%
 A temporal property checking that finality can eventually be reached.
}%
\@xx{}%
\@x{ CheckSafetyOracle \.{\defeq}}%
 \@x{\@s{8.2} \.{\LET} n \.{\defeq} RandomElement ( nodes \.{\,\backslash\,}
 GlobalFaultySet )}%
 \@x{\@s{8.2} \.{\IN} {\Diamond} ( \E\, v \.{\in} values \.{:}
 EcliqueEstimateSafety ( v ,\, HonestReceivedMsgs ( n ) ) )}%
\@pvspace{8.0pt}%
\@x{}\midbar\@xx{}%
\@x{}%
\@y{\@s{0}%
 Protocol Messages \& States
}%
\@xx{}%
\@x{}%
\@y{\@s{0}%
 Protocol messages have an estimate given by the estimator applied to the
 justification.
}%
\@xx{}%
\@x{ ValidMsg ( msg ) \.{\defeq}}%
\@x{\@s{8.2} \.{\lor} msg \.{=} genesis\@s{45.1}}%
\@y{\@s{0}%
 genesis is a valid message
}%
\@xx{}%
\@x{\@s{8.2} \.{\lor} \.{\land} msg \.{\neq} genesis\@s{33.98}}%
\@y{\@s{0}%
 non-genesis message is valid if sender and estimate are valid
}%
\@xx{}%
\@x{\@s{19.31} \.{\land} Sender ( msg ) \.{\in} nodes}%
 \@x{\@s{19.31} \.{\land} Estimate ( msg ) \.{\in} GHOST ( Justification ( msg
 ) )}%
\@pvspace{8.0pt}%
 \@x{ ProtocolMsgs \.{\defeq} \{ m \.{\in} {\UNION} ( \{ ToSet ( sent\_msgs [
 n ] ) \.{:} n \.{\in} nodes \} ) \.{:} ValidMsg ( m ) \}}%
\@pvspace{8.0pt}%
\@x{}%
\@y{\@s{0}%
 Protocol states are finite sets of protocol messages which contain
}%
\@xx{}%
\@x{}%
\@y{\@s{0}%
 their justifications and have fault weight less than the theshold.
}%
\@xx{}%
\@x{ ValidState ( state ) \.{\defeq}}%
\@x{\@s{8.2} \.{\lor} state \.{=} \{ genesis \}}%
 \@x{\@s{8.2} \.{\lor} \A\, m \.{\in} ( state \.{\,\backslash\,} \{ genesis \}
 ) \.{:}}%
\@x{\@s{19.31} \.{\land} Justification ( m ) \.{\subseteq} state}%
\@x{\@s{19.31} \.{\land} FaultWeight ( state ) \.{<} threshold}%
\@pvspace{8.0pt}%
\@x{ ProtocolStates \.{\defeq}}%
\@x{\@s{8.2} \{ s \.{\in} {\SUBSET} ( ProtocolMsgs ) \.{:}}%
\@x{\@s{18.10} \.{\land} ValidState ( s )}%
\@x{\@s{18.10} \.{\land} IsFiniteSet ( s )}%
\@x{\@s{8.2} \}}%
\@pvspace{8.0pt}%
 \@x{ SentSet\@s{4.1} \.{\defeq} {\UNION} ( \{ ToSet ( sent\_msgs [ n ] )
 \.{:} n \.{\in} nodes \} )}%
 \@x{ StateSet\@s{1.28} \.{\defeq} {\UNION} ( \{ states [ n ] \.{:} n \.{\in}
 nodes \} )}%
\@pvspace{8.0pt}%
\@x{}\midbar\@xx{}%
\@x{}%
\@y{\@s{0}%
 Decisions \& Consistency
}%
\@xx{}%
\@x{}%
\@y{\@s{0}%
 Futures of a given state.
}%
\@xx{}%
 \@x{ Futures ( state ) \.{\defeq} \{ s \.{\in} ProtocolStates \.{:} state
 \.{\subseteq} s \}}%
\@pvspace{8.0pt}%
\@x{}%
\@y{\@s{0}%
 Check whether a given property is decided in a given state.
}%
\@xx{}%
 \@x{ Decided ( prop ,\, state ) \.{\defeq} \A\, s \.{\in} Futures ( state )
 \.{:} prop [ s ]}%
\@pvspace{8.0pt}%
\@x{}%
\@y{\@s{0}%
 Decisions in a given state: set of properties which are decided in the state.
}%
\@xx{}%
\@x{ Decisions ( state ) \.{\defeq}}%
 \@x{\@s{8.2} \{ prop \.{\in} [ ProtocolStates \.{\rightarrow} \{ {\FALSE} ,\,
 {\TRUE} \} ] \.{:} Decided ( prop ,\, state ) \}}%
\@pvspace{8.0pt}%
\@x{}\midbar\@xx{}%
\@x{}%
\@y{\@s{0}%
 Previous messages.
}%
\@xx{}%
\@x{ PrevMsg ( msg ) \.{\defeq}}%
\@x{\@s{8.2} {\IF} Justification ( msg ) \.{=} \{ genesis \}}%
\@x{\@s{8.2} \.{\THEN} \{ genesis \}}%
\@x{\@s{8.2} \.{\ELSE} \{ genesis \} \.{\cup}}%
 \@x{\@s{39.51} {\UNION} \{ LatestMsgs ( n ,\, Justification ( msg ) ) \.{:} n
 \.{\in} Observed ( Justification ( msg ) ) \}}%
\@pvspace{8.0pt}%
\@x{}%
\@y{\@s{0}%
 Previous estimates.
}%
\@xx{}%
\@x{ PrevEst ( msg ) \.{\defeq}}%
\@x{\@s{8.2} {\IF} Justification ( msg ) \.{=} \{ genesis \}}%
\@x{\@s{8.2} \.{\THEN} \{ \}}%
 \@x{\@s{8.2} \.{\ELSE} {\UNION} \{ LatestEsts ( n ,\, Justification ( msg ) )
 \.{:} n \.{\in} Observed ( Justification ( msg ) ) \}}%
\@pvspace{8.0pt}%
\@x{}%
\@y{\@s{0}%
 Message ancestry.
}%
\@xx{}%
\@x{ {\RECURSIVE} n\_cestorMsg ( \_ ,\, \_ )}%
\@x{ n\_cestorMsg ( msg ,\, n ) \.{\defeq}}%
\@x{\@s{8.2} {\IF} n \.{=} 0 \.{\lor} msg \.{=} genesis}%
\@x{\@s{8.2} \.{\THEN} msg}%
 \@x{\@s{8.2} \.{\ELSE} {\UNION} ( n\_cestorMsg ( PrevMsg ( msg ) ,\, n \.{-}
 1 ) )}%
\@pvspace{8.0pt}%
\@x{}%
\@y{\@s{0}%
 Estimate ancestry.
}%
\@xx{}%
\@x{ {\RECURSIVE} n\_cestorEst ( \_ ,\, \_ )}%
\@x{ n\_cestorEst ( msg ,\, n ) \.{\defeq}}%
\@x{\@s{8.2} {\IF} msg \.{=} genesis}%
\@x{\@s{8.2} \.{\THEN} \{ \}}%
\@x{\@s{8.2} \.{\ELSE} {\IF} n \.{=} 0}%
\@x{\@s{39.51} \.{\THEN} msg}%
 \@x{\@s{39.51} \.{\ELSE} {\UNION} ( n\_cestorMsg ( PrevEst ( msg ) ,\, n
 \.{-} 1 ) )}%
\@pvspace{8.0pt}%
\@x{}%
\@y{\@s{0}%
 Block membership: \ensuremath{b1} is conatined in \ensuremath{b2}\mbox{'}s
 chain/\ensuremath{dag}.
}%
\@xx{}%
\@x{}%
\@y{%
 \ensuremath{Membership(b1,\,b2) \.{\defeq} \E\, n \.{\in} Nat} :
 \ensuremath{b1 \.{=} n\_cestor(b2,\,n)
}}%
\@xx{}%
\@x{ Membership ( m1 ,\, m2 ) \.{\defeq}}%
\@x{\@s{4.1} \.{\lor} m1 \.{=} genesis}%
\@x{\@s{4.1} \.{\lor} m1 \.{=} m2}%
\@x{\@s{4.1} \.{\lor} \.{\land} m1 \.{\neq} genesis}%
 \@x{\@s{15.21} \.{\land} Estimate ( m1 ) \.{\in} Estimates ( \{ m2 \}
 \.{\cup} Dep ( m2 ) )}%
\@pvspace{8.0pt}%
\@x{}%
\@y{\@s{0}%
 Set of validators supporting a given estimate in a \ensuremath{dag}.
}%
\@xx{}%
\@x{ {\RECURSIVE} Supporters ( \_ ,\, \_ )}%
\@x{ Supporters ( est ,\, state ) \.{\defeq}}%
 \@x{\@s{8.2} {\IF} state \.{=} \{ genesis \} \.{\lor} est \.{\notin}
 Estimates ( state )}%
\@x{\@s{8.2} \.{\THEN} \{ \}}%
 \@x{\@s{8.2} \.{\ELSE} \.{\LET} m \.{\defeq} Pick ( state \.{\,\backslash\,}
 \{ genesis \} )}%
\@x{\@s{39.51} \.{\IN} {\IF} est \.{\in} Estimates ( \{ m \} )}%
 \@x{\@s{59.91} \.{\THEN} \{ Sender ( m ) \} \.{\cup} Supporters ( est ,\,
 Justification ( m ) ) \.{\cup} Supporters ( est ,\, ( state
 \.{\,\backslash\,} \{ m \} ) )}%
 \@x{\@s{59.91} \.{\ELSE} Supporters ( est ,\, ( state \.{\,\backslash\,} \{ m
 \} ) )}%
\@pvspace{8.0pt}%
\@x{}%
\@y{\@s{0}%
 Score of a block (estimate) in a given state.
}%
\@xx{}%
\@x{}%
\@y{%
 \ensuremath{Score(msg,\,state) \.{\defeq}
}}%
\@xx{}%
\@x{}%
\@y{\@s{2.5}%
 \ensuremath{\.{\LET} S \.{\defeq} \{n \.{\in} nodes\.{:} \E\, m \.{\in}
 LatestHonestEsts(n,\,state) \.{:} Membership(msg,\,m)\}
}}%
\@xx{}%
\@x{}%
\@y{\@s{2.5}%
 \ensuremath{\.{\IN} SeqSum([n \.{\in} S \.{\mapsto} weights[n]])
}}%
\@xx{}%
\@pvspace{8.0pt}%
 \@x{ Score ( est ,\, state ) \.{\defeq} Weight ( Supporters ( est ,\, state )
 \.{\,\backslash\,} GlobalFaultySet )}%
\@pvspace{8.0pt}%
\@x{}%
\@y{\@s{0}%
 Children: a child of a block has that block as (one of) its
 \ensuremath{Prev} blocks.
}%
\@xx{}%
 \@x{ Children ( msg ,\, state ) \.{\defeq} \{ m \.{\in} state \.{:} msg
 \.{\in} PrevMsg ( m ) \}}%
\@pvspace{8.0pt}%
\@x{}%
\@y{\@s{0}%
 Updates scored message scores in current state.
}%
\@xx{}%
\@x{ {\RECURSIVE} UpdateScores ( \_ ,\, \_ )}%
\@x{ UpdateScores ( n ,\, scored ) \.{\defeq}}%
\@x{\@s{8.2} {\IF} scored \.{=} {\langle} {\rangle}}%
\@x{\@s{8.2} \.{\THEN} {\langle} {\rangle}}%
\@x{\@s{8.2} \.{\ELSE} \.{\LET} hd \.{\defeq} Head ( scored )}%
\@x{\@s{64.01} tl\@s{0.24} \.{\defeq} Tail ( scored )}%
\@x{\@s{39.51} \.{\IN}\@s{4.09} {\IF} hd \.{=} genesis}%
 \@x{\@s{64.01} \.{\THEN} {\langle} ScoredMsg ( genesis ,\, TotalWeight )
 {\rangle} \.{\circ} UpdateScores ( n ,\, tl )}%
 \@x{\@s{64.01} \.{\ELSE} {\langle} ScoredMsg ( hd . msg ,\, Score ( Estimate
 ( hd . msg ) ,\, states [ n ] ) ) {\rangle}}%
\@x{\@s{95.32} \.{\circ} UpdateScores ( n ,\, tl )}%
\@pvspace{8.0pt}%
\@x{}%
\@y{\@s{0}%
 Scores all \ensuremath{unscored} messages in current state.
}%
\@xx{}%
\@x{ {\RECURSIVE} ScoreUnscored ( \_ ,\, \_ )}%
\@x{ ScoreUnscored ( n ,\, unscored ) \.{\defeq}}%
\@x{\@s{8.2} {\IF} unscored \.{=} {\langle} {\rangle}}%
\@x{\@s{8.2} \.{\THEN} {\langle} {\rangle}}%
\@x{\@s{8.2} \.{\ELSE} \.{\LET} hd \.{\defeq} Head ( unscored )}%
\@x{\@s{64.01} tl\@s{0.24} \.{\defeq} Tail ( unscored )}%
\@x{\@s{39.51} \.{\IN}\@s{4.09} {\IF} hd \.{\in} EquivReceived ( n )}%
\@x{\@s{64.01} \.{\THEN} ScoreUnscored ( n ,\, tl )}%
 \@x{\@s{64.01} \.{\ELSE} {\langle} ScoredMsg ( hd ,\, Score ( hd ,\, states [
 n ] ) ) {\rangle} \.{\circ} ScoreUnscored ( n ,\, tl )}%
\@pvspace{8.0pt}%
\@x{}\midbar\@xx{}%
\@x{}%
\@y{\@s{0}%
 Local \ensuremath{DAG} views
}%
\@xx{}%
\@x{}%
\@y{\@s{0}%
 \ensuremath{\.{-} dags[n]} consists of a set of nested sets of estimates
}%
\@xx{}%
\@x{}%
\@y{\@s{0}%
 - what is the exact relation between \ensuremath{dags[n]} and
 \ensuremath{states[n]}\.{?} refinement\.{?}
}%
\@xx{}%
\@pvspace{8.0pt}%
\@x{}%
\@y{\@s{0}%
 Set of estimates present in a \ensuremath{DAG}.
}%
\@xx{}%
\@x{ {\RECURSIVE} DagEstimateSet ( \_ )}%
\@x{ DagEstimateSet ( dag ) \.{\defeq}}%
\@x{\@s{8.2} \.{\LET} l \.{\defeq} Len ( dag )}%
\@x{\@s{8.2} \.{\IN} {\IF} dag \.{=} {\langle} genesis {\rangle}}%
\@x{\@s{28.59} \.{\THEN} \{ \}}%
 \@x{\@s{28.59} \.{\ELSE} ToSet ( SubSeq ( dag ,\, 1 ,\, l \.{-} 1 ) )
 \.{\cup} DagEstimateSet ( dag [ l ] )}%
\@pvspace{8.0pt}%
\@x{}%
\@y{\@s{0}%
 \ensuremath{DAG} height.
}%
\@xx{}%
\@x{ {\RECURSIVE} DagHeight ( \_ )}%
\@x{ DagHeight ( dag ) \.{\defeq}}%
\@x{\@s{8.2} {\IF} Len ( dag ) \.{\leq} 1}%
\@x{\@s{8.2} \.{\THEN} 0}%
\@x{\@s{8.2} \.{\ELSE} 1 \.{+} DagHeight ( dag [ Len ( dag ) ] )}%
\@pvspace{8.0pt}%
\@x{}%
\@y{\@s{0}%
 Depth of estimate in \ensuremath{DAG}.
}%
\@xx{}%
\@x{ {\RECURSIVE} DagDepth ( \_ ,\, \_ )}%
\@x{ DagDepth ( est ,\, dag ) \.{\defeq}}%
\@x{\@s{8.2} \.{\LET} l\@s{6.65} \.{\defeq} Len ( dag )}%
\@x{\@s{32.69} d \.{\defeq} DagDepth ( est ,\, dag [ l ] )}%
\@x{\@s{8.2} \.{\IN}\@s{4.09} {\IF} est \.{=} genesis}%
\@x{\@s{32.69} \.{\THEN} DagHeight ( dag )}%
\@x{\@s{32.69} \.{\ELSE} {\IF} est \.{\notin} DagEstimateSet ( dag )}%
\@x{\@s{64.01} \.{\THEN} \.{-} 1}%
\@x{\@s{64.01} \.{\ELSE} {\IF} l \.{\leq} 1}%
\@x{\@s{95.32} \.{\THEN} 0}%
\@x{\@s{95.32} \.{\ELSE} 1 \.{+} d}%
\@pvspace{8.0pt}%
\@x{}%
\@y{\@s{0}%
 Set of \ensuremath{DAG} tips.
}%
\@xx{}%
 \@x{ Tips ( dag ) \.{\defeq} ToSet ( SubSeq ( dag ,\, 1 ,\, Len ( dag ) \.{-}
 1 ) )}%
\@pvspace{8.0pt}%
\@x{}%
\@y{\@s{0}%
 Add scored estimate at level.
}%
\@xx{}%
\@x{ AddAtLevel ( est ,\, dag ) \.{\defeq}}%
\@x{\@s{8.2} {\IF} dag \.{=} {\langle} {\rangle}}%
\@x{\@s{8.2} \.{\THEN} {\langle} est {\rangle}}%
\@x{\@s{8.2} \.{\ELSE} {\IF} Depth ( {\langle} {\rangle} )}%
\@y{\@s{0}%
 finish
}%
\@xx{}%
\@x{\@s{39.51} \.{\THEN} {\langle} {\rangle}\@s{20.5}}%
\@y{\@s{0}%
 finish
}%
\@xx{}%
\@x{\@s{39.51} \.{\ELSE} {\langle} {\rangle}\@s{15.08}}%
\@y{\@s{0}%
 finish
}%
\@xx{}%
\@pvspace{8.0pt}%
\@x{}%
\@y{\@s{0}%
 Add estimate to \ensuremath{dag}.
}%
\@xx{}%
\@x{ AddEstimateToDag ( n ,\, est ) \.{\defeq}}%
\@x{\@s{8.2} \.{\LET} e\@s{4.89} \.{\defeq} {\langle} est {\rangle}}%
\@x{\@s{32.69} d \.{\defeq} dags [ n ]}%
\@x{\@s{8.2} \.{\IN}\@s{4.09} {\IF} dags [ n ] \.{=} {\langle} {\rangle}}%
\@x{\@s{32.69} \.{\THEN} e}%
\@x{\@s{32.69} \.{\ELSE} {\IF} Depth ( est . est ) \.{>} DagHeight ( d )}%
\@x{\@s{64.01} \.{\THEN} e \.{\circ} {\langle} d {\rangle}}%
\@x{\@s{64.01} \.{\ELSE} {\langle} {\rangle}}%
\@y{\@s{0}%
 finish
}%
\@xx{}%
\@pvspace{8.0pt}%
\@x{}\midbar\@xx{}%
\@x{}%
\@y{\@s{0}%
 Preliminary conditions
}%
\@xx{}%
 \@x{ ThresholdCheck \.{\defeq} threshold \.{\geq} 0 \.{\land} threshold \.{<}
 TotalWeight}%
\@x{ NodeWeightLen \.{\defeq} Len ( weights ) \.{=} Cardinality ( nodes )}%
 \@x{ AllSendsValid\@s{7.79} \.{\defeq} SentSet\@s{4.1} \.{=} \{ m \.{\in}
 SentSet \.{:} ValidMsg ( m ) \}}%
 \@x{ AllStatesValid\@s{6.77} \.{\defeq} StateSet\@s{1.28} \.{=} \{ s\@s{4.03}
 \.{\in} StateSet \.{:} ValidState ( s ) \}}%
\@pvspace{8.0pt}%
\@x{}%
\@y{\@s{0}%
 Must hold in all reachable states.
}%
\@xx{}%
\@x{ TypeOK \.{\defeq}}%
\@x{\@s{8.2} \.{\land} AllSendsValid}%
\@x{\@s{8.2} \.{\land} AllStatesValid}%
\@pvspace{8.0pt}%
\@x{}\midbar\@xx{}%
\@x{}%
\@y{\@s{0}%
 Initial state conditions
}%
\@xx{}%
\@x{}%
\@y{\@s{0}%
 All validators start with scored genesis block only.
}%
\@xx{}%
\@x{ Init \.{\defeq}}%
\@x{\@s{8.2} \.{\land} ThresholdCheck}%
\@x{\@s{8.2} \.{\land} NodeWeightLen}%
 \@x{\@s{8.2} \.{\land} dags\@s{29.34} \.{=} Initialize ( {\langle} genesis
 {\rangle} )}%
\@x{\@s{8.2} \.{\land} faulty\@s{23.91} \.{=} Initialize ( \{ \} )}%
 \@x{\@s{8.2} \.{\land} scored\_q\@s{11.86} \.{=} Initialize ( {\langle}
 ScoredMsg ( genesis ,\, TotalWeight ) {\rangle} )}%
 \@x{\@s{8.2} \.{\land} unscored\_q\@s{0.87} \.{=} Initialize ( {\langle}
 {\rangle} )}%
 \@x{\@s{8.2} \.{\land} sent\_msgs\@s{4.21} \.{=} Initialize ( {\langle}
 {\rangle} )}%
\@x{\@s{8.2} \.{\land} equiv\_msgs \.{=} Initialize ( \{ \} )}%
\@x{\@s{8.2} \.{\land} estimates\@s{7.87} \.{=} Initialize ( values )}%
\@x{\@s{8.2} \.{\land} states\@s{23.72} \.{=} Initialize ( \{ genesis \} )}%
\@pvspace{8.0pt}%
\@x{}\midbar\@xx{}%
\@x{}%
\@y{\@s{0}%
 Updates
}%
\@xx{}%
\@x{}%
\@y{\@s{0}%
 A validator can update their set of valid estimates.
}%
\@xx{}%
\@x{ Update\_Estimates ( n ) \.{\defeq}}%
 \@x{\@s{8.2} \.{\land} estimates \.{'} \.{=} [ estimates {\EXCEPT} {\bang} [
 n ] \.{=} GHOST ( states [ n ] ) ]}%
 \@x{\@s{8.2} \.{\land} {\UNCHANGED} {\langle} dags ,\, faulty ,\, scored\_q
 ,\, unscored\_q ,\, sent\_msgs ,\, equiv\_msgs ,\, states {\rangle}}%
\@pvspace{8.0pt}%
\@x{}%
\@y{\@s{0}%
 A validator can score \ensuremath{unscored} estimates and update their
 scores.
}%
\@xx{}%
\@x{ Update\_Scores ( msg ,\, n ,\, rec ) \.{\defeq}}%
 \@x{\@s{8.2} \.{\land} scored\_q \.{'}\@s{8.2} \.{=} [ scored\_q {\EXCEPT}
 {\bang} [ n ] \.{=}}%
 \@x{\@s{19.31} UpdateScores ( scored\_q [ n ] ,\, states [ n ] ) \.{\circ}
 ScoreUnscored ( unscored\_q [ n ] ,\, states [ n ] ) ]}%
 \@x{\@s{8.2} \.{\land} unscored\_q \.{'} \.{=} [ unscored\_q {\EXCEPT}
 {\bang} [ n ] \.{=} {\langle} {\rangle} ]}%
 \@x{\@s{8.2} \.{\land} {\UNCHANGED} {\langle} dags ,\, faulty ,\, sent\_msgs
 ,\, equiv\_msgs ,\, estimates ,\, states {\rangle}}%
\@pvspace{8.0pt}%
\@x{ Update ( n ) \.{\defeq}}%
\@x{}%
\@y{\@s{2.5}%
 \ensuremath{\.{\land} Update\_Scores(n)
}}%
\@xx{}%
\@x{\@s{8.2} \.{\land} Update\_Estimates ( n )}%
\@pvspace{8.0pt}%
\@x{}\midbar\@xx{}%
\begin{lcom}{0}%
\begin{cpar}{0}{F}{F}{0}{0}{}%
Transitions
\end{cpar}%
\end{lcom}%
\@x{}%
\@y{\@s{0}%
 Sending/Receiving/Dropping messages
}%
\@xx{}%
\@x{}%
\@y{\@s{0}%
 Given validator sends given message to given set of validators.
}%
\@xx{}%
\@x{ SendMsg ( msg ,\, n ,\, rec ) \.{\defeq}}%
 \@x{\@s{8.2} \.{\land} unscored\_q \.{'} \.{=} unscored\_q \.{\circ}
 Broadcast ( msg ,\, n ,\, rec )}%
 \@x{\@s{8.2} \.{\land} sent\_msgs \.{'}\@s{4.1} \.{=} [ sent\_msgs {\EXCEPT}
 {\bang} [ n ] \.{=} sent\_msgs [ n ] \.{\circ} {\langle} msg {\rangle} ]}%
 \@x{\@s{8.2} \.{\land} scored\_q \.{'}\@s{10.98} \.{=} [ scored\_q\@s{4.1}
 {\EXCEPT} {\bang} [ n ]\@s{4.30} \.{=} scored\_q [ n ] \.{\circ} {\langle}
 ScoredMsg ( msg ,\, weights [ n ] ) {\rangle} ]}%
 \@x{\@s{8.2} \.{\land} states \.{'}\@s{22.84} \.{=} [ states\@s{12.29}
 {\EXCEPT} {\bang} [ n ]\@s{7.96} \.{=} states [ n ] \.{\cup} \{ msg \} ]}%
\@pvspace{8.0pt}%
\@x{}%
\@y{\@s{0}%
 Honest validator sends honest message.
}%
\@xx{}%
\@x{ Send\_Honest \.{\defeq}}%
 \@x{\@s{8.2} \.{\land} \E\, n \.{\in} ( nodes \.{\,\backslash\,}
 GlobalFaultySet ) \.{:} estimates [ n ] \.{\neq} \{ \}}%
\@y{\@s{0}%
 enabling condition: honest node with valid estimates
}%
\@xx{}%
 \@x{\@s{8.2} \.{\land} \.{\LET} v\@s{3.77} \.{\defeq} RandomElement ( \{ n
 \.{\in} ( nodes \.{\,\backslash\,} GlobalFaultySet ) \.{:} estimates [ n ]
 \.{\neq} \{ \} \} )}%
\@y{\@s{0}%
 honest validator with valid estimate
}%
\@xx{}%
\@x{\@s{43.81} e \.{\defeq} RandomElement ( estimates [ v ] )}%
 \@x{\@s{19.31} \.{\IN}\@s{4.09} \.{\land} SendMsg ( Msg ( e ,\, v ,\, states
 [ v ] ) ,\, v ,\, nodes )}%
 \@x{\@s{43.81} \.{\land} {\UNCHANGED} {\langle} dags ,\, faulty ,\,
 equiv\_msgs ,\, estimates {\rangle}}%
\@pvspace{8.0pt}%
\@x{}%
\@y{\@s{0}%
 Dropped message.
}%
\@xx{}%
\@x{ Send\_Drop \.{\defeq}}%
 \@x{\@s{8.2} \.{\land} \E\, n \.{\in} nodes \.{:} estimates [ n ] \.{\neq} \{
 \}\@s{28.7}}%
\@y{\@s{0}%
 enabling condition: node with valid estimates
}%
\@xx{}%
 \@x{\@s{8.2} \.{\land} \.{\LET} v\@s{3.77} \.{\defeq} RandomElement ( \{ n
 \.{\in} nodes \.{:} estimates [ n ] \.{\neq} \{ \} \} )}%
\@x{\@s{43.81} e \.{\defeq} RandomElement ( estimates [ v ] )}%
 \@x{\@s{19.31} \.{\IN}\@s{4.09} \.{\land} sent\_msgs \.{'}\@s{4.1} \.{=} [
 sent\_msgs {\EXCEPT} {\bang} [ v ] \.{=} sent\_msgs [ v ] \.{\circ}
 {\langle} Msg ( e ,\, v ,\, states [ v ] ) {\rangle} ]}%
 \@x{\@s{43.81} \.{\land} scored\_q \.{'}\@s{11.74} \.{=} [ scored\_q\@s{4.1}
 {\EXCEPT} {\bang} [ v ]\@s{3.54} \.{=} scored\_q [ v ] \.{\circ} {\langle}
 ScoredMsg ( Msg ( e ,\, v ,\, states [ v ] ) ,\, weights [ v ] ) {\rangle}
 ]}%
 \@x{\@s{43.81} \.{\land} states \.{'}\@s{23.60} \.{=} [ states\@s{12.29}
 {\EXCEPT} {\bang} [ v ]\@s{7.20} \.{=} states [ v ] \.{\cup} \{ Msg ( e ,\,
 v ,\, states [ v ] ) \} ]}%
 \@x{\@s{43.81} \.{\land} {\UNCHANGED} {\langle} dags ,\, faulty ,\,
 unscored\_q ,\, equiv\_msgs ,\, estimates {\rangle}}%
\@pvspace{8.0pt}%
\@x{}%
\@y{\@s{0}%
 Equivocations.
}%
\@xx{}%
\@x{}%
\@y{\@s{0}%
 Send messages with different estimates to disjoint sets of validators.
}%
\@xx{}%
\@x{ Send\_Equiv\_Est \.{\defeq}}%
 \@x{\@s{8.2} \.{\land} \E\, n \.{\in} nodes \.{:} Cardinality ( estimates [ n
 ] ) \.{>} 1}%
 \@x{\@s{8.2} \.{\land} \.{\LET} v\@s{4.67} \.{\defeq} RandomElement ( \{ n
 \.{\in} nodes \.{:} Cardinality ( estimates [ n ] ) \.{>} 1 \} )}%
 \@x{\@s{31.61} N1\@s{4.37} \.{\defeq} RandomElement ( \{ sub1 \.{\in}
 {\SUBSET} ( nodes \.{\,\backslash\,} \{ v \} ) \.{:} sub1 \.{\neq} \{ \} \}
 )}%
 \@x{\@s{31.61} N2\@s{4.37} \.{\defeq} RandomElement ( \{ sub2 \.{\in}
 {\SUBSET} ( nodes \.{\,\backslash\,} ( N1 \.{\cup} \{ v \} ) ) \.{:} sub2
 \.{\neq} \{ \} \} )}%
\@x{\@s{31.61} e1\@s{8.09} \.{\defeq} RandomElement ( estimates [ v ] )}%
 \@x{\@s{39.71} e2 \.{\defeq} RandomElement ( estimates [ v ]
 \.{\,\backslash\,} \{ e1 \} )}%
 \@x{\@s{19.31} \.{\IN} \.{\land} SendMsg ( Msg ( e1 ,\, v ,\, states [ v ] )
 ,\, v ,\, N1 )}%
 \@x{\@s{39.71} \.{\land} SendMsg ( Msg ( e2 ,\, v ,\, states [ v ] ) ,\, v
 ,\, N2 )}%
 \@x{\@s{39.71} \.{\land} {\UNCHANGED} {\langle} dags ,\, faulty ,\,
 equiv\_msgs {\rangle}}%
\@pvspace{8.0pt}%
\@x{}%
\@y{\@s{0}%
 Send messages with different justifications to disjoint sets of validators.
}%
\@xx{}%
\@x{ Send\_Equiv\_Just \.{\defeq}}%
 \@x{\@s{8.2} \.{\land} \E\, n \.{\in} nodes \.{:} Cardinality ( states [ n ]
 ) \.{>} 1 \.{\land} estimates [ n ]\@s{40.40} \.{\neq} \{ \}}%
 \@x{\@s{8.2} \.{\land} \.{\LET} v\@s{3.83} \.{\defeq} RandomElement ( \{ n
 \.{\in} nodes \.{:} Cardinality ( states [ n ] ) \.{>} 1 \} )}%
\@x{\@s{39.71} e\@s{4.16} \.{\defeq} RandomElement ( estimates [ v ] )}%
 \@x{\@s{31.61} N1\@s{3.54} \.{\defeq} RandomElement ( \{ sub1 \.{\in}
 {\SUBSET} ( nodes \.{\,\backslash\,} \{ v \} ) \.{:} sub1 \.{\neq} \{ \} \}
 )}%
 \@x{\@s{31.61} N2\@s{3.54} \.{\defeq} RandomElement ( \{ sub2 \.{\in}
 {\SUBSET} ( nodes \.{\,\backslash\,} ( N1 \.{\cup} \{ v \} ) ) \.{:} sub2
 \.{\neq} \{ \} \} )}%
 \@x{\@s{31.61} j1\@s{8.09} \.{\defeq} RandomElement ( {\SUBSET} ( states [ v
 ] ) )}%
 \@x{\@s{39.71} j2 \.{\defeq} RandomElement ( \{ j \.{\in} {\SUBSET} ( states
 [ v ] ) \.{:} j \.{\neq} j1 \} )}%
 \@x{\@s{19.31} \.{\IN} \.{\land} SendMsg ( Msg ( e ,\, v ,\, j1 ) ,\, v ,\,
 N1 )}%
\@x{\@s{39.71} \.{\land} SendMsg ( Msg ( e ,\, v ,\, j2 ) ,\, v ,\, N2 )}%
 \@x{\@s{39.71} \.{\land} {\UNCHANGED} {\langle} dags ,\, faulty ,\,
 equiv\_msgs {\rangle}}%
\@pvspace{8.0pt}%
\@x{}%
\@y{\@s{0}%
 Send messages with different estimates and different justifications to
 disjoint sets of validators.
}%
\@xx{}%
\@x{ Send\_Equiv\_Both \.{\defeq}}%
 \@x{\@s{8.2} \.{\land} \E\, n \.{\in} nodes \.{:} Cardinality ( states [ n ]
 ) \.{>} 1 \.{\land} Cardinality ( estimates [ n ] ) \.{>} 1}%
 \@x{\@s{8.2} \.{\land} \.{\LET} v\@s{3.83} \.{\defeq} RandomElement ( \{ n
 \.{\in} nodes \.{:} Cardinality ( states [ n ] ) \.{>} 1 \} )}%
\@x{\@s{31.61} e1\@s{7.26} \.{\defeq} RandomElement ( estimates [ v ] )}%
 \@x{\@s{31.61} e2\@s{7.26} \.{\defeq} RandomElement ( estimates [ v ]
 \.{\,\backslash\,} \{ e1 \} )}%
 \@x{\@s{31.61} N1\@s{3.54} \.{\defeq} RandomElement ( \{ sub1 \.{\in}
 {\SUBSET} ( nodes \.{\,\backslash\,} \{ v \} ) \.{:} sub1 \.{\neq} \{ \} \}
 )}%
 \@x{\@s{31.61} N2\@s{3.54} \.{\defeq} RandomElement ( \{ sub2 \.{\in}
 {\SUBSET} ( nodes \.{\,\backslash\,} ( N1 \.{\cup} \{ v \} ) ) \.{:} sub2
 \.{\neq} \{ \} \} )}%
 \@x{\@s{31.61} j1\@s{8.09} \.{\defeq} RandomElement ( {\SUBSET} ( states [ v
 ] ) )}%
 \@x{\@s{39.71} j2 \.{\defeq} RandomElement ( \{ j \.{\in} {\SUBSET} ( states
 [ v ] ) \.{:} j \.{\neq} j1 \} )}%
 \@x{\@s{19.31} \.{\IN} \.{\land} SendMsg ( Msg ( e1 ,\, v ,\, j1 ) ,\, v ,\,
 N1 )}%
\@x{\@s{39.71} \.{\land} SendMsg ( Msg ( e2 ,\, v ,\, j2 ) ,\, v ,\, N2 )}%
 \@x{\@s{39.71} \.{\land} {\UNCHANGED} {\langle} dags ,\, faulty ,\,
 equiv\_msgs {\rangle}}%
\@pvspace{8.0pt}%
\@x{ Send\_Success \.{\defeq}}%
\@x{\@s{8.2} \.{\lor} Send\_Honest}%
\@x{\@s{8.2} \.{\lor} Send\_Equiv\_Est}%
\@x{\@s{8.2} \.{\lor} Send\_Equiv\_Just}%
\@x{\@s{8.2} \.{\lor} Send\_Equiv\_Both}%
\@pvspace{8.0pt}%
\@x{ Send \.{\defeq}}%
\@x{\@s{8.2} \.{\lor}\@s{2.67} Send\_Success}%
\@x{\@s{8.2} \.{\lor}\@s{2.67} Send\_Drop}%
\@pvspace{8.0pt}%
\@x{}%
\@y{\@s{0}%
 \ensuremath{vars \.{\defeq}
 {\langle}faulty,\,scored\_q,\,unscored\_q,\,sent\_msgs,\,equiv\_msgs,\,estimates,\,states{\rangle}
}}%
\@xx{}%
\@pvspace{8.0pt}%
\@x{}\midbar\@xx{}%
\@x{}%
\@y{%
 \ensuremath{TODO
}}%
\@xx{}%
\@x{}%
\@y{\@s{0}%
 Upon detection of an equivocation, all validators except the equivocator add
 equivicator to faulty set
}%
\@xx{}%
\@x{}%
\@y{\@s{0}%
 - check dependencies of all received messages for equivocations
}%
\@xx{}%
\@x{}%
\@y{\@s{0}%
 - put equivocated message pairs in \ensuremath{equiv\_msgs
}}%
\@xx{}%
\@x{ HandleEquiv \.{\defeq}}%
 \@x{\@s{8.2} \.{\land} \E\, n \.{\in} nodes \.{:} CheckDepsForEquiv (
 ReceivedMsgs ( n ) )}%
 \@x{\@s{8.2} \.{\land} \.{\LET} n\@s{3.45} \.{\defeq} RandomElement ( \{ v
 \.{\in} nodes \.{:} CheckDepsForEquiv ( ReceivedMsgs ( v ) ) \} )}%
\@x{\@s{39.71} E\@s{1.85} \.{\defeq} EquivPairsInDeps ( ReceivedMsgs ( n ) )}%
\@x{\@s{43.81} p \.{\defeq} Pick ( E )}%
 \@x{\@s{19.31} \.{\IN}\@s{4.09} \.{\land} faulty \.{'}\@s{23.91} \.{=} [
 faulty {\EXCEPT} {\bang} [ n ] \.{=} faulty [ n ] \.{\cup} \{ Sender ( Head
 ( p ) ) \} ]}%
 \@x{\@s{43.81} \.{\land} equiv\_msgs \.{'} \.{=} [ equiv\_msgs {\EXCEPT}
 {\bang} [ n ] \.{=} equiv\_msgs [ n ] \.{\cup} E ]}%
 \@x{\@s{43.81} \.{\land} {\UNCHANGED} {\langle} dags ,\, scored\_q ,\,
 unscored\_q ,\, sent\_msgs ,\, estimates ,\, states {\rangle}}%
\@pvspace{8.0pt}%
\@x{ Next \.{\defeq}}%
\@x{\@s{8.2} \.{\lor}\@s{1.63} Send}%
\@x{\@s{8.2} \.{\lor}\@s{1.63} HandleEquiv}%
\@pvspace{8.0pt}%
\@x{ SafetySpec \.{\defeq}}%
\@x{\@s{8.2} \.{\land} Init}%
\@x{\@s{8.2} \.{\land} {\Box} [ Next ]_{ vars}}%
\@pvspace{8.0pt}%
\@x{ LivenessSpec \.{\defeq}}%
\@x{\@s{8.2} \.{\land} {\WF}_{ vars} ( Send )}%
 \@x{\@s{8.2} \.{\land} {\SF}_{ vars} ( \E\, n \.{\in} nodes \.{:} Update ( n
 ) )}%
\@x{\@s{8.2} \.{\land} {\SF}_{ vars} ( HandleEquiv )}%
\@pvspace{8.0pt}%
\@x{ Spec \.{\defeq}}%
\@x{\@s{8.2} \.{\land}\@s{0.16} SafetySpec}%
\@x{\@s{8.2} \.{\land}\@s{0.16} LivenessSpec}%
\@pvspace{8.0pt}%
\@x{}\bottombar\@xx{}%
\setboolean{shading}{false}
\begin{lcom}{0}%
\begin{cpar}{0}{F}{F}{0}{0}{}%
\ensuremath{\.{\,\backslash\,}\.{*}} Modification History
\end{cpar}%
\begin{cpar}{0}{F}{F}{0}{0}{}%
 \ensuremath{\.{\,\backslash\,}\.{*}} Last modified Sat \ensuremath{Dec} 14
 13:00:41 \ensuremath{EST} 2019 by \ensuremath{isaac
}%
\end{cpar}%
\begin{cpar}{0}{F}{F}{0}{0}{}%
 \ensuremath{\.{\,\backslash\,}\.{*}} Created \ensuremath{Wed}
 \ensuremath{Nov} 27 17:00:08 \ensuremath{EST} 2019 by \ensuremath{isaac
}%
\end{cpar}%
\end{lcom}%
\end{document}
