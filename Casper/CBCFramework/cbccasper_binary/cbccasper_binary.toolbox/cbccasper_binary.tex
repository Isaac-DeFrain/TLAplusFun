\batchmode %% Suppresses most terminal output.
\documentclass{article}
\usepackage{color}
\definecolor{boxshade}{gray}{0.85}
\setlength{\textwidth}{360pt}
\setlength{\textheight}{541pt}
\usepackage{latexsym}
\usepackage{ifthen}
% \usepackage{color}
%%%%%%%%%%%%%%%%%%%%%%%%%%%%%%%%%%%%%%%%%%%%%%%%%%%%%%%%%%%%%%%%%%%%%%%%%%%%%
% SWITCHES                                                                  %
%%%%%%%%%%%%%%%%%%%%%%%%%%%%%%%%%%%%%%%%%%%%%%%%%%%%%%%%%%%%%%%%%%%%%%%%%%%%%
\newboolean{shading} 
\setboolean{shading}{false}
\makeatletter
 %% this is needed only when inserted into the file, not when
 %% used as a package file.
%%%%%%%%%%%%%%%%%%%%%%%%%%%%%%%%%%%%%%%%%%%%%%%%%%%%%%%%%%%%%%%%%%%%%%%%%%%%%
%                                                                           %
% DEFINITIONS OF SYMBOL-PRODUCING COMMANDS                                  %
%                                                                           %
%    TLA+      LaTeX                                                        %
%    symbol    command                                                      %
%    ------    -------                                                      %
%    =>        \implies                                                     %
%    <:        \ltcolon                                                     %
%    :>        \colongt                                                     %
%    ==        \defeq                                                       %
%    ..        \dotdot                                                      %
%    ::        \coloncolon                                                  %
%    =|        \eqdash                                                      %
%    ++        \pp                                                          %
%    --        \mm                                                          %
%    **        \stst                                                        %
%    //        \slsl                                                        %
%    ^         \ct                                                          %
%    \A        \A                                                           %
%    \E        \E                                                           %
%    \AA       \AA                                                          %
%    \EE       \EE                                                          %
%%%%%%%%%%%%%%%%%%%%%%%%%%%%%%%%%%%%%%%%%%%%%%%%%%%%%%%%%%%%%%%%%%%%%%%%%%%%%
\newlength{\symlength}
\newcommand{\implies}{\Rightarrow}
\newcommand{\ltcolon}{\mathrel{<\!\!\mbox{:}}}
\newcommand{\colongt}{\mathrel{\!\mbox{:}\!\!>}}
\newcommand{\defeq}{\;\mathrel{\smash   %% keep this symbol from being too tall
    {{\stackrel{\scriptscriptstyle\Delta}{=}}}}\;}
\newcommand{\dotdot}{\mathrel{\ldotp\ldotp}}
\newcommand{\coloncolon}{\mathrel{::\;}}
\newcommand{\eqdash}{\mathrel = \joinrel \hspace{-.28em}|}
\newcommand{\pp}{\mathbin{++}}
\newcommand{\mm}{\mathbin{--}}
\newcommand{\stst}{*\!*}
\newcommand{\slsl}{/\!/}
\newcommand{\ct}{\hat{\hspace{.4em}}}
\newcommand{\A}{\forall}
\newcommand{\E}{\exists}
\renewcommand{\AA}{\makebox{$\raisebox{.05em}{\makebox[0pt][l]{%
   $\forall\hspace{-.517em}\forall\hspace{-.517em}\forall$}}%
   \forall\hspace{-.517em}\forall \hspace{-.517em}\forall\,$}}
\newcommand{\EE}{\makebox{$\raisebox{.05em}{\makebox[0pt][l]{%
   $\exists\hspace{-.517em}\exists\hspace{-.517em}\exists$}}%
   \exists\hspace{-.517em}\exists\hspace{-.517em}\exists\,$}}
\newcommand{\whileop}{\.{\stackrel
  {\mbox{\raisebox{-.3em}[0pt][0pt]{$\scriptscriptstyle+\;\,$}}}%
  {-\hspace{-.16em}\triangleright}}}

% Commands are defined to produce the upper-case keywords.
% Note that some have space after them.
\newcommand{\ASSUME}{\textsc{assume }}
\newcommand{\ASSUMPTION}{\textsc{assumption }}
\newcommand{\AXIOM}{\textsc{axiom }}
\newcommand{\BOOLEAN}{\textsc{boolean }}
\newcommand{\CASE}{\textsc{case }}
\newcommand{\CONSTANT}{\textsc{constant }}
\newcommand{\CONSTANTS}{\textsc{constants }}
\newcommand{\ELSE}{\settowidth{\symlength}{\THEN}%
   \makebox[\symlength][l]{\textsc{ else}}}
\newcommand{\EXCEPT}{\textsc{ except }}
\newcommand{\EXTENDS}{\textsc{extends }}
\newcommand{\FALSE}{\textsc{false}}
\newcommand{\IF}{\textsc{if }}
\newcommand{\IN}{\settowidth{\symlength}{\LET}%
   \makebox[\symlength][l]{\textsc{in}}}
\newcommand{\INSTANCE}{\textsc{instance }}
\newcommand{\LET}{\textsc{let }}
\newcommand{\LOCAL}{\textsc{local }}
\newcommand{\MODULE}{\textsc{module }}
\newcommand{\OTHER}{\textsc{other }}
\newcommand{\STRING}{\textsc{string}}
\newcommand{\THEN}{\textsc{ then }}
\newcommand{\THEOREM}{\textsc{theorem }}
\newcommand{\LEMMA}{\textsc{lemma }}
\newcommand{\PROPOSITION}{\textsc{proposition }}
\newcommand{\COROLLARY}{\textsc{corollary }}
\newcommand{\TRUE}{\textsc{true}}
\newcommand{\VARIABLE}{\textsc{variable }}
\newcommand{\VARIABLES}{\textsc{variables }}
\newcommand{\WITH}{\textsc{ with }}
\newcommand{\WF}{\textrm{WF}}
\newcommand{\SF}{\textrm{SF}}
\newcommand{\CHOOSE}{\textsc{choose }}
\newcommand{\ENABLED}{\textsc{enabled }}
\newcommand{\UNCHANGED}{\textsc{unchanged }}
\newcommand{\SUBSET}{\textsc{subset }}
\newcommand{\UNION}{\textsc{union }}
\newcommand{\DOMAIN}{\textsc{domain }}
% Added for tla2tex
\newcommand{\BY}{\textsc{by }}
\newcommand{\OBVIOUS}{\textsc{obvious }}
\newcommand{\HAVE}{\textsc{have }}
\newcommand{\QED}{\textsc{qed }}
\newcommand{\TAKE}{\textsc{take }}
\newcommand{\DEF}{\textsc{ def }}
\newcommand{\HIDE}{\textsc{hide }}
\newcommand{\RECURSIVE}{\textsc{recursive }}
\newcommand{\USE}{\textsc{use }}
\newcommand{\DEFINE}{\textsc{define }}
\newcommand{\PROOF}{\textsc{proof }}
\newcommand{\WITNESS}{\textsc{witness }}
\newcommand{\PICK}{\textsc{pick }}
\newcommand{\DEFS}{\textsc{defs }}
\newcommand{\PROVE}{\settowidth{\symlength}{\ASSUME}%
   \makebox[\symlength][l]{\textsc{prove}}\@s{-4.1}}%
  %% The \@s{-4.1) is a kludge added on 24 Oct 2009 [happy birthday, Ellen]
  %% so the correct alignment occurs if the user types
  %%   ASSUME X
  %%   PROVE  Y
  %% because it cancels the extra 4.1 pts added because of the 
  %% extra space after the PROVE.  This seems to works OK.
  %% However, the 4.1 equals Parameters.LaTeXLeftSpace(1) and
  %% should be changed if that method ever changes.
\newcommand{\SUFFICES}{\textsc{suffices }}
\newcommand{\NEW}{\textsc{new }}
\newcommand{\LAMBDA}{\textsc{lambda }}
\newcommand{\STATE}{\textsc{state }}
\newcommand{\ACTION}{\textsc{action }}
\newcommand{\TEMPORAL}{\textsc{temporal }}
\newcommand{\ONLY}{\textsc{only }}              %% added by LL on 2 Oct 2009
\newcommand{\OMITTED}{\textsc{omitted }}        %% added by LL on 31 Oct 2009
\newcommand{\@pfstepnum}[2]{\ensuremath{\langle#1\rangle}\textrm{#2}}
\newcommand{\bang}{\@s{1}\mbox{\small !}\@s{1}}
%% We should format || differently in PlusCal code than in TLA+ formulas.
\newcommand{\p@barbar}{\ifpcalsymbols
   \,\,\rule[-.25em]{.075em}{1em}\hspace*{.2em}\rule[-.25em]{.075em}{1em}\,\,%
   \else \,||\,\fi}
%% PlusCal keywords
\newcommand{\p@fair}{\textbf{fair }}
\newcommand{\p@semicolon}{\textbf{\,; }}
\newcommand{\p@algorithm}{\textbf{algorithm }}
\newcommand{\p@mmfair}{\textbf{-{}-fair }}
\newcommand{\p@mmalgorithm}{\textbf{-{}-algorithm }}
\newcommand{\p@assert}{\textbf{assert }}
\newcommand{\p@await}{\textbf{await }}
\newcommand{\p@begin}{\textbf{begin }}
\newcommand{\p@end}{\textbf{end }}
\newcommand{\p@call}{\textbf{call }}
\newcommand{\p@define}{\textbf{define }}
\newcommand{\p@do}{\textbf{ do }}
\newcommand{\p@either}{\textbf{either }}
\newcommand{\p@or}{\textbf{or }}
\newcommand{\p@goto}{\textbf{goto }}
\newcommand{\p@if}{\textbf{if }}
\newcommand{\p@then}{\,\,\textbf{then }}
\newcommand{\p@else}{\ifcsyntax \textbf{else } \else \,\,\textbf{else }\fi}
\newcommand{\p@elsif}{\,\,\textbf{elsif }}
\newcommand{\p@macro}{\textbf{macro }}
\newcommand{\p@print}{\textbf{print }}
\newcommand{\p@procedure}{\textbf{procedure }}
\newcommand{\p@process}{\textbf{process }}
\newcommand{\p@return}{\textbf{return}}
\newcommand{\p@skip}{\textbf{skip}}
\newcommand{\p@variable}{\textbf{variable }}
\newcommand{\p@variables}{\textbf{variables }}
\newcommand{\p@while}{\textbf{while }}
\newcommand{\p@when}{\textbf{when }}
\newcommand{\p@with}{\textbf{with }}
\newcommand{\p@lparen}{\textbf{(\,\,}}
\newcommand{\p@rparen}{\textbf{\,\,) }}   
\newcommand{\p@lbrace}{\textbf{\{\,\,}}   
\newcommand{\p@rbrace}{\textbf{\,\,\} }}

%%%%%%%%%%%%%%%%%%%%%%%%%%%%%%%%%%%%%%%%%%%%%%%%%%%%%%%%%
% REDEFINE STANDARD COMMANDS TO MAKE THEM FORMAT BETTER %
%                                                       %
% We redefine \in and \notin                            %
%%%%%%%%%%%%%%%%%%%%%%%%%%%%%%%%%%%%%%%%%%%%%%%%%%%%%%%%%
\renewcommand{\_}{\rule{.4em}{.06em}\hspace{.05em}}
\newlength{\equalswidth}
\let\oldin=\in
\let\oldnotin=\notin
\renewcommand{\in}{%
   {\settowidth{\equalswidth}{$\.{=}$}\makebox[\equalswidth][c]{$\oldin$}}}
\renewcommand{\notin}{%
   {\settowidth{\equalswidth}{$\.{=}$}\makebox[\equalswidth]{$\oldnotin$}}}


%%%%%%%%%%%%%%%%%%%%%%%%%%%%%%%%%%%%%%%%%%%%%%%%%%%%
%                                                  %
% HORIZONTAL BARS:                                 %
%                                                  %
%   \moduleLeftDash    |~~~~~~~~~~                 %
%   \moduleRightDash    ~~~~~~~~~~|                %
%   \midbar            |----------|                %
%   \bottombar         |__________|                %
%%%%%%%%%%%%%%%%%%%%%%%%%%%%%%%%%%%%%%%%%%%%%%%%%%%%
\newlength{\charwidth}\settowidth{\charwidth}{{\small\tt M}}
\newlength{\boxrulewd}\setlength{\boxrulewd}{.4pt}
\newlength{\boxlineht}\setlength{\boxlineht}{.5\baselineskip}
\newcommand{\boxsep}{\charwidth}
\newlength{\boxruleht}\setlength{\boxruleht}{.5ex}
\newlength{\boxruledp}\setlength{\boxruledp}{-\boxruleht}
\addtolength{\boxruledp}{\boxrulewd}
\newcommand{\boxrule}{\leaders\hrule height \boxruleht depth \boxruledp
                      \hfill\mbox{}}
\newcommand{\@computerule}{%
  \setlength{\boxruleht}{.5ex}%
  \setlength{\boxruledp}{-\boxruleht}%
  \addtolength{\boxruledp}{\boxrulewd}}

\newcommand{\bottombar}{\hspace{-\boxsep}%
  \raisebox{-\boxrulewd}[0pt][0pt]{\rule[.5ex]{\boxrulewd}{\boxlineht}}%
  \boxrule
  \raisebox{-\boxrulewd}[0pt][0pt]{%
      \rule[.5ex]{\boxrulewd}{\boxlineht}}\hspace{-\boxsep}\vspace{0pt}}

\newcommand{\moduleLeftDash}%
   {\hspace*{-\boxsep}%
     \raisebox{-\boxlineht}[0pt][0pt]{\rule[.5ex]{\boxrulewd
               }{\boxlineht}}%
    \boxrule\hspace*{.4em }}

\newcommand{\moduleRightDash}%
    {\hspace*{.4em}\boxrule
    \raisebox{-\boxlineht}[0pt][0pt]{\rule[.5ex]{\boxrulewd
               }{\boxlineht}}\hspace{-\boxsep}}%\vspace{.2em}

\newcommand{\midbar}{\hspace{-\boxsep}\raisebox{-.5\boxlineht}[0pt][0pt]{%
   \rule[.5ex]{\boxrulewd}{\boxlineht}}\boxrule\raisebox{-.5\boxlineht%
   }[0pt][0pt]{\rule[.5ex]{\boxrulewd}{\boxlineht}}\hspace{-\boxsep}}

%%%%%%%%%%%%%%%%%%%%%%%%%%%%%%%%%%%%%%%%%%%%%%%%%%%%%%%%%%%%%%%%%%%%%%%%%%%%%
% FORMATING COMMANDS                                                        %
%%%%%%%%%%%%%%%%%%%%%%%%%%%%%%%%%%%%%%%%%%%%%%%%%%%%%%%%%%%%%%%%%%%%%%%%%%%%%

%%%%%%%%%%%%%%%%%%%%%%%%%%%%%%%%%%%%%%%%%%%%%%%%%%%%%%%%%%%%%%%%%%%%%%%%%%%%%
% PLUSCAL SHADING                                                           %
%%%%%%%%%%%%%%%%%%%%%%%%%%%%%%%%%%%%%%%%%%%%%%%%%%%%%%%%%%%%%%%%%%%%%%%%%%%%%

% The TeX pcalshading switch is set on to cause PlusCal shading to be
% performed.  This changes the behavior of the following commands and
% environments to cause full-width shading to be performed on all lines.
% 
%   \tstrut \@x cpar mcom \@pvspace
% 
% The TeX pcalsymbols switch is turned on when typesetting a PlusCal algorithm,
% whether or not shading is being performed.  It causes symbols (other than
% parentheses and braces and PlusCal-only keywords) that should be typeset
% differently depending on whether they are in an algorithm to be typeset
% appropriately.  Currently, the only such symbol is "||".
%
% The TeX csyntax switch is turned on when typesetting a PlusCal algorithm in
% c-syntax.  This allows symbols to be format differently in the two syntaxes.
% The "else" keyword is the only one that is.

\newif\ifpcalshading \pcalshadingfalse
\newif\ifpcalsymbols \pcalsymbolsfalse
\newif\ifcsyntax     \csyntaxtrue

% The \@pvspace command makes a vertical space.  It uses \vspace
% except with \ifpcalshading, in which case it sets \pvcalvspace
% and the space is added by a following \@x command.
%
\newlength{\pcalvspace}\setlength{\pcalvspace}{0pt}%
\newcommand{\@pvspace}[1]{%
  \ifpcalshading
     \par\global\setlength{\pcalvspace}{#1}%
  \else
     \par\vspace{#1}%
  \fi
}

% The lcom environment was changed to set \lcomindent equal to
% the indentation it produces.  This length is used by the
% cpar environment to make shading extend for the full width
% of the line.  This assumes that lcom environments are not
% nested.  I hope TLATeX does not nest them.
%
\newlength{\lcomindent}%
\setlength{\lcomindent}{0pt}%

%\tstrut: A strut to produce inter-paragraph space in a comment.
%\rstrut: A strut to extend the bottom of a one-line comment so
%         there's no break in the shading between comments on 
%         successive lines.
\newcommand\tstrut%
  {\raisebox{\vshadelen}{\raisebox{-.25em}{\rule{0pt}{1.15em}}}%
   \global\setlength{\vshadelen}{0pt}}
\newcommand\rstrut{\raisebox{-.25em}{\rule{0pt}{1.15em}}%
 \global\setlength{\vshadelen}{0pt}}


% \.{op} formats operator op in math mode with empty boxes on either side.
% Used because TeX otherwise vary the amount of space it leaves around op.
\renewcommand{\.}[1]{\ensuremath{\mbox{}#1\mbox{}}}

% \@s{n} produces an n-point space
\newcommand{\@s}[1]{\hspace{#1pt}}           

% \@x{txt} starts a specification line in the beginning with txt
% in the final LaTeX source.
\newlength{\@xlen}
\newcommand\xtstrut%
  {\setlength{\@xlen}{1.05em}%
   \addtolength{\@xlen}{\pcalvspace}%
    \raisebox{\vshadelen}{\raisebox{-.25em}{\rule{0pt}{\@xlen}}}%
   \global\setlength{\vshadelen}{0pt}%
   \global\setlength{\pcalvspace}{0pt}}

\newcommand{\@x}[1]{\par
  \ifpcalshading
  \makebox[0pt][l]{\shadebox{\xtstrut\hspace*{\textwidth}}}%
  \fi
  \mbox{$\mbox{}#1\mbox{}$}}  

% \@xx{txt} continues a specification line with the text txt.
\newcommand{\@xx}[1]{\mbox{$\mbox{}#1\mbox{}$}}  

% \@y{cmt} produces a one-line comment.
\newcommand{\@y}[1]{\mbox{\footnotesize\hspace{.65em}%
  \ifthenelse{\boolean{shading}}{%
      \shadebox{#1\hspace{-\the\lastskip}\rstrut}}%
               {#1\hspace{-\the\lastskip}\rstrut}}}

% \@z{cmt} produces a zero-width one-line comment.
\newcommand{\@z}[1]{\makebox[0pt][l]{\footnotesize
  \ifthenelse{\boolean{shading}}{%
      \shadebox{#1\hspace{-\the\lastskip}\rstrut}}%
               {#1\hspace{-\the\lastskip}\rstrut}}}


% \@w{str} produces the TLA+ string "str".
\newcommand{\@w}[1]{\textsf{``{#1}''}}             


%%%%%%%%%%%%%%%%%%%%%%%%%%%%%%%%%%%%%%%%%%%%%%%%%%%%%%%%%%%%%%%%%%%%%%%%%%%%%
% SHADING                                                                   %
%%%%%%%%%%%%%%%%%%%%%%%%%%%%%%%%%%%%%%%%%%%%%%%%%%%%%%%%%%%%%%%%%%%%%%%%%%%%%
\def\graymargin{1}
  % The number of points of margin in the shaded box.

% \definecolor{boxshade}{gray}{.85}
% Defines the darkness of the shading: 1 = white, 0 = black
% Added by TLATeX only if needed.

% \shadebox{txt} puts txt in a shaded box.
\newlength{\templena}
\newlength{\templenb}
\newsavebox{\tempboxa}
\newcommand{\shadebox}[1]{{\setlength{\fboxsep}{\graymargin pt}%
     \savebox{\tempboxa}{#1}%
     \settoheight{\templena}{\usebox{\tempboxa}}%
     \settodepth{\templenb}{\usebox{\tempboxa}}%
     \hspace*{-\fboxsep}\raisebox{0pt}[\templena][\templenb]%
        {\colorbox{boxshade}{\usebox{\tempboxa}}}\hspace*{-\fboxsep}}}

% \vshade{n} makes an n-point inter-paragraph space, with
%  shading if the `shading' flag is true.
\newlength{\vshadelen}
\setlength{\vshadelen}{0pt}
\newcommand{\vshade}[1]{\ifthenelse{\boolean{shading}}%
   {\global\setlength{\vshadelen}{#1pt}}%
   {\vspace{#1pt}}}

\newlength{\boxwidth}
\newlength{\multicommentdepth}

%%%%%%%%%%%%%%%%%%%%%%%%%%%%%%%%%%%%%%%%%%%%%%%%%%%%%%%%%%%%%%%%%%%%%%%%%%%%%
% THE cpar ENVIRONMENT                                                      %
% ^^^^^^^^^^^^^^^^^^^^                                                      %
% The LaTeX input                                                           %
%                                                                           %
%   \begin{cpar}{pop}{nest}{isLabel}{d}{e}{arg6}                            %
%     XXXXXXXXXXXXXXX                                                       %
%     XXXXXXXXXXXXXXX                                                       %
%     XXXXXXXXXXXXXXX                                                       %
%   \end{cpar}                                                              %
%                                                                           %
% produces one of two possible results.  If isLabel is the letter "T",      %
% it produces the following, where [label] is the result of typesetting     %
% arg6 in an LR box, and d is is a number representing a distance in        %
% points.                                                                   %
%                                                                           %
%   prevailing |<-- d -->[label]<- e ->XXXXXXXXXXXXXXX                      %
%         left |                       XXXXXXXXXXXXXXX                      %
%       margin |                       XXXXXXXXXXXXXXX                      %
%                                                                           %
% If isLabel is the letter "F", then it produces                            %
%                                                                           %
%   prevailing |<-- d -->XXXXXXXXXXXXXXXXXXXXXXX                            %
%         left |         <- e ->XXXXXXXXXXXXXXXX                            %
%       margin |                XXXXXXXXXXXXXXXX                            %
%                                                                           %
% where d and e are numbers representing distances in points.               %
%                                                                           %
% The prevailing left margin is the one in effect before the most recent    %
% pop (argument 1) cpar environments with "T" as the nest argument, where   %
% pop is a number \geq 0.                                                   %
%                                                                           %
% If the nest argument is the letter "T", then the prevailing left          %
% margin is moved to the left of the second (and following) lines of        %
% X's.  Otherwise, the prevailing left margin is left unchanged.            %
%                                                                           %
% An \unnest{n} command moves the prevailing left margin to where it was    %
% before the most recent n cpar environments with "T" as the nesting        %
% argument.                                                                 %
%                                                                           %
% The environment leaves no vertical space above or below it, or between    %
% its paragraphs.  (TLATeX inserts the proper amount of vertical space.)    %
%%%%%%%%%%%%%%%%%%%%%%%%%%%%%%%%%%%%%%%%%%%%%%%%%%%%%%%%%%%%%%%%%%%%%%%%%%%%%

\newcounter{pardepth}
\setcounter{pardepth}{0}

% \setgmargin{txt} defines \gmarginN to be txt, where N is \roman{pardepth}.
% \thegmargin equals \gmarginN, where N is \roman{pardepth}.
\newcommand{\setgmargin}[1]{%
  \expandafter\xdef\csname gmargin\roman{pardepth}\endcsname{#1}}
\newcommand{\thegmargin}{\csname gmargin\roman{pardepth}\endcsname}
\newcommand{\gmargin}{0pt}

\newsavebox{\tempsbox}

\newlength{\@cparht}
\newlength{\@cpardp}
\newenvironment{cpar}[6]{%
  \addtocounter{pardepth}{-#1}%
  \ifthenelse{\boolean{shading}}{\par\begin{lrbox}{\tempsbox}%
                                 \begin{minipage}[t]{\linewidth}}{}%
  \begin{list}{}{%
     \edef\temp{\thegmargin}
     \ifthenelse{\equal{#3}{T}}%
       {\settowidth{\leftmargin}{\hspace{\temp}\footnotesize #6\hspace{#5pt}}%
        \addtolength{\leftmargin}{#4pt}}%
       {\setlength{\leftmargin}{#4pt}%
        \addtolength{\leftmargin}{#5pt}%
        \addtolength{\leftmargin}{\temp}%
        \setlength{\itemindent}{-#5pt}}%
      \ifthenelse{\equal{#2}{T}}{\addtocounter{pardepth}{1}%
                                 \setgmargin{\the\leftmargin}}{}%
      \setlength{\labelwidth}{0pt}%
      \setlength{\labelsep}{0pt}%
      \setlength{\itemindent}{-\leftmargin}%
      \setlength{\topsep}{0pt}%
      \setlength{\parsep}{0pt}%
      \setlength{\partopsep}{0pt}%
      \setlength{\parskip}{0pt}%
      \setlength{\itemsep}{0pt}
      \setlength{\itemindent}{#4pt}%
      \addtolength{\itemindent}{-\leftmargin}}%
   \ifthenelse{\equal{#3}{T}}%
      {\item[\tstrut\footnotesize \hspace{\temp}{#6}\hspace{#5pt}]
        }%
      {\item[\tstrut\hspace{\temp}]%
         }%
   \footnotesize}
 {\hspace{-\the\lastskip}\tstrut
 \end{list}%
  \ifthenelse{\boolean{shading}}%
          {\end{minipage}%
           \end{lrbox}%
           \ifpcalshading
             \setlength{\@cparht}{\ht\tempsbox}%
             \setlength{\@cpardp}{\dp\tempsbox}%
             \addtolength{\@cparht}{.15em}%
             \addtolength{\@cpardp}{.2em}%
             \addtolength{\@cparht}{\@cpardp}%
            % I don't know what's going on here.  I want to add a
            % \pcalvspace high shaded line, but I don't know how to
            % do it.  A little trial and error shows that the following
            % does a reasonable job approximating that, eliminating
            % the line if \pcalvspace is small.
            \addtolength{\@cparht}{\pcalvspace}%
             \ifdim \pcalvspace > .8em
               \addtolength{\pcalvspace}{-.2em}%
               \hspace*{-\lcomindent}%
               \shadebox{\rule{0pt}{\pcalvspace}\hspace*{\textwidth}}\par
               \global\setlength{\pcalvspace}{0pt}%
               \fi
             \hspace*{-\lcomindent}%
             \makebox[0pt][l]{\raisebox{-\@cpardp}[0pt][0pt]{%
                 \shadebox{\rule{0pt}{\@cparht}\hspace*{\textwidth}}}}%
             \hspace*{\lcomindent}\usebox{\tempsbox}%
             \par
           \else
             \shadebox{\usebox{\tempsbox}}\par
           \fi}%
           {}%
  }

%%%%%%%%%%%%%%%%%%%%%%%%%%%%%%%%%%%%%%%%%%%%%%%%%%%%%%%%%%%%%%%%%%%%%%%%%%%%%%
% THE ppar ENVIRONMENT                                                       %
% ^^^^^^^^^^^^^^^^^^^^                                                       %
% The environment                                                            %
%                                                                            %
%   \begin{ppar} ... \end{ppar}                                              %
%                                                                            %
% is equivalent to                                                           %
%                                                                            %
%   \begin{cpar}{0}{F}{F}{0}{0}{} ... \end{cpar}                             %
%                                                                            %
% The environment is put around each line of the output for a PlusCal        %
% algorithm.                                                                 %
%%%%%%%%%%%%%%%%%%%%%%%%%%%%%%%%%%%%%%%%%%%%%%%%%%%%%%%%%%%%%%%%%%%%%%%%%%%%%%
%\newenvironment{ppar}{%
%  \ifthenelse{\boolean{shading}}{\par\begin{lrbox}{\tempsbox}%
%                                 \begin{minipage}[t]{\linewidth}}{}%
%  \begin{list}{}{%
%     \edef\temp{\thegmargin}
%        \setlength{\leftmargin}{0pt}%
%        \addtolength{\leftmargin}{\temp}%
%        \setlength{\itemindent}{0pt}%
%      \setlength{\labelwidth}{0pt}%
%      \setlength{\labelsep}{0pt}%
%      \setlength{\itemindent}{-\leftmargin}%
%      \setlength{\topsep}{0pt}%
%      \setlength{\parsep}{0pt}%
%      \setlength{\partopsep}{0pt}%
%      \setlength{\parskip}{0pt}%
%      \setlength{\itemsep}{0pt}
%      \setlength{\itemindent}{0pt}%
%      \addtolength{\itemindent}{-\leftmargin}}%
%      \item[\tstrut\hspace{\temp}]}%
% {\hspace{-\the\lastskip}\tstrut
% \end{list}%
%  \ifthenelse{\boolean{shading}}{\end{minipage}  
%                                 \end{lrbox}%
%                                 \shadebox{\usebox{\tempsbox}}\par}{}%
%  }

 %%% TESTING
 \newcommand{\xtest}[1]{\par
 \makebox[0pt][l]{\shadebox{\xtstrut\hspace*{\textwidth}}}%
 \mbox{$\mbox{}#1\mbox{}$}} 

% \newcommand{\xxtest}[1]{\par
% \makebox[0pt][l]{\shadebox{\xtstrut{#1}\hspace*{\textwidth}}}%
% \mbox{$\mbox{}#1\mbox{}$}} 

%\newlength{\pcalvspace}
%\setlength{\pcalvspace}{0pt}
% \newlength{\xxtestlen}
% \setlength{\xxtestlen}{0pt}
% \newcommand\xtstrut%
%   {\setlength{\xxtestlen}{1.15em}%
%    \addtolength{\xxtestlen}{\pcalvspace}%
%     \raisebox{\vshadelen}{\raisebox{-.25em}{\rule{0pt}{\xxtestlen}}}%
%    \global\setlength{\vshadelen}{0pt}%
%    \global\setlength{\pcalvspace}{0pt}}
   
   %%%% TESTING
   
   %% The xcpar environment
   %%  Note: overloaded use of \pcalvspace for testing.
   %%
%   \newlength{\xcparht}%
%   \newlength{\xcpardp}%
   
%   \newenvironment{xcpar}[6]{%
%  \addtocounter{pardepth}{-#1}%
%  \ifthenelse{\boolean{shading}}{\par\begin{lrbox}{\tempsbox}%
%                                 \begin{minipage}[t]{\linewidth}}{}%
%  \begin{list}{}{%
%     \edef\temp{\thegmargin}%
%     \ifthenelse{\equal{#3}{T}}%
%       {\settowidth{\leftmargin}{\hspace{\temp}\footnotesize #6\hspace{#5pt}}%
%        \addtolength{\leftmargin}{#4pt}}%
%       {\setlength{\leftmargin}{#4pt}%
%        \addtolength{\leftmargin}{#5pt}%
%        \addtolength{\leftmargin}{\temp}%
%        \setlength{\itemindent}{-#5pt}}%
%      \ifthenelse{\equal{#2}{T}}{\addtocounter{pardepth}{1}%
%                                 \setgmargin{\the\leftmargin}}{}%
%      \setlength{\labelwidth}{0pt}%
%      \setlength{\labelsep}{0pt}%
%      \setlength{\itemindent}{-\leftmargin}%
%      \setlength{\topsep}{0pt}%
%      \setlength{\parsep}{0pt}%
%      \setlength{\partopsep}{0pt}%
%      \setlength{\parskip}{0pt}%
%      \setlength{\itemsep}{0pt}%
%      \setlength{\itemindent}{#4pt}%
%      \addtolength{\itemindent}{-\leftmargin}}%
%   \ifthenelse{\equal{#3}{T}}%
%      {\item[\xtstrut\footnotesize \hspace{\temp}{#6}\hspace{#5pt}]%
%        }%
%      {\item[\xtstrut\hspace{\temp}]%
%         }%
%   \footnotesize}
% {\hspace{-\the\lastskip}\tstrut
% \end{list}%
%  \ifthenelse{\boolean{shading}}{\end{minipage}  
%                                 \end{lrbox}%
%                                 \setlength{\xcparht}{\ht\tempsbox}%
%                                 \setlength{\xcpardp}{\dp\tempsbox}%
%                                 \addtolength{\xcparht}{.15em}%
%                                 \addtolength{\xcpardp}{.2em}%
%                                 \addtolength{\xcparht}{\xcpardp}%
%                                 \hspace*{-\lcomindent}%
%                                 \makebox[0pt][l]{\raisebox{-\xcpardp}[0pt][0pt]{%
%                                      \shadebox{\rule{0pt}{\xcparht}\hspace*{\textwidth}}}}%
%                                 \hspace*{\lcomindent}\usebox{\tempsbox}%
%                                 \par}{}%
%  }
%  
% \newlength{\xmcomlen}
%\newenvironment{xmcom}[1]{%
%  \setcounter{pardepth}{0}%
%  \hspace{.65em}%
%  \begin{lrbox}{\alignbox}\sloppypar%
%      \setboolean{shading}{false}%
%      \setlength{\boxwidth}{#1pt}%
%      \addtolength{\boxwidth}{-.65em}%
%      \begin{minipage}[t]{\boxwidth}\footnotesize
%      \parskip=0pt\relax}%
%       {\end{minipage}\end{lrbox}%
%       \setlength{\xmcomlen}{\textwidth}%
%       \addtolength{\xmcomlen}{-\wd\alignbox}%
%       \settodepth{\alignwidth}{\usebox{\alignbox}}%
%       \global\setlength{\multicommentdepth}{\alignwidth}%
%       \setlength{\boxwidth}{\alignwidth}%
%       \global\addtolength{\alignwidth}{-\maxdepth}%
%       \addtolength{\boxwidth}{.1em}%
%       \raisebox{0pt}[0pt][0pt]{%
%        \ifthenelse{\boolean{shading}}%
%          {\hspace*{-\xmcomlen}\shadebox{\rule[-\boxwidth]{0pt}{0pt}%
%                                 \hspace*{\xmcomlen}\usebox{\alignbox}}}%
%          {\usebox{\alignbox}}}%
%       \vspace*{\alignwidth}\pagebreak[0]\vspace{-\alignwidth}\par}
% % a multi-line comment, whose first argument is its width in points.
%  
   
%%%%%%%%%%%%%%%%%%%%%%%%%%%%%%%%%%%%%%%%%%%%%%%%%%%%%%%%%%%%%%%%%%%%%%%%%%%%%%
% THE lcom ENVIRONMENT                                                       %
% ^^^^^^^^^^^^^^^^^^^^                                                       %
% A multi-line comment with no text to its left is typeset in an lcom        % 
% environment, whose argument is a number representing the indentation       % 
% of the left margin, in points.  All the text of the comment should be      % 
% inside cpar environments.                                                  % 
%%%%%%%%%%%%%%%%%%%%%%%%%%%%%%%%%%%%%%%%%%%%%%%%%%%%%%%%%%%%%%%%%%%%%%%%%%%%%%
\newenvironment{lcom}[1]{%
  \setlength{\lcomindent}{#1pt} % Added for PlusCal handling.
  \par\vspace{.2em}%
  \sloppypar
  \setcounter{pardepth}{0}%
  \footnotesize
  \begin{list}{}{%
    \setlength{\leftmargin}{#1pt}
    \setlength{\labelwidth}{0pt}%
    \setlength{\labelsep}{0pt}%
    \setlength{\itemindent}{0pt}%
    \setlength{\topsep}{0pt}%
    \setlength{\parsep}{0pt}%
    \setlength{\partopsep}{0pt}%
    \setlength{\parskip}{0pt}}
    \item[]}%
  {\end{list}\vspace{.3em}\setlength{\lcomindent}{0pt}%
 }


%%%%%%%%%%%%%%%%%%%%%%%%%%%%%%%%%%%%%%%%%%%%%%%%%%%%%%%%%%%%%%%%%%%%%%%%%%%%%
% THE mcom ENVIRONMENT AND \mutivspace COMMAND                              %
% ^^^^^^^^^^^^^^^^^^^^^^^^^^^^^^^^^^^^^^^^^^^^                              %
%                                                                           %
% A part of the spec containing a right-comment of the form                 %
%                                                                           %
%      xxxx (*************)                                                 %
%      yyyy (* ccccccccc *)                                                 %
%      ...  (* ccccccccc *)                                                 %
%           (* ccccccccc *)                                                 %
%           (* ccccccccc *)                                                 %
%           (*************)                                                 %
%                                                                           %
% is typeset by                                                             %
%                                                                           %
%     XXXX \begin{mcom}{d}                                                  %
%            CCCC ... CCC                                                   %
%          \end{mcom}                                                       %
%     YYYY ...                                                              %
%     \multivspace{n}                                                       %
%                                                                           %
% where the number d is the width in points of the comment, n is the        %
% number of xxxx, yyyy, ...  lines to the left of the comment.              %
% All the text of the comment should be typeset in cpar environments.       %
%                                                                           %
% This puts the comment into a single box (so no page breaks can occur      %
% within it).  The entire box is shaded iff the shading flag is true.       %
%%%%%%%%%%%%%%%%%%%%%%%%%%%%%%%%%%%%%%%%%%%%%%%%%%%%%%%%%%%%%%%%%%%%%%%%%%%%%
\newlength{\xmcomlen}%
\newenvironment{mcom}[1]{%
  \setcounter{pardepth}{0}%
  \hspace{.65em}%
  \begin{lrbox}{\alignbox}\sloppypar%
      \setboolean{shading}{false}%
      \setlength{\boxwidth}{#1pt}%
      \addtolength{\boxwidth}{-.65em}%
      \begin{minipage}[t]{\boxwidth}\footnotesize
      \parskip=0pt\relax}%
       {\end{minipage}\end{lrbox}%
       \setlength{\xmcomlen}{\textwidth}%       % For PlusCal shading
       \addtolength{\xmcomlen}{-\wd\alignbox}%  % For PlusCal shading
       \settodepth{\alignwidth}{\usebox{\alignbox}}%
       \global\setlength{\multicommentdepth}{\alignwidth}%
       \setlength{\boxwidth}{\alignwidth}%      % For PlusCal shading
       \global\addtolength{\alignwidth}{-\maxdepth}%
       \addtolength{\boxwidth}{.1em}%           % For PlusCal shading
      \raisebox{0pt}[0pt][0pt]{%
        \ifthenelse{\boolean{shading}}%
          {\ifpcalshading
             \hspace*{-\xmcomlen}%
             \shadebox{\rule[-\boxwidth]{0pt}{0pt}\hspace*{\xmcomlen}%
                          \usebox{\alignbox}}%
           \else
             \shadebox{\usebox{\alignbox}}
           \fi
          }%
          {\usebox{\alignbox}}}%
       \vspace*{\alignwidth}\pagebreak[0]\vspace{-\alignwidth}\par}
 % a multi-line comment, whose first argument is its width in points.


% \multispace{n} produces the vertical space indicated by "|"s in 
% this situation
%   
%     xxxx (*************)
%     xxxx (* ccccccccc *)
%      |   (* ccccccccc *)
%      |   (* ccccccccc *)
%      |   (* ccccccccc *)
%      |   (*************)
%
% where n is the number of "xxxx" lines.
\newcommand{\multivspace}[1]{\addtolength{\multicommentdepth}{-#1\baselineskip}%
 \addtolength{\multicommentdepth}{1.2em}%
 \ifthenelse{\lengthtest{\multicommentdepth > 0pt}}%
    {\par\vspace{\multicommentdepth}\par}{}}

%\newenvironment{hpar}[2]{%
%  \begin{list}{}{\setlength{\leftmargin}{#1pt}%
%                 \addtolength{\leftmargin}{#2pt}%
%                 \setlength{\itemindent}{-#2pt}%
%                 \setlength{\topsep}{0pt}%
%                 \setlength{\parsep}{0pt}%
%                 \setlength{\partopsep}{0pt}%
%                 \setlength{\parskip}{0pt}%
%                 \addtolength{\labelsep}{0pt}}%
%  \item[]\footnotesize}{\end{list}}
%    %%%%%%%%%%%%%%%%%%%%%%%%%%%%%%%%%%%%%%%%%%%%%%%%%%%%%%%%%%%%%%%%%%%%%%%%
%    % Typesets a sequence of paragraphs like this:                         %
%    %                                                                      %
%    %      left |<-- d1 --> XXXXXXXXXXXXXXXXXXXXXXXX                       %
%    %    margin |           <- d2 -> XXXXXXXXXXXXXXX                       %
%    %           |                    XXXXXXXXXXXXXXX                       %
%    %           |                                                          %
%    %           |                    XXXXXXXXXXXXXXX                       %
%    %           |                    XXXXXXXXXXXXXXX                       %
%    %                                                                      %
%    % where d1 = #1pt and d2 = #2pt, but with no vspace between            %
%    % paragraphs.                                                          %
%    %%%%%%%%%%%%%%%%%%%%%%%%%%%%%%%%%%%%%%%%%%%%%%%%%%%%%%%%%%%%%%%%%%%%%%%%

%%%%%%%%%%%%%%%%%%%%%%%%%%%%%%%%%%%%%%%%%%%%%%%%%%%%%%%%%%%%%%%%%%%%%%
% Commands for repeated characters that produce dashes.              %
%%%%%%%%%%%%%%%%%%%%%%%%%%%%%%%%%%%%%%%%%%%%%%%%%%%%%%%%%%%%%%%%%%%%%%
% \raisedDash{wd}{ht}{thk} makes a horizontal line wd characters wide, 
% raised a distance ht ex's above the baseline, with a thickness of 
% thk em's.
\newcommand{\raisedDash}[3]{\raisebox{#2ex}{\setlength{\alignwidth}{.5em}%
  \rule{#1\alignwidth}{#3em}}}

% The following commands take a single argument n and produce the
% output for n repeated characters, as follows
%   \cdash:    -
%   \tdash:    ~
%   \ceqdash:  =
%   \usdash:   _
\newcommand{\cdash}[1]{\raisedDash{#1}{.5}{.04}}
\newcommand{\usdash}[1]{\raisedDash{#1}{0}{.04}}
\newcommand{\ceqdash}[1]{\raisedDash{#1}{.5}{.08}}
\newcommand{\tdash}[1]{\raisedDash{#1}{1}{.08}}

\newlength{\spacewidth}
\setlength{\spacewidth}{.2em}
\newcommand{\e}[1]{\hspace{#1\spacewidth}}
%% \e{i} produces space corresponding to i input spaces.


%% Alignment-file Commands

\newlength{\alignboxwidth}
\newlength{\alignwidth}
\newsavebox{\alignbox}

% \al{i}{j}{txt} is used in the alignment file to put "%{i}{j}{wd}"
% in the log file, where wd is the width of the line up to that point,
% and txt is the following text.
\newcommand{\al}[3]{%
  \typeout{\%{#1}{#2}{\the\alignwidth}}%
  \cl{#3}}

%% \cl{txt} continues a specification line in the alignment file
%% with text txt.
\newcommand{\cl}[1]{%
  \savebox{\alignbox}{\mbox{$\mbox{}#1\mbox{}$}}%
  \settowidth{\alignboxwidth}{\usebox{\alignbox}}%
  \addtolength{\alignwidth}{\alignboxwidth}%
  \usebox{\alignbox}}

% \fl{txt} in the alignment file begins a specification line that
% starts with the text txt.
\newcommand{\fl}[1]{%
  \par
  \savebox{\alignbox}{\mbox{$\mbox{}#1\mbox{}$}}%
  \settowidth{\alignwidth}{\usebox{\alignbox}}%
  \usebox{\alignbox}}



  
%%%%%%%%%%%%%%%%%%%%%%%%%%%%%%%%%%%%%%%%%%%%%%%%%%%%%%%%%%%%%%%%%%%%%%%%%%%%%
% Ordinarily, TeX typesets letters in math mode in a special math italic    %
% font.  This makes it typeset "it" to look like the product of the         %
% variables i and t, rather than like the word "it".  The following         %
% commands tell TeX to use an ordinary italic font instead.                 %
%%%%%%%%%%%%%%%%%%%%%%%%%%%%%%%%%%%%%%%%%%%%%%%%%%%%%%%%%%%%%%%%%%%%%%%%%%%%%
\ifx\documentclass\undefined
\else
  \DeclareSymbolFont{tlaitalics}{\encodingdefault}{cmr}{m}{it}
  \let\itfam\symtlaitalics
\fi

\makeatletter
\newcommand{\tlx@c}{\c@tlx@ctr\advance\c@tlx@ctr\@ne}
\newcounter{tlx@ctr}
\c@tlx@ctr=\itfam \multiply\c@tlx@ctr"100\relax \advance\c@tlx@ctr "7061\relax
\mathcode`a=\tlx@c \mathcode`b=\tlx@c \mathcode`c=\tlx@c \mathcode`d=\tlx@c
\mathcode`e=\tlx@c \mathcode`f=\tlx@c \mathcode`g=\tlx@c \mathcode`h=\tlx@c
\mathcode`i=\tlx@c \mathcode`j=\tlx@c \mathcode`k=\tlx@c \mathcode`l=\tlx@c
\mathcode`m=\tlx@c \mathcode`n=\tlx@c \mathcode`o=\tlx@c \mathcode`p=\tlx@c
\mathcode`q=\tlx@c \mathcode`r=\tlx@c \mathcode`s=\tlx@c \mathcode`t=\tlx@c
\mathcode`u=\tlx@c \mathcode`v=\tlx@c \mathcode`w=\tlx@c \mathcode`x=\tlx@c
\mathcode`y=\tlx@c \mathcode`z=\tlx@c
\c@tlx@ctr=\itfam \multiply\c@tlx@ctr"100\relax \advance\c@tlx@ctr "7041\relax
\mathcode`A=\tlx@c \mathcode`B=\tlx@c \mathcode`C=\tlx@c \mathcode`D=\tlx@c
\mathcode`E=\tlx@c \mathcode`F=\tlx@c \mathcode`G=\tlx@c \mathcode`H=\tlx@c
\mathcode`I=\tlx@c \mathcode`J=\tlx@c \mathcode`K=\tlx@c \mathcode`L=\tlx@c
\mathcode`M=\tlx@c \mathcode`N=\tlx@c \mathcode`O=\tlx@c \mathcode`P=\tlx@c
\mathcode`Q=\tlx@c \mathcode`R=\tlx@c \mathcode`S=\tlx@c \mathcode`T=\tlx@c
\mathcode`U=\tlx@c \mathcode`V=\tlx@c \mathcode`W=\tlx@c \mathcode`X=\tlx@c
\mathcode`Y=\tlx@c \mathcode`Z=\tlx@c
\makeatother

%%%%%%%%%%%%%%%%%%%%%%%%%%%%%%%%%%%%%%%%%%%%%%%%%%%%%%%%%%
%                THE describe ENVIRONMENT                %
%%%%%%%%%%%%%%%%%%%%%%%%%%%%%%%%%%%%%%%%%%%%%%%%%%%%%%%%%%
%
%
% It is like the description environment except it takes an argument
% ARG that should be the text of the widest label.  It adjusts the
% indentation so each item with label LABEL produces
%%      LABEL             blah blah blah
%%      <- width of ARG ->blah blah blah
%%                        blah blah blah
\newenvironment{describe}[1]%
   {\begin{list}{}{\settowidth{\labelwidth}{#1}%
            \setlength{\labelsep}{.5em}%
            \setlength{\leftmargin}{\labelwidth}% 
            \addtolength{\leftmargin}{\labelsep}%
            \addtolength{\leftmargin}{\parindent}%
            \def\makelabel##1{\rm ##1\hfill}}%
            \setlength{\topsep}{0pt}}%% 
                % Sets \topsep to 0 to reduce vertical space above
                % and below embedded displayed equations
   {\end{list}}

%   For tlatex.TeX
\usepackage{verbatim}
\makeatletter
\def\tla{\let\%\relax%
         \@bsphack
         \typeout{\%{\the\linewidth}}%
             \let\do\@makeother\dospecials\catcode`\^^M\active
             \let\verbatim@startline\relax
             \let\verbatim@addtoline\@gobble
             \let\verbatim@processline\relax
             \let\verbatim@finish\relax
             \verbatim@}
\let\endtla=\@esphack

\let\pcal=\tla
\let\endpcal=\endtla
\let\ppcal=\tla
\let\endppcal=\endtla

% The tlatex environment is used by TLATeX.TeX to typeset TLA+.
% TLATeX.TLA starts its files by writing a \tlatex command.  This
% command/environment sets \parindent to 0 and defines \% to its
% standard definition because the writing of the log files is messed up
% if \% is defined to be something else.  It also executes
% \@computerule to determine the dimensions for the TLA horizonatl
% bars.
\newenvironment{tlatex}{\@computerule%
                        \setlength{\parindent}{0pt}%
                       \makeatletter\chardef\%=`\%}{}


% The notla environment produces no output.  You can turn a 
% tla environment to a notla environment to prevent tlatex.TeX from
% re-formatting the environment.

\def\notla{\let\%\relax%
         \@bsphack
             \let\do\@makeother\dospecials\catcode`\^^M\active
             \let\verbatim@startline\relax
             \let\verbatim@addtoline\@gobble
             \let\verbatim@processline\relax
             \let\verbatim@finish\relax
             \verbatim@}
\let\endnotla=\@esphack

\let\nopcal=\notla
\let\endnopcal=\endnotla
\let\noppcal=\notla
\let\endnoppcal=\endnotla

%%%%%%%%%%%%%%%%%%%%%%%% end of tlatex.sty file %%%%%%%%%%%%%%%%%%%%%%% 
% last modified on Fri  3 August 2012 at 14:23:49 PST by lamport

\begin{document}
\tlatex
\setboolean{shading}{true}
\@x{}\moduleLeftDash\@xx{ {\MODULE} cbccasper\_binary}\moduleRightDash\@xx{}%
\@x{ {\EXTENDS} Integers ,\, Sequences ,\, FiniteSets ,\, TLC}%
\@pvspace{8.0pt}%
\@x{}%
\@y{\@s{0}%
 The first four are parameters of the protocol, and the rest are defined for
 purposes of model checking
}%
\@xx{}%
\@x{}%
\@y{\@s{7.5}%
 \ensuremath{\.{-} validators}: a set specifying the names (consecutive
 integers) of the \ensuremath{validators
}}%
\@xx{}%
\@x{}%
\@y{\@s{7.5}%
 \ensuremath{\.{-} validator\_weights}: a tuple assigning a weight (integer)
 to each \ensuremath{validator
}}%
\@xx{}%
\@x{}%
\@y{\@s{7.5}%
 \ensuremath{\.{-} byzantine\_threshold}: a number less than the sum of the
 weights of all the \ensuremath{validators
}}%
\@xx{}%
\@x{}%
\@y{\@s{7.5}%
 \ensuremath{\.{-} consensus\_values}: the set of values the
 \ensuremath{validators} decide on; the binary values 0 and 1 in this model
}%
\@xx{}%
\@x{}%
\@y{\@s{7.5}%
 \ensuremath{\.{-} validators\_initial\_values}: the estimates without
 receiving other messages
}%
\@xx{}%
\@x{}%
\@y{\@s{7.5}%
 \ensuremath{\.{-} message\_ids}: a number used to limit the number of
 messages sent in the model
}%
\@xx{}%
\@x{}%
\@y{\@s{7.5}%
 \ensuremath{\.{-} byzantine\_fault\_nodes}: define the equivocating nodes
}%
\@xx{}%
\@x{ {\CONSTANTS}}%
\@x{\@s{16.4} validators ,\,}%
\@x{\@s{16.4} validator\_weights ,\,}%
\@x{\@s{16.4} byzantine\_threshold ,\,}%
\@x{\@s{16.4} consensus\_values ,\,}%
\@x{\@s{16.4} validators\_initial\_values ,\,}%
\@x{\@s{16.4} message\_ids ,\,}%
\@x{\@s{16.4} byzantine\_fault\_nodes}%
\@pvspace{8.0pt}%
\@x{\@s{16.4} Even \.{\defeq} \{ i \.{\in} Integers \.{:} i mod 2 \.{=} 0 \}}%
\@pvspace{8.0pt}%
\@x{}%
\@y{\@s{0}%
 The following is written in \ensuremath{PlusCal}, which will be transpiled
 to TLA+.
}%
\@xx{}%
\@x{}%
\@y{\@s{0}%
 The transpiled TLA+ code is appended to the \ensuremath{PlusCal} code, and
 the \ensuremath{PlusCal} code will appear as comments.
}%
\@xx{}%
\@pvspace{8.0pt}%
\pcalsymbolstrue
\pcalshadingtrue
\csyntaxfalse
\@x{\@s{12.29} {\p@mmalgorithm} algo}%
\@x{}%
\@y{\@s{7.5}%
 variables are updated as the model checker runs:
}%
\@xx{}%
\@x{}%
\@y{\@s{15.0}%
 \ensuremath{\.{-} all\_msg}: a set of all messages ids (integers)
}%
\@xx{}%
\@x{}%
\@y{\@s{15.0}%
 \ensuremath{\.{-} equivocating\_msg}: a set specifying all the equivocating
 messages
}%
\@xx{}%
\@x{}%
\@y{\@s{62.5}%
 (not all the \ensuremath{validators} receive that same message)
}%
\@xx{}%
\@x{}%
\@y{\@s{15.0}%
 \ensuremath{\.{-} msg\_sender}: a tuple specifying the sender of the message
 with the given id
}%
\@xx{}%
\@x{}%
\@y{\@s{15.0}%
 \ensuremath{\.{-} msg\_estimate}: a tuple specifying the estimate of the
 message with the given id
}%
\@xx{}%
\@x{}%
\@y{\@s{15.0}%
 \ensuremath{\.{-} msg\_justification}: a tuple specifying the justification
 of the message
}%
\@xx{}%
\@x{}%
\@y{\@s{65.0}%
 (set of messages used to calculate the estimate) with the given id,
}%
\@xx{}%
\@x{}%
\@y{\@s{15.0}%
 \ensuremath{\.{-} cur\_msg\_id}: the id of the current message being sent;
 incremented after every message
}%
\@xx{}%
\@x{}%
\@y{\@s{15.0}%
 \ensuremath{\.{-} validator\_init\_done}: a tuple indicating whether the
 \ensuremath{validator} has sent the initial message
}%
\@xx{}%
\@x{}%
\@y{\@s{70.0}%
 (1 is done; all the \ensuremath{validators} are initialized to 0 in the
 beginning)
}%
\@xx{}%
\@x{}%
\@y{\@s{15.0}%
 \ensuremath{\.{-} equiv\_msg\_receivers}: a tuple specifying a subset of all
 \ensuremath{validators} receiving a certain equivocating message;
}%
\@xx{}%
\@x{}%
\@y{\@s{72.5}%
 if the message is not equivocating append the empty set
}%
\@xx{}%
\@x{}%
\@y{\@s{15.0}%
 \ensuremath{\.{-} cur\_subset}: a \mbox{'}temp\mbox{'} variable used to
 store the set of \ensuremath{validators} receiving the current equivocating
 message
}%
\@xx{}%
\@x{\@s{16.4} {\p@variables}}%
\@x{\@s{32.8} all\_msg \.{=} \{ \} ,\,}%
\@x{\@s{32.8} equivocating\_msg \.{=} \{ \} ,\,}%
\@x{\@s{32.8} msg\_sender \.{=} {\langle} {\rangle} ,\,}%
\@x{\@s{32.8} msg\_estimate \.{=} {\langle} {\rangle} ,\,}%
\@x{\@s{32.8} msg\_justification \.{=} {\langle} {\rangle} ,\,}%
\@x{\@s{32.8} cur\_msg\_id \.{=} 1 ,\,}%
 \@x{\@s{32.8} validator\_init\_done\@s{1.92} \.{=} {\langle} 0 ,\, 0 ,\, 0
 ,\, 0 ,\, 0 ,\, 0 ,\, 0 ,\, 0 ,\, 0 ,\, 0 {\rangle} ,\,}%
\@x{\@s{32.8} equiv\_msg\_receivers \.{=} {\langle} {\rangle} ,\,}%
\@x{\@s{32.8} cur\_subset}%
\@pvspace{8.0pt}%
\@x{\@s{16.4} {\p@define}}%
\@x{}%
\@y{\@s{15.0}%
 The dependencies of a message \ensuremath{m1} are the messages in the
 justification of \ensuremath{m1
}}%
\@xx{}%
\@x{}%
\@y{\@s{15.0}%
 and in the justifications of the justifications of \ensuremath{m1} and so
 on. The set is generated
}%
\@xx{}%
\@x{}%
\@y{\@s{15.0}%
 using a recursive function until the base case is reached - the only
 justification of
}%
\@xx{}%
\@x{}%
\@y{\@s{15.0}%
 a message is itself.
}%
\@xx{}%
\@x{\@s{36.89} dependencies ( message ) \.{\defeq}}%
\@x{\@s{49.19} \.{\LET}}%
\@x{\@s{65.6} {\RECURSIVE} dep ( \_ )}%
\@x{\@s{65.6} dep ( msg ) \.{\defeq}}%
 \@x{\@s{84.94} {\IF} Cardinality ( msg\_justification [ msg ] ) \.{=} 1
 \.{\land} msg \.{\in} msg\_justification [ msg ]}%
\@x{\@s{84.94} \.{\THEN} \{ msg \}}%
 \@x{\@s{84.94} \.{\ELSE} {\UNION} \{ dep ( msg2 ) \.{:} msg2 \.{\in}
 msg\_justification [ msg ] \}}%
\@x{\@s{49.19} \.{\IN} dep ( message )}%
\@pvspace{8.0pt}%
\@x{}%
\@y{\@s{12.5}%
 Gets the set of dependencies of all the messages in a set of messages.
}%
\@xx{}%
 \@x{\@s{32.8} dep\_set ( messages ) \.{\defeq} messages \.{\cup} {\UNION} \{
 dependencies ( m ) \.{:} m \.{\in} messages \}}%
\@pvspace{8.0pt}%
\@x{}%
\@y{\@s{12.5}%
 The latest message of a \ensuremath{validator} in an observed set of
 messages is the message for which
}%
\@xx{}%
\@x{}%
\@y{\@s{12.5}%
 no other messages sent by the same \ensuremath{validator} justifies it.
}%
\@xx{}%
\@x{\@s{32.8} latest\_message ( validator ,\, messages ) \.{\defeq}}%
\@x{\@s{49.19} \{ msg \.{\in} messages \.{:}}%
\@x{\@s{66.50} \.{\land} msg\_sender [ msg ] \.{=} validator}%
\@x{\@s{66.50} \.{\land} {\lnot} \E\, msg2 \.{\in} messages \.{:}}%
\@x{\@s{84.27} \.{\land} msg\_sender [ msg2 ] \.{=} validator}%
\@x{\@s{84.27} \.{\land} msg \.{\neq} msg2}%
\@x{\@s{84.27} \.{\land} msg \.{\in} dependencies ( msg2 )}%
\@x{\@s{49.19} \}}%
\@pvspace{8.0pt}%
\@x{}%
\@y{\@s{12.5}%
 Defines the \ensuremath{Pick} operator used to choose an arbitrary element
 from a set.
}%
\@xx{}%
\@x{\@s{32.8} Pick ( S ) \.{\defeq} {\CHOOSE} s \.{\in} S \.{:} {\TRUE}}%
\@pvspace{8.0pt}%
\@x{}%
\@y{\@s{12.5}%
 Defines the \ensuremath{Sum} operator used to find the sum of all the
 elements in a set.
}%
\@xx{}%
\@x{\@s{32.8} {\RECURSIVE} SetReduce ( \_ ,\, \_ ,\, \_ )}%
\@x{\@s{49.19} SetReduce ( Op ( \_ ,\, \_ ) ,\, S ,\, value ) \.{\defeq}}%
\@x{\@s{65.6} {\IF} S \.{=} \{ \} \.{\THEN} value}%
 \@x{\@s{65.6} \.{\ELSE} \.{\LET} s \.{\defeq} Pick ( S ) \.{\IN} SetReduce (
 Op ,\, S \.{\,\backslash\,} \{ s \} ,\, Op ( s ,\, value ) )}%
\@pvspace{8.0pt}%
 \@x{\@s{41.0} Sum ( S ) \.{\defeq} \.{\LET} \_op ( a ,\, b ) \.{\defeq} a
 \.{+} b \.{\IN} SetReduce ( \_op ,\, S ,\, 0 )}%
\@pvspace{8.0pt}%
\@x{}%
\@y{\@s{12.5}%
 The following defines the estimator used in the binary consesnsus protocol.
}%
\@xx{}%
\@x{}%
\@y{\@s{12.5}%
 The score of an estimate is the sum of the weights of all the
 \ensuremath{validators} having
}%
\@xx{}%
\@x{}%
\@y{\@s{12.5}%
 the given estimate in its latest message in a set of observed messages.
}%
\@xx{}%
\@x{}%
\@y{\@s{12.5}%
 The estimator returns the estimate with the larger score. If there\mbox{'}s
 a tie,
}%
\@xx{}%
\@x{}%
\@y{\@s{12.5}%
 in this example, the value 0 is used.
}%
\@xx{}%
\@x{\@s{32.8} score ( estimate ,\, messages ) \.{\defeq}}%
\@x{\@s{49.19} \.{\LET} ss \.{\defeq}}%
\@x{\@s{69.59} \{ v \.{\in} validators \.{:}}%
 \@x{\@s{84.37} \.{\land} Cardinality ( latest\_message ( v ,\, messages ) )
 \.{=} 1}%
 \@x{\@s{84.37} \.{\land} \E\, m \.{\in} latest\_message ( v ,\, messages )
 \.{:}}%
\@x{\@s{99.58} msg\_estimate [ m ] \.{=} estimate \}}%
 \@x{\@s{69.59} ss2 \.{\defeq} \{ validator\_weights [ v ] \.{:} v \.{\in} ss
 \}}%
\@x{\@s{49.19} \.{\IN} Sum ( ss2 )}%
\@x{\@s{32.8} binary\_estimator ( messages ) \.{\defeq}}%
\@x{\@s{49.19} {\IF} score ( 1 ,\, messages ) \.{>} score ( 0 ,\, messages )}%
\@x{\@s{49.19} \.{\THEN} 1}%
\@x{\@s{49.19} \.{\ELSE} 0}%
\@pvspace{8.0pt}%
\@x{}%
\@y{\@s{12.5}%
 Two messages are equivocating if they have the same sender but do not
 justify each other.
}%
\@xx{}%
\@x{\@s{32.8} equivocation ( m1 ,\, m2 ) \.{\defeq}}%
\@x{\@s{49.19} \.{\land} m1 \.{\neq} m2}%
\@x{\@s{49.19} \.{\land} msg\_sender [ m1 ] \.{=} msg\_sender [ m2 ]}%
\@x{\@s{49.19} \.{\land} m1 \.{\notin} dependencies ( m2 )}%
\@x{\@s{49.19} \.{\land} m2 \.{\notin} dependencies ( m1 )}%
\@pvspace{8.0pt}%
\@x{}%
\@y{\@s{12.5}%
 A \ensuremath{validator} is byzantine if it sends equivocating messages.
}%
\@xx{}%
\@x{\@s{32.8} faulty\_node ( validator ,\, messages ) \.{\defeq}}%
\@x{\@s{49.19} \.{\land} \E\, m1 \.{\in} dep\_set ( messages ) \.{:}}%
\@x{\@s{64.41} \.{\land} \E\, m2 \.{\in} dep\_set ( messages ) \.{:}}%
\@x{\@s{79.62} \.{\land} validator \.{=} msg\_sender [ m1 ]}%
\@x{\@s{79.62} \.{\land} equivocation ( m1 ,\, m2 )}%
\@pvspace{8.0pt}%
\@x{}%
\@y{\@s{12.5}%
 Set of byzantine \ensuremath{validators} in an observed set of messages.
}%
\@xx{}%
 \@x{\@s{32.8} faulty\_nodes ( messages ) \.{\defeq} \{ v \.{\in} validators
 \.{:} faulty\_node ( v ,\, messages ) \}}%
\@pvspace{8.0pt}%
\@x{}%
\@y{\@s{12.5}%
 Returns the total weight of all byzatine \ensuremath{validators}.
}%
\@xx{}%
\@x{\@s{32.8} fault\_weight ( messages ) \.{\defeq}}%
\@x{\@s{49.19} \.{\LET} byz \.{\defeq} faulty\_nodes ( messages )}%
 \@x{\@s{49.19} \.{\IN} Sum ( \{ validator\_weights [ v ] \.{:} v \.{\in} byz
 \} )}%
\@pvspace{8.0pt}%
\@x{}%
\@y{\@s{12.5}%
 For a given fault tolerance threshold \ensuremath{t}, a protocol state is
 the set of messages
}%
\@xx{}%
\@x{}%
\@y{\@s{12.5}%
 with fault weight less than \ensuremath{t}.
}%
\@xx{}%
\@x{}%
\@y{\@s{12.5}%
 A state transition is possible if one state is a subset of another.
}%
\@xx{}%
 \@x{\@s{32.8} protocol\_states ( messages ,\, t ) \.{\defeq} \{ s \.{\in}
 {\SUBSET} ( messages ) \.{:} fault\_weight ( s ) \.{<} t \}}%
 \@x{\@s{32.8} protocol\_executions ( state1 ,\, state2 ) \.{\defeq} state1
 \.{\subseteq} state2}%
\@pvspace{8.0pt}%
\@x{}%
\@y{\@s{12.5}%
 Two \ensuremath{validators} \ensuremath{v1} and \ensuremath{v2} are agreeing
 with each other if:
}%
\@xx{}%
\@x{}%
\@y{\@s{17.5}%
 \ensuremath{\.{-} v1} has exactly one latest message in messages
}%
\@xx{}%
\@x{}%
\@y{\@s{17.5}%
 \ensuremath{\.{-} v2} has exactly one latest message in the justification of
 \ensuremath{v1}\mbox{'}s latest message
}%
\@xx{}%
\@x{}%
\@y{\@s{17.5}%
 - the latest messages have estimates that agree with each other
}%
\@xx{}%
 \@x{\@s{32.8} validators\_agreeing ( v1 ,\, v2 ,\, estimate ,\, messages )
 \.{\defeq}}%
 \@x{\@s{49.19} \.{\land} Cardinality ( latest\_message ( v1 ,\, messages ) )
 \.{=} 1}%
 \@x{\@s{49.19} \.{\land} \.{\LET} v1\_latest\_msg \.{\defeq} Pick (
 latest\_message ( v1 ,\, messages ) )}%
 \@x{\@s{64.41} \.{\IN} Cardinality ( latest\_message ( v2 ,\,
 msg\_justification [ v1\_latest\_msg ] ) ) \.{=} 1}%
 \@x{\@s{64.41} \.{\land}\@s{9.28} \.{\LET} v2\_latest\_msg \.{\defeq} Pick (
 latest\_message ( v2 ,\, msg\_justification [ v1\_latest\_msg ] ) )}%
\@x{\@s{88.91} \.{\IN} estimate \.{=} msg\_estimate [ v2\_latest\_msg ]}%
\@pvspace{8.0pt}%
\@x{}%
\@y{\@s{12.5}%
 Two \ensuremath{validators} are disagreeing with each other if:
}%
\@xx{}%
\@x{}%
\@y{\@s{17.5}%
 \ensuremath{\.{-} v1} has exactly one latest message in messages
}%
\@xx{}%
\@x{}%
\@y{\@s{17.5}%
 \ensuremath{\.{-} v2} has exactly one latest message in the justification of
 \ensuremath{v1}\mbox{'}s latest message
}%
\@xx{}%
\@x{}%
\@y{\@s{17.5}%
 \ensuremath{\.{-} v2} has a new latest message that doens\mbox{'}t agree
 with the estimate
}%
\@xx{}%
 \@x{\@s{32.8} validators\_disagreeing ( v1 ,\, v2 ,\, estimate ,\, messages )
 \.{\defeq}}%
 \@x{\@s{49.19} \.{\land} Cardinality ( latest\_message ( v1 ,\, messages ) )
 \.{=} 1}%
 \@x{\@s{49.19} \.{\land} \.{\LET} v1\_latest\_msg \.{\defeq} {\CHOOSE} x
 \.{\in} latest\_message ( v1 ,\, messages ) \.{:} {\TRUE}}%
 \@x{\@s{64.41} \.{\IN} Cardinality ( latest\_message ( v2 ,\,
 msg\_justification [ v1\_latest\_msg ] ) ) \.{=} 1}%
 \@x{\@s{64.41} \.{\land}\@s{9.28} \.{\LET} v2\_latest\_msg \.{\defeq}
 {\CHOOSE} x \.{\in} latest\_message ( v2 ,\, msg\_justification [
 v1\_latest\_msg ] ) \.{:} {\TRUE}}%
 \@x{\@s{88.91} \.{\IN} \E\, m \.{\in} messages \.{:} v2\_latest\_msg \.{\in}
 dependencies ( m )}%
\@x{\@s{113.41} \.{\land} estimate \.{\neq} msg\_estimate [ m ]}%
\@pvspace{16.0pt}%
\@x{}%
\@y{\@s{17.5}%
 An ``e-clique'' is a group of non-byzantine nodes in a set of observed
 messages such that :
}%
\@xx{}%
\@x{}%
\@y{\@s{27.5}%
 - they mutually see each other agreeing with estimate in messages
}%
\@xx{}%
\@x{}%
\@y{\@s{27.5}%
 - they mutually cannot see each other disagreeing with estimate in messages
}%
\@xx{}%
\@x{\@s{32.8} e\_clique ( estimate ,\, messages ) \.{\defeq} \{}%
\@x{\@s{49.19} ss \.{\in} {\SUBSET} ( validators ) \.{:}}%
\@x{\@s{62.29} \.{\land} \A\, v1 \.{\in} ss \.{:}}%
\@x{\@s{77.50} \.{\land} \A\, v2 \.{\in} ss \.{:}}%
\@x{\@s{92.72} {\IF} v1 \.{=} v2}%
\@x{\@s{92.72} \.{\THEN} {\TRUE}}%
\@x{\@s{92.72} \.{\ELSE}}%
 \@x{\@s{109.12} \.{\land} validators\_agreeing ( v1 ,\, v2 ,\, estimate ,\,
 messages )}%
 \@x{\@s{109.12} \.{\land} {\lnot} validators\_disagreeing ( v1 ,\, v2 ,\,
 estimate ,\, messages )}%
\@x{\@s{109.12}}%
\@y{\@s{0}%
 isn\mbox{'}t this just \ensuremath{\.{\lor}}\.{?}
}%
\@xx{}%
\@x{\@s{49.19} \}}%
\@x{}%
\@y{\@s{17.5}%
 Finds the existence of an e-clique
}%
\@xx{}%
 \@x{\@s{32.8} e\_clique\_estimate\_safety ( estimate ,\, messages )
 \.{\defeq}}%
 \@x{\@s{49.19} \.{\land} \E\, ss \.{\in} e\_clique ( estimate ,\, messages )
 \.{:}}%
 \@x{\@s{64.41} \.{\land} 2 \.{*} Sum ( \{ validator\_weights [ v ] \.{:} v
 \.{\in} ss \} ) \.{>} Sum ( \{ validator\_weights [ v ] \.{:} v \.{\in}
 validators \} ) \.{+} byzantine\_threshold \.{-} fault\_weight ( messages )}%
\@pvspace{8.0pt}%
\@x{}%
\@y{\@s{17.5}%
 Gets the set of messges received by a particular valiadtor
}%
\@xx{}%
\@x{\@s{32.8} validator\_received\_msg ( validator ) \.{\defeq}}%
\@x{\@s{49.19} ( all\_msg \.{\,\backslash\,} equivocating\_msg ) \.{\cup}}%
 \@x{\@s{49.19} \{ x \.{\in} equivocating\_msg \.{:} validator \.{\in}
 equiv\_msg\_receivers [ x ] \}}%
\@pvspace{8.0pt}%
\@x{}%
\@y{\@s{17.5}%
 Returns a subset of received messages for an equivocating
 \ensuremath{validator} to generate potentially equivocating messages
}%
\@xx{}%
 \@x{\@s{32.8} get\_equivocation\_subset\_msg ( validator ) \.{\defeq}
 {\CHOOSE} x \.{\in} {\SUBSET} ( validator\_received\_msg ( validator ) )
 \.{:} {\TRUE}}%
\@pvspace{8.0pt}%
\@x{}%
\@y{\@s{17.5}%
 Returns a subset of \ensuremath{validators} receiving a particular
 equivocating message
}%
\@xx{}%
 \@x{\@s{32.8} get\_equiv\_receivers ( validator ) \.{\defeq} {\CHOOSE} x
 \.{\in} ( {\SUBSET} ( validators \.{\,\backslash\,} \{ validator \} ) )
 \.{:} x \.{\neq} \{ \}}%
\@x{}%
\@y{\@s{17.5}%
 A temporal property checking that finality can eventually be reached in a
 binary consensus protocol
}%
\@xx{}%
\@x{\@s{32.8} check\_safety\_with\_oracle \.{\defeq}}%
 \@x{\@s{49.19} \.{\LET} v \.{\defeq} {\CHOOSE} v \.{\in} ( validators
 \.{\,\backslash\,} byzantine\_fault\_nodes ) \.{:} {\TRUE}}%
 \@x{\@s{49.19} \.{\IN} {\Diamond} ( e\_clique\_estimate\_safety ( 0 ,\,
 validator\_received\_msg ( v ) ) \.{\lor} e\_clique\_estimate\_safety ( 1
 ,\, validator\_received\_msg ( v ) ) )}%
\@pvspace{8.0pt}%
\@x{\@s{16.4} {\p@end} {\p@define} {\p@semicolon}}%
\@x{}%
\@y{\@s{7.5}%
 A message from a non-byzantine \ensuremath{validator} is received by all the
 \ensuremath{validators
}}%
\@xx{}%
 \@x{\@s{16.4} {\p@macro} make\_message ( validator ,\, estimate ,\,
 justification ) {\p@begin}}%
 \@x{\@s{59.20} equiv\_msg\_receivers \.{:=} Append ( equiv\_msg\_receivers
 ,\, \{ \} ) {\p@semicolon}}%
 \@x{\@s{59.20} msg\_sender \.{:=} Append ( msg\_sender ,\, validator )
 {\p@semicolon}}%
 \@x{\@s{59.20} msg\_estimate \.{:=} Append ( msg\_estimate ,\, estimate )
 {\p@semicolon}}%
 \@x{\@s{59.20} msg\_justification \.{:=} Append ( msg\_justification ,\,
 justification ) {\p@semicolon}}%
 \@x{\@s{59.20} all\_msg \.{:=} all\_msg \.{\cup} \{ cur\_msg\_id \}
 {\p@semicolon}}%
\@x{\@s{59.20} cur\_msg\_id \.{:=} cur\_msg\_id \.{+} 1 {\p@semicolon}}%
\@x{\@s{16.4} {\p@end} {\p@macro} {\p@semicolon}}%
\@pvspace{8.0pt}%
\@x{}%
\@y{\@s{7.5}%
 A \ensuremath{validators} sends an initial message without receiving
 information from other \ensuremath{validators
}}%
\@xx{}%
\@x{}%
\@y{\@s{7.5}%
 An initial message is only justified by itself
}%
\@xx{}%
\@x{\@s{16.4} {\p@macro} init\_validator ( validator ) {\p@begin}}%
 \@x{\@s{32.8} make\_message ( validator ,\, validators\_initial\_values [
 validator ] ,\, \{ cur\_msg\_id \} ) {\p@semicolon}}%
\@x{\@s{32.8} validator\_init\_done [ validator ] \.{:=} 1 {\p@semicolon}}%
\@x{\@s{16.4} {\p@end} {\p@macro} {\p@semicolon}}%
\@pvspace{8.0pt}%
\@x{}%
\@y{\@s{7.5}%
 An equivocating \ensuremath{validator} takes different subsets of its
 received messages to generate
}%
\@xx{}%
\@x{}%
\@y{\@s{7.5}%
 different estimates and sends the different messages to different subsets of
 \ensuremath{validators}.
}%
\@xx{}%
 \@x{\@s{16.4} {\p@macro} make\_equivocating\_messages ( validator )
 {\p@begin}}%
 \@x{\@s{32.8} equiv\_msg\_receivers \.{:=} Append ( equiv\_msg\_receivers ,\,
 get\_equiv\_receivers ( validator ) ) {\p@semicolon}}%
 \@x{\@s{32.8} cur\_subset \.{:=} get\_equivocation\_subset\_msg ( validator )
 {\p@semicolon}}%
 \@x{\@s{32.8} equivocating\_msg \.{:=} equivocating\_msg \.{\cup} \{
 cur\_msg\_id \} {\p@semicolon}}%
 \@x{\@s{32.8} msg\_sender \.{:=} Append ( msg\_sender ,\, validator )
 {\p@semicolon}}%
 \@x{\@s{32.8} msg\_estimate \.{:=} Append ( msg\_estimate ,\,
 binary\_estimator ( cur\_subset ) ) {\p@semicolon}}%
 \@x{\@s{32.8} msg\_justification \.{:=} Append ( msg\_justification ,\,
 cur\_subset ) {\p@semicolon}}%
 \@x{\@s{32.8} all\_msg \.{:=} all\_msg \.{\cup} \{ cur\_msg\_id \}
 {\p@semicolon}}%
\@x{\@s{32.8} cur\_msg\_id \.{:=} cur\_msg\_id \.{+} 1 {\p@semicolon}}%
\@x{\@s{16.4} {\p@end} {\p@macro} {\p@semicolon}}%
\@pvspace{8.0pt}%
\@x{}%
\@y{\@s{7.5}%
 A general macro for sending messages
}%
\@xx{}%
\@x{}%
\@y{\@s{7.5}%
 Non-equivocating and equivocating \ensuremath{validators} behave differently
}%
\@xx{}%
\@x{}%
\@y{\@s{7.5}%
 No honest \ensuremath{validator} will send the same message multiple times
 consecutively
}%
\@xx{}%
\@x{}%
\@y{\@s{7.5}%
 Equivocating \ensuremath{validators} may send different messages to
 different subsets of \ensuremath{validators} consecutively.
}%
\@xx{}%
\@x{\@s{16.4} {\p@macro} send\_message ( validator ) {\p@begin}}%
 \@x{\@s{32.8} {\p@if} ( cur\_msg\_id \.{>} 1 \.{\land} msg\_sender [
 cur\_msg\_id \.{-} 1 ] \.{\neq} validator ) \.{\lor} ( validator \.{\in}
 byzantine\_fault\_nodes ) {\p@then}}%
 \@x{\@s{47.23} {\p@if} validator \.{\notin} byzantine\_fault\_nodes
 {\p@then}}%
 \@x{\@s{61.87} make\_message ( validator ,\, binary\_estimator (
 validator\_received\_msg ( validator ) ) ,\, validator\_received\_msg (
 validator ) ) {\p@semicolon}}%
\@x{\@s{47.23} {\p@else}}%
\@x{\@s{63.63} make\_equivocating\_messages ( validator ) {\p@semicolon}}%
\@x{\@s{47.23} {\p@end} {\p@if} {\p@semicolon}}%
\@x{\@s{32.8} {\p@else}}%
\@x{\@s{65.6} {\p@skip} {\p@semicolon}}%
\@x{\@s{32.8} {\p@end} {\p@if} {\p@semicolon}}%
\@x{\@s{16.4} {\p@end} {\p@macro} {\p@semicolon}}%
\@pvspace{8.0pt}%
\@x{}%
\@y{\@s{7.5}%
 Each process is an individual \ensuremath{validator
}}%
\@xx{}%
\@x{}%
\@y{\@s{7.5}%
 Validators send messages in random orders
}%
\@xx{}%
\@x{}%
\@y{\@s{7.5}%
 Validators keep on sending messages until the maximum limit is reached
}%
\@xx{}%
\@x{\@s{16.4} {\p@fair} {\p@process} v \.{\in} validators {\p@begin}}%
\@x{\@s{32.8} Validate\@s{.5}\textrm{:}\@s{3}}%
\@x{\@s{32.8} {\p@while} cur\_msg\_id \.{\leq} message\_ids {\p@do}}%
 \@x{\@s{49.19} {\p@if} validator\_init\_done [ self ] \.{=} 0 \.{\land} self
 \.{\notin} byzantine\_fault\_nodes {\p@then}}%
\@x{\@s{63.84} init\_validator ( self ) {\p@semicolon}}%
\@x{\@s{49.19} {\p@else}}%
\@x{\@s{65.6} send\_message ( self ) {\p@semicolon}}%
\@x{\@s{49.19} {\p@end} {\p@if} {\p@semicolon}}%
\@x{\@s{32.8} {\p@end} {\p@while} {\p@semicolon}}%
\@x{\@s{16.4} {\p@end} {\p@process} {\p@semicolon}}%
\@pvspace{8.0pt}%
\@x{ {\p@end} {\p@algorithm}}%
\@y{%
 ; ***
}%
\@xx{}%
\pcalshadingfalse \pcalsymbolsfalse
\@pvspace{8.0pt}%
\@x{}%
\@y{\@s{0}%
 BEGIN TRANSLATION
}%
\@xx{}%
\@x{ {\CONSTANT} defaultInitValue}%
 \@x{ {\VARIABLES} all\_msg ,\, equivocating\_msg ,\, msg\_sender ,\,
 msg\_estimate ,\,}%
 \@x{\@s{51.42} msg\_justification ,\, cur\_msg\_id ,\, validator\_init\_done
 ,\,}%
\@x{\@s{51.42} equiv\_msg\_receivers ,\, cur\_subset ,\, pc}%
\@pvspace{8.0pt}%
\@x{}%
\@y{\@s{0}%
 define statement
}%
\@xx{}%
\@x{\@s{4.1} dependencies ( message ) \.{\defeq}}%
\@x{\@s{16.4} \.{\LET}}%
\@x{\@s{32.8} {\RECURSIVE} dep ( \_ )}%
\@x{\@s{32.8} dep ( msg ) \.{\defeq}}%
 \@x{\@s{52.14} {\IF} Cardinality ( msg\_justification [ msg ] ) \.{=} 1
 \.{\land} msg \.{\in} msg\_justification [ msg ]}%
\@x{\@s{52.14} \.{\THEN} \{ msg \}}%
 \@x{\@s{52.14} \.{\ELSE} {\UNION} \{ dep ( msg2 ) \.{:} msg2 \.{\in}
 msg\_justification [ msg ] \}}%
\@x{\@s{16.4} \.{\IN} dep ( message )}%
\@pvspace{16.0pt}%
\@x{ dep\_set ( messages ) \.{\defeq}}%
 \@x{\@s{16.4} messages \.{\cup} {\UNION} \{ dependencies ( m ) \.{:} m
 \.{\in} messages \}}%
\@pvspace{16.0pt}%
\@x{ latest\_message ( validator ,\, messages ) \.{\defeq}}%
\@x{\@s{16.4} \{ msg \.{\in} messages \.{:}}%
\@x{\@s{33.70} \.{\land} msg\_sender [ msg ] \.{=} validator}%
\@x{\@s{33.70} \.{\land} {\lnot} \E\, msg2 \.{\in} messages \.{:}}%
\@x{\@s{51.47} \.{\land} msg\_sender [ msg2 ] \.{=} validator}%
\@x{\@s{51.47} \.{\land} msg \.{\neq} msg2}%
\@x{\@s{51.47} \.{\land} msg \.{\in} dependencies ( msg2 )}%
\@x{\@s{16.4} \}}%
\@pvspace{16.0pt}%
\@x{ Pick ( S ) \.{\defeq} {\CHOOSE} s \.{\in} S \.{:} {\TRUE}}%
\@x{ {\RECURSIVE} SetReduce ( \_ ,\, \_ ,\, \_ )}%
\@x{\@s{16.4} SetReduce ( Op ( \_ ,\, \_ ) ,\, S ,\, value ) \.{\defeq}}%
\@x{\@s{32.8} {\IF} S \.{=} \{ \} \.{\THEN} value}%
 \@x{\@s{32.8} \.{\ELSE} \.{\LET} s \.{\defeq} Pick ( S ) \.{\IN} SetReduce (
 Op ,\, S \.{\,\backslash\,} \{ s \} ,\, Op ( s ,\, value ) )}%
\@pvspace{8.0pt}%
 \@x{\@s{16.4} Sum ( S ) \.{\defeq} \.{\LET} \_op ( a ,\, b ) \.{\defeq} a
 \.{+} b \.{\IN} SetReduce ( \_op ,\, S ,\, 0 )}%
\@pvspace{16.0pt}%
\@x{ score ( estimate ,\, messages ) \.{\defeq}}%
\@x{\@s{16.4} \.{\LET} ss \.{\defeq}}%
\@x{\@s{36.79} \{ v \.{\in} validators \.{:}}%
 \@x{\@s{51.57} \.{\land} Cardinality ( latest\_message ( v ,\, messages ) )
 \.{=} 1}%
 \@x{\@s{51.57} \.{\land} \E\, m \.{\in} latest\_message ( v ,\, messages )
 \.{:}}%
\@x{\@s{66.78} msg\_estimate [ m ] \.{=} estimate \}}%
 \@x{\@s{36.79} ss2 \.{\defeq} \{ validator\_weights [ v ] \.{:} v \.{\in} ss
 \}}%
\@x{\@s{16.4} \.{\IN} Sum ( ss2 )}%
\@pvspace{16.0pt}%
\@x{ binary\_estimator ( messages ) \.{\defeq}}%
\@x{\@s{16.4} {\IF} score ( 1 ,\, messages ) \.{>} score ( 0 ,\, messages )}%
\@x{\@s{16.4} \.{\THEN} 1}%
\@x{\@s{16.4} \.{\ELSE} 0}%
\@pvspace{16.0pt}%
\@x{ equivocation ( m1 ,\, m2 ) \.{\defeq}}%
\@x{\@s{16.4} \.{\land} m1 \.{\neq} m2}%
\@x{\@s{16.4} \.{\land} msg\_sender [ m1 ] \.{=} msg\_sender [ m2 ]}%
\@x{\@s{16.4} \.{\land} m1 \.{\notin} dependencies ( m2 )}%
\@x{\@s{16.4} \.{\land} m2 \.{\notin} dependencies ( m1 )}%
\@pvspace{16.0pt}%
\@x{ faulty\_node ( validator ,\, messages ) \.{\defeq}}%
\@x{\@s{16.4} \.{\land} \E\, m1 \.{\in} dep\_set ( messages ) \.{:}}%
\@x{\@s{31.61} \.{\land} \E\, m2 \.{\in} dep\_set ( messages ) \.{:}}%
\@x{\@s{46.82} \.{\land} validator \.{=} msg\_sender [ m1 ]}%
\@x{\@s{46.82} \.{\land} equivocation ( m1 ,\, m2 )}%
\@pvspace{16.0pt}%
\@x{ faulty\_nodes ( messages ) \.{\defeq}}%
 \@x{\@s{16.4} \{ v \.{\in} validators \.{:} faulty\_node ( v ,\, messages )
 \}}%
\@pvspace{16.0pt}%
\@x{ fault\_weight ( messages ) \.{\defeq}}%
\@x{\@s{16.4} \.{\LET} byz \.{\defeq} faulty\_nodes ( messages )}%
 \@x{\@s{16.4} \.{\IN} Sum ( \{ validator\_weights [ v ] \.{:} v \.{\in} byz
 \} )}%
\@pvspace{16.0pt}%
 \@x{ protocol\_states ( messages ,\, t ) \.{\defeq} \{ ss \.{\in} {\SUBSET} (
 messages ) \.{:} fault\_weight ( ss ) \.{<} t \}}%
 \@x{ protocol\_executions ( state1 ,\, state2 ) \.{\defeq} state1
 \.{\subseteq} state2}%
\@pvspace{16.0pt}%
\@x{ validators\_agreeing ( v1 ,\, v2 ,\, estimate ,\, messages ) \.{\defeq}}%
 \@x{\@s{16.4} \.{\land} Cardinality ( latest\_message ( v1 ,\, messages ) )
 \.{=} 1}%
 \@x{\@s{16.4} \.{\land} \.{\LET} v1\_latest\_msg \.{\defeq} {\CHOOSE} x
 \.{\in} latest\_message ( v1 ,\, messages ) \.{:} {\TRUE}}%
 \@x{\@s{31.61} \.{\IN} Cardinality ( latest\_message ( v2 ,\,
 msg\_justification [ v1\_latest\_msg ] ) ) \.{=} 1}%
 \@x{\@s{31.61} \.{\land}\@s{9.28} \.{\LET} v2\_latest\_msg \.{\defeq}
 {\CHOOSE} x \.{\in} latest\_message ( v2 ,\, msg\_justification [
 v1\_latest\_msg ] ) \.{:} {\TRUE}}%
\@x{\@s{56.11} \.{\IN} estimate \.{=} msg\_estimate [ v2\_latest\_msg ]}%
\@pvspace{16.0pt}%
 \@x{ validators\_disagreeing ( v1 ,\, v2 ,\, estimate ,\, messages )
 \.{\defeq}}%
 \@x{\@s{16.4} \.{\land} Cardinality ( latest\_message ( v1 ,\, messages ) )
 \.{=} 1}%
 \@x{\@s{16.4} \.{\land} \.{\LET} v1\_latest\_msg \.{\defeq} {\CHOOSE} x
 \.{\in} latest\_message ( v1 ,\, messages ) \.{:} {\TRUE}}%
 \@x{\@s{31.61} \.{\IN} Cardinality ( latest\_message ( v2 ,\,
 msg\_justification [ v1\_latest\_msg ] ) ) \.{=} 1}%
 \@x{\@s{31.61} \.{\land}\@s{9.28} \.{\LET} v2\_latest\_msg \.{\defeq}
 {\CHOOSE} x \.{\in} latest\_message ( v2 ,\, msg\_justification [
 v1\_latest\_msg ] ) \.{:} {\TRUE}}%
 \@x{\@s{56.11} \.{\IN} \E\, m \.{\in} messages \.{:} v2\_latest\_msg \.{\in}
 dependencies ( m )}%
\@x{\@s{80.61} \.{\land} estimate \.{\neq} msg\_estimate [ m ]}%
\@pvspace{16.0pt}%
\@x{ e\_clique ( estimate ,\, messages ) \.{\defeq} \{}%
\@x{\@s{16.4} ss \.{\in} {\SUBSET} ( validators ) \.{:}}%
\@x{\@s{29.49} \.{\land} \A\, v1 \.{\in} ss \.{:}}%
\@x{\@s{44.70} \.{\land} \A\, v2 \.{\in} ss \.{:}}%
\@x{\@s{59.92} {\IF} v1 \.{=} v2}%
\@x{\@s{59.92} \.{\THEN} {\TRUE}}%
\@x{\@s{59.92} \.{\ELSE}}%
 \@x{\@s{76.32} \.{\land} validators\_agreeing ( v1 ,\, v2 ,\, estimate ,\,
 messages )}%
 \@x{\@s{76.32} \.{\land} {\lnot} validators\_disagreeing ( v1 ,\, v2 ,\,
 estimate ,\, messages )}%
\@x{\@s{16.4} \}}%
\@pvspace{16.0pt}%
\@x{ e\_clique\_estimate\_safety ( estimate ,\, messages ) \.{\defeq}}%
 \@x{\@s{16.4} \.{\land} \E\, ss \.{\in} e\_clique ( estimate ,\, messages )
 \.{:}}%
 \@x{\@s{31.61} \.{\land} 2 \.{*} Sum ( \{ validator\_weights [ v ] \.{:} v
 \.{\in} ss \} ) \.{>} Sum ( \{ validator\_weights [ v ] \.{:} v \.{\in}
 validators \} ) \.{+} byzantine\_threshold \.{-} fault\_weight ( messages )}%
\@pvspace{16.0pt}%
\@x{ validator\_received\_msg ( validator ) \.{\defeq}}%
\@x{\@s{16.4} ( all\_msg \.{\,\backslash\,} equivocating\_msg ) \.{\cup}}%
 \@x{\@s{16.4} \{ x \.{\in} equivocating\_msg \.{:} validator \.{\in}
 equiv\_msg\_receivers [ x ] \}}%
\@pvspace{16.0pt}%
 \@x{ get\_equivocation\_subset\_msg ( validator ) \.{\defeq} {\CHOOSE} x
 \.{\in} {\SUBSET} ( validator\_received\_msg ( validator ) ) \.{:} {\TRUE}}%
\@pvspace{16.0pt}%
 \@x{ get\_equiv\_receivers ( validator ) \.{\defeq} {\CHOOSE} x \.{\in} (
 {\SUBSET} ( validators \.{\,\backslash\,} \{ validator \} ) ) \.{:} x
 \.{\neq} \{ \}}%
\@pvspace{16.0pt}%
\@x{ check\_safety\_with\_oracle \.{\defeq}}%
 \@x{\@s{16.4} \.{\LET} v \.{\defeq} {\CHOOSE} v \.{\in} ( validators
 \.{\,\backslash\,} byzantine\_fault\_nodes ) \.{:} {\TRUE}}%
 \@x{\@s{16.4} \.{\IN} {\Diamond} ( e\_clique\_estimate\_safety ( 0 ,\,
 validator\_received\_msg ( v ) ) \.{\lor} e\_clique\_estimate\_safety ( 1
 ,\, validator\_received\_msg ( v ) ) )}%
\@pvspace{16.0pt}%
 \@x{ vars \.{\defeq} {\langle} all\_msg ,\, equivocating\_msg ,\, msg\_sender
 ,\, msg\_estimate ,\,}%
 \@x{\@s{41.61} msg\_justification ,\, cur\_msg\_id ,\, validator\_init\_done
 ,\,}%
\@x{\@s{41.61} equiv\_msg\_receivers ,\, cur\_subset ,\, pc {\rangle}}%
\@pvspace{8.0pt}%
\@x{ ProcSet \.{\defeq} ( validators )}%
\@pvspace{8.0pt}%
\@x{ Init \.{\defeq}}%
\@y{\@s{0}%
 Global variables
}%
\@xx{}%
\@x{\@s{35.70} \.{\land} all\_msg \.{=} \{ \}}%
\@x{\@s{35.70} \.{\land} equivocating\_msg \.{=} \{ \}}%
\@x{\@s{35.70} \.{\land} msg\_sender \.{=} {\langle} {\rangle}}%
\@x{\@s{35.70} \.{\land} msg\_estimate \.{=} {\langle} {\rangle}}%
\@x{\@s{35.70} \.{\land} msg\_justification \.{=} {\langle} {\rangle}}%
\@x{\@s{35.70} \.{\land} cur\_msg\_id \.{=} 1}%
 \@x{\@s{35.70} \.{\land} validator\_init\_done\@s{1.92} \.{=} {\langle} 0 ,\,
 0 ,\, 0 ,\, 0 ,\, 0 ,\, 0 ,\, 0 ,\, 0 ,\, 0 ,\, 0 {\rangle}}%
\@x{\@s{35.70} \.{\land} equiv\_msg\_receivers \.{=} {\langle} {\rangle}}%
\@x{\@s{35.70} \.{\land} cur\_subset \.{=} defaultInitValue}%
 \@x{\@s{35.70} \.{\land} pc \.{=} [ self\@s{4.46} \.{\in} ProcSet
 \.{\mapsto}\@w{Validate} ]}%
\@pvspace{8.0pt}%
\@x{ Validate ( self ) \.{\defeq} \.{\land} pc [ self ] \.{=}\@w{Validate}}%
\@x{\@s{79.39} \.{\land} {\IF} cur\_msg\_id \.{\leq} message\_ids}%
 \@x{\@s{102.65} \.{\THEN} \.{\land} {\IF} validator\_init\_done [ self ]
 \.{=} 0 \.{\land} self \.{\notin} byzantine\_fault\_nodes}%
 \@x{\@s{157.23} \.{\THEN} \.{\land} equiv\_msg\_receivers \.{'} \.{=} Append
 ( equiv\_msg\_receivers ,\, \{ \} )}%
 \@x{\@s{188.54} \.{\land} msg\_sender \.{'} \.{=} Append ( msg\_sender ,\,
 self )}%
 \@x{\@s{188.54} \.{\land} msg\_estimate \.{'} \.{=} Append ( msg\_estimate
 ,\, ( validators\_initial\_values [ self ] ) )}%
 \@x{\@s{188.54} \.{\land} msg\_justification \.{'} \.{=} Append (
 msg\_justification ,\, ( \{ cur\_msg\_id \} ) )}%
 \@x{\@s{188.54} \.{\land} all\_msg \.{'} \.{=} ( all\_msg \.{\cup} \{
 cur\_msg\_id \} )}%
\@x{\@s{188.54} \.{\land} cur\_msg\_id \.{'} \.{=} cur\_msg\_id \.{+} 1}%
 \@x{\@s{188.54} \.{\land} validator\_init\_done \.{'} \.{=} [
 validator\_init\_done {\EXCEPT} {\bang} [ self ] \.{=} 1 ]}%
\@x{\@s{188.54} \.{\land} {\UNCHANGED} {\langle} equivocating\_msg ,\,}%
\@x{\@s{262.13} cur\_subset {\rangle}}%
 \@x{\@s{157.23} \.{\ELSE} \.{\land} {\IF} ( cur\_msg\_id \.{>} 1 \.{\land}
 msg\_sender [ cur\_msg\_id \.{-} 1 ] \.{\neq} self ) \.{\lor} ( self \.{\in}
 byzantine\_fault\_nodes )}%
 \@x{\@s{211.81} \.{\THEN} \.{\land} {\IF} self \.{\notin}
 byzantine\_fault\_nodes}%
 \@x{\@s{266.39} \.{\THEN} \.{\land} equiv\_msg\_receivers \.{'} \.{=} Append
 ( equiv\_msg\_receivers ,\, \{ \} )}%
 \@x{\@s{297.70} \.{\land} msg\_sender \.{'} \.{=} Append ( msg\_sender ,\,
 self )}%
 \@x{\@s{297.70} \.{\land} msg\_estimate \.{'} \.{=} Append ( msg\_estimate
 ,\, ( binary\_estimator ( validator\_received\_msg ( self ) ) ) )}%
 \@x{\@s{297.70} \.{\land} msg\_justification \.{'} \.{=} Append (
 msg\_justification ,\, ( validator\_received\_msg ( self ) ) )}%
 \@x{\@s{297.70} \.{\land} all\_msg \.{'} \.{=} ( all\_msg \.{\cup} \{
 cur\_msg\_id \} )}%
\@x{\@s{297.70} \.{\land} cur\_msg\_id \.{'} \.{=} cur\_msg\_id \.{+} 1}%
\@x{\@s{297.70} \.{\land} {\UNCHANGED} {\langle} equivocating\_msg ,\,}%
\@x{\@s{371.29} cur\_subset {\rangle}}%
 \@x{\@s{266.39} \.{\ELSE} \.{\land} equiv\_msg\_receivers \.{'} \.{=} Append
 ( equiv\_msg\_receivers ,\, get\_equiv\_receivers ( self ) )}%
 \@x{\@s{297.70} \.{\land} cur\_subset \.{'} \.{=}
 get\_equivocation\_subset\_msg ( self )}%
 \@x{\@s{297.70} \.{\land} equivocating\_msg \.{'} \.{=} ( equivocating\_msg
 \.{\cup} \{ cur\_msg\_id \} )}%
 \@x{\@s{297.70} \.{\land} msg\_sender \.{'} \.{=} Append ( msg\_sender ,\,
 self )}%
 \@x{\@s{297.70} \.{\land} msg\_estimate \.{'} \.{=} Append ( msg\_estimate
 ,\, binary\_estimator ( cur\_subset \.{'} ) )}%
 \@x{\@s{297.70} \.{\land} msg\_justification \.{'} \.{=} Append (
 msg\_justification ,\, cur\_subset \.{'} )}%
 \@x{\@s{297.70} \.{\land} all\_msg \.{'} \.{=} ( all\_msg \.{\cup} \{
 cur\_msg\_id \} )}%
\@x{\@s{297.70} \.{\land} cur\_msg\_id \.{'} \.{=} cur\_msg\_id \.{+} 1}%
\@x{\@s{211.81} \.{\ELSE} \.{\land} {\TRUE}}%
\@x{\@s{243.12} \.{\land} {\UNCHANGED} {\langle} all\_msg ,\,}%
\@x{\@s{316.71} equivocating\_msg ,\,}%
\@x{\@s{316.71} msg\_sender ,\,}%
\@x{\@s{316.71} msg\_estimate ,\,}%
\@x{\@s{316.71} msg\_justification ,\,}%
\@x{\@s{316.71} cur\_msg\_id ,\,}%
\@x{\@s{316.71} equiv\_msg\_receivers ,\,}%
\@x{\@s{316.71} cur\_subset {\rangle}}%
\@x{\@s{188.54} \.{\land} {\UNCHANGED} validator\_init\_done}%
 \@x{\@s{133.96} \.{\land} pc \.{'} \.{=} [ pc {\EXCEPT} {\bang} [ self ]
 \.{=}\@w{Validate} ]}%
 \@x{\@s{102.65} \.{\ELSE} \.{\land} pc \.{'} \.{=} [ pc {\EXCEPT} {\bang} [
 self ] \.{=}\@w{Done} ]}%
 \@x{\@s{133.96} \.{\land} {\UNCHANGED} {\langle} all\_msg ,\,
 equivocating\_msg ,\,}%
\@x{\@s{207.55} msg\_sender ,\, msg\_estimate ,\,}%
\@x{\@s{207.55} msg\_justification ,\, cur\_msg\_id ,\,}%
\@x{\@s{207.55} validator\_init\_done ,\,}%
\@x{\@s{207.55} equiv\_msg\_receivers ,\, cur\_subset {\rangle}}%
\@pvspace{8.0pt}%
\@x{ v ( self ) \.{\defeq} Validate ( self )}%
\@pvspace{8.0pt}%
\@x{ Next \.{\defeq} ( \E\, self \.{\in} validators \.{:} v ( self ) )}%
\@x{\@s{47.82} \.{\lor}}%
\@y{\@s{0}%
 Disjunct to prevent deadlock on termination
}%
\@xx{}%
 \@x{\@s{58.93} ( ( \A\, self \.{\in} ProcSet \.{:} pc [ self ] \.{=}\@w{Done}
 ) \.{\land} {\UNCHANGED} vars )}%
\@pvspace{8.0pt}%
\@x{ Spec\@s{1.46} \.{\defeq} \.{\land} Init}%
 \@x{\@s{39.83} \.{\land} {\Box} [ Next ]_{ vars} \.{\defeq} {\Box} ( Next
 \.{\lor} {\UNCHANGED} vars )}%
 \@x{\@s{39.83} \.{\land} \A\, self \.{\in} validators \.{:} {\WF}_{ vars} ( v
 ( self ) )}%
\@pvspace{8.0pt}%
 \@x{ Termination \.{\defeq} {\Diamond} ( \A\, self \.{\in} ProcSet \.{:} pc [
 self ] \.{=}\@w{Done} )}%
\@pvspace{8.0pt}%
\@x{}%
\@y{\@s{0}%
 END TRANSLATION
}%
\@xx{}%
\@pvspace{8.0pt}%
\@x{}\bottombar\@xx{}%
\setboolean{shading}{false}
\begin{lcom}{0}%
\begin{cpar}{0}{F}{F}{0}{0}{}%
\ensuremath{\.{\,\backslash\,}\.{*}} Modification History
\end{cpar}%
\begin{cpar}{0}{F}{F}{0}{0}{}%
 \ensuremath{\.{\,\backslash\,}\.{*}} Last modified \ensuremath{Fri}
 \ensuremath{Nov} 22 11:21:27 \ensuremath{EST} 2019 by \ensuremath{isaac
}%
\end{cpar}%
\begin{cpar}{0}{F}{F}{0}{0}{}%
 \ensuremath{\.{\,\backslash\,}\.{*}} Created \ensuremath{Tue}
 \ensuremath{Nov} 19 11:24:16 \ensuremath{EST} 2019 by \ensuremath{isaac
}%
\end{cpar}%
\end{lcom}%
\end{document}
